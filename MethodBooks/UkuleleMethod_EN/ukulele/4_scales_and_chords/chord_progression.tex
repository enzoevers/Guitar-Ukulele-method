\section{Chord progression}

\subsection{Chords in a scale}

A chord progression of chords. For example, if you would play the chords C and G repeatedly for (part of) a song then the chord progression would be C - G. However, assuming that we are in the key of C major and therefore the C would be the root note/chord, this chord progression isn't universal. For that reason chord progressions are often referred to with numbers (notated with roman numerals).

Using numbers the chord progression C - G would become 1 - 5, or with roman numerals \RomanNumeralCaps{1} - \RomanNumeralCaps{5}. The indexes in the major scale can be shown using the roman numerals as well. See \autoref{tab:ukulele_major_scale_chords}. The chord and the 7th index (\RomanNumeral{7}\textsuperscript{o}) is called a diminished chord, but we will ignore this for now. The 8th index is not assigned a chord because it's the same as index 1.

Two things to note:

\begin{itemize}
	\item \textbf{Capitalized} roman numerals correspond to \textbf{major} chords
	\item \textbf{Non-capitalized} roman numerals correspond to \textbf{minor} chords
\end{itemize}

\begin{table}[h]
	\centering
	\begin{tabular}{*{16}{c}}
		& \multicolumn{2}{P{4mm}}{\large{W}} & \multicolumn{2}{P{4mm}}{\large{W}} & \multicolumn{2}{P{4mm}}{\large{H}} & \multicolumn{2}{P{4mm}}{\large{W}} & \multicolumn{2}{P{4mm}}{\large{W}} & \multicolumn{2}{P{4mm}}{\large{W}} & \multicolumn{2}{P{4mm}}{\large{H}} & \\
		\multicolumn{2}{P{4mm}}{1} & \multicolumn{2}{P{4mm}}{2} & \multicolumn{2}{P{4mm}}{3} & \multicolumn{2}{P{4mm}}{4} & \multicolumn{2}{P{4mm}}{5} & \multicolumn{2}{P{4mm}}{6} & \multicolumn{2}{P{4mm}}{7} & \multicolumn{2}{P{4mm}}{8} \\
		\multicolumn{2}{P{4mm}}{\RomanNumeralCaps{1}} & \multicolumn{2}{P{4mm}}{\RomanNumeral{2}} & \multicolumn{2}{P{4mm}}{\RomanNumeral{3}} & \multicolumn{2}{P{4mm}}{\RomanNumeralCaps{4}} & \multicolumn{2}{P{4mm}}{\RomanNumeralCaps{5}} & \multicolumn{2}{P{4mm}}{\RomanNumeral{6}} & \multicolumn{2}{P{4mm}}{\RomanNumeral{7}\textsuperscript{o}} &
	\end{tabular}
	\caption{Chords in the major scale}
	\label{tab:ukulele_major_scale_chords}
\end{table}

The same can be done for the minor scale (\autoref{tab:ukulele_minor_scale_chords}).

\begin{table}[h]
	\centering
	\begin{tabular}{*{16}{c}}
		& \multicolumn{2}{P{4mm}}{\large{W}} & \multicolumn{2}{P{4mm}}{\large{H}} & \multicolumn{2}{P{4mm}}{\large{W}} & \multicolumn{2}{P{4mm}}{\large{W}} & \multicolumn{2}{P{4mm}}{\large{H}} & \multicolumn{2}{P{4mm}}{\large{W}} & \multicolumn{2}{P{4mm}}{\large{W}} & \\
		\multicolumn{2}{P{4mm}}{1} & \multicolumn{2}{P{4mm}}{2} & \multicolumn{2}{P{4mm}}{3} & \multicolumn{2}{P{4mm}}{4} & \multicolumn{2}{P{4mm}}{5} & \multicolumn{2}{P{4mm}}{6} & \multicolumn{2}{P{4mm}}{7} & \multicolumn{2}{P{4mm}}{8}  \\
		\multicolumn{2}{P{4mm}}{\RomanNumeral{1}} & \multicolumn{2}{P{4mm}}{\RomanNumeral{2}\textsuperscript{o}} & \multicolumn{2}{P{4mm}}{\RomanNumeralCaps{3}} & \multicolumn{2}{P{4mm}}{\RomanNumeral{4}} & \multicolumn{2}{P{4mm}}{\RomanNumeral{5}} & \multicolumn{2}{P{4mm}}{\RomanNumeralCaps{6}} & \multicolumn{2}{P{4mm}}{\RomanNumeralCaps{7}} &
	\end{tabular}
	\caption{Chords in the minor scale}
	\label{tab:ukulele_minor_scale_chords}
\end{table}

How and why some chords are major and others are minor will be explained later. Lets first see how these can be used.

\subsection{Analyzing "Knockin' On Heaven's Door - Bob Dylan"}

In the previous section you played "Knockin' On Heaven's Door" using the G, D, C, and Am chords. There were two alternating sequences:

\begin{itemize}
	\item G - D - Am
	\item G - D - C
\end{itemize}
 
The song is in the G major key. The G major scale is shown again in \autoref{tab:ukulele_g_major_scale_with_chords}. The bold notes are the chords from "Knockin' On Heaven's Door".

\begin{table}[h]
	\centering
	\begin{tabular}{*{16}{c}}
		& \multicolumn{2}{P{4mm}}{\large{W}} & \multicolumn{2}{P{4mm}}{\large{W}} & \multicolumn{2}{P{4mm}}{\large{H}} & \multicolumn{2}{P{4mm}}{\large{W}} & \multicolumn{2}{P{4mm}}{\large{W}} & \multicolumn{2}{P{4mm}}{\large{W}} & \multicolumn{2}{P{4mm}}{\large{H}} & \\
		\multicolumn{2}{P{4mm}}{1} & \multicolumn{2}{P{4mm}}{2} & \multicolumn{2}{P{4mm}}{3} & \multicolumn{2}{P{4mm}}{4} & \multicolumn{2}{P{4mm}}{5} & \multicolumn{2}{P{4mm}}{6} & \multicolumn{2}{P{4mm}}{7} & \multicolumn{2}{P{4mm}}{8} \\
		\multicolumn{2}{P{4mm}}{\RomanNumeralCaps{1}} & \multicolumn{2}{P{4mm}}{\RomanNumeral{2}} & \multicolumn{2}{P{4mm}}{\RomanNumeral{3}} & \multicolumn{2}{P{4mm}}{\RomanNumeralCaps{4}} & \multicolumn{2}{P{4mm}}{\RomanNumeralCaps{5}} & \multicolumn{2}{P{4mm}}{\RomanNumeral{6}} & \multicolumn{2}{P{4mm}}{\RomanNumeral{7}\textsuperscript{o}} & \\
		\multicolumn{2}{P{4mm}}{\textbf{G}} & \multicolumn{2}{P{4mm}}{\textbf{A}} & \multicolumn{2}{P{4mm}}{B} & \multicolumn{2}{P{4mm}}{\textbf{C}} & \multicolumn{2}{P{4mm}}{\textbf{D}} & \multicolumn{2}{P{4mm}}{E} & \multicolumn{2}{P{4mm}}{F\sharp} & \multicolumn{2}{P{4mm}}{G}
	\end{tabular}
	\caption{G major scale with chords}
	\label{tab:ukulele_g_major_scale_with_chords}
\end{table}

With this knowledge, the chords sequences can be described in terms of roman numerals instead of chords. 

\begin{itemize}
	\item G - D - Am: \RomanNumeralCaps{1} - \RomanNumeralCaps{5} - \RomanNumeral{2}
	\item G - D - C: \RomanNumeralCaps{1} - \RomanNumeralCaps{5} - \RomanNumeralCaps{4} 
\end{itemize}

\subsubsection{Changing key}

Lets say that your singer is more comfortable in a different key, or you just want to play the song in a different key for whatever reason. Then you can change the key of the song.

As an example, "Knockin' On Heaven's Door" will be changed to the D major key. By using the intervals that we've seen, and applying them to the D major key, we get a new sequence of chords.

\begin{itemize}
	\item D - A - Em: \RomanNumeralCaps{1} - \RomanNumeralCaps{5} - \RomanNumeral{2}
	\item D - A - G: \RomanNumeralCaps{1} - \RomanNumeralCaps{5} - \RomanNumeralCaps{4} 
\end{itemize}

\begin{table}[h]
	\centering
	\begin{tabular}{*{16}{c}}
		& \multicolumn{2}{P{4mm}}{\large{W}} & \multicolumn{2}{P{4mm}}{\large{W}} & \multicolumn{2}{P{4mm}}{\large{H}} & \multicolumn{2}{P{4mm}}{\large{W}} & \multicolumn{2}{P{4mm}}{\large{W}} & \multicolumn{2}{P{4mm}}{\large{W}} & \multicolumn{2}{P{4mm}}{\large{H}} & \\
		\multicolumn{2}{P{4mm}}{1} & \multicolumn{2}{P{4mm}}{2} & \multicolumn{2}{P{4mm}}{3} & \multicolumn{2}{P{4mm}}{4} & \multicolumn{2}{P{4mm}}{5} & \multicolumn{2}{P{4mm}}{6} & \multicolumn{2}{P{4mm}}{7} & \multicolumn{2}{P{4mm}}{8} \\
		\multicolumn{2}{P{4mm}}{\RomanNumeralCaps{1}} & \multicolumn{2}{P{4mm}}{\RomanNumeral{2}} & \multicolumn{2}{P{4mm}}{\RomanNumeral{3}} & \multicolumn{2}{P{4mm}}{\RomanNumeralCaps{4}} & \multicolumn{2}{P{4mm}}{\RomanNumeralCaps{5}} & \multicolumn{2}{P{4mm}}{\RomanNumeral{6}} & \multicolumn{2}{P{4mm}}{\RomanNumeral{7}\textsuperscript{o}} & \\
		\multicolumn{2}{P{4mm}}{\textbf{D}} & \multicolumn{2}{P{4mm}}{\textbf{E}} & \multicolumn{2}{P{4mm}}{F\sharp} & \multicolumn{2}{P{4mm}}{\textbf{G}} & \multicolumn{2}{P{4mm}}{\textbf{A}} & \multicolumn{2}{P{4mm}}{B} & \multicolumn{2}{P{4mm}}{C\sharp} & \multicolumn{2}{P{4mm}}{D}
	\end{tabular}
	\caption{D major scale with chords}
	\label{tab:ukulele_D_major_scale_with_chords}
\end{table}

\subsection{Analyzing "Perfect - Ed Sheeran"}

This song is in the key of A$\flat$ (\autoref{tab:ukulele_a_flat_major_scale_with_chords}). The reason that A$\flat$ is used instead of G$\sharp$ is because A$\flat$ uses more natural notes.

\begin{table}[h]
	\centering
	\begin{tabular}{*{16}{c}}
		& \multicolumn{2}{P{4mm}}{\large{W}} & \multicolumn{2}{P{4mm}}{\large{W}} & \multicolumn{2}{P{4mm}}{\large{H}} & \multicolumn{2}{P{4mm}}{\large{W}} & \multicolumn{2}{P{4mm}}{\large{W}} & \multicolumn{2}{P{4mm}}{\large{W}} & \multicolumn{2}{P{4mm}}{\large{H}} & \\
		\multicolumn{2}{P{4mm}}{1} & \multicolumn{2}{P{4mm}}{2} & \multicolumn{2}{P{4mm}}{3} & \multicolumn{2}{P{4mm}}{4} & \multicolumn{2}{P{4mm}}{5} & \multicolumn{2}{P{4mm}}{6} & \multicolumn{2}{P{4mm}}{7} & \multicolumn{2}{P{4mm}}{8} \\
		\multicolumn{2}{P{4mm}}{\RomanNumeralCaps{1}} & \multicolumn{2}{P{4mm}}{\RomanNumeral{2}} & \multicolumn{2}{P{4mm}}{\RomanNumeral{3}} & \multicolumn{2}{P{4mm}}{\RomanNumeralCaps{4}} & \multicolumn{2}{P{4mm}}{\RomanNumeralCaps{5}} & \multicolumn{2}{P{4mm}}{\RomanNumeral{6}} & \multicolumn{2}{P{4mm}}{\RomanNumeral{7}\textsuperscript{o}} & \\
		\multicolumn{2}{P{4mm}}{\textbf{A\flat}} & \multicolumn{2}{P{4mm}}{B\flat} & \multicolumn{2}{P{4mm}}{C} & \multicolumn{2}{P{4mm}}{\textbf{D\flat}} & \multicolumn{2}{P{4mm}}{\textbf{E\flat}} & \multicolumn{2}{P{4mm}}{\textbf{F}} & \multicolumn{2}{P{4mm}}{G} & \multicolumn{2}{P{4mm}}{A\flat}
	\end{tabular}
	\caption{A$\flat$ major scale with chords}
	\label{tab:ukulele_a_flat_major_scale_with_chords}
\end{table}

In the first verse of this song (shown in the previously) the following chord progression is used:

\begin{itemize}
	\item A$\flat$ - Fm - D$\flat$ - E$\flat$: \RomanNumeralCaps{1} - \RomanNumeral{6} - \RomanNumeralCaps{4} - \RomanNumeralCaps{5}
\end{itemize}

\subsection{Deviating from the scale chords}

Of course, the things described so far are a good starting point, but music is a creative endeavor after all. So feel free to experiment.

The song "Hey Ya!" from Outkast did that a bit as well. The song is in the key of G major. The chords sequence played through the whole song is:
\begin{itemize}
	\item G - C - D - E
\end{itemize}

Strictly speaking this doesn't fit in the chords of the G major scale. You would expect an Em chord instead of an E (see \autoref{tab:ukulele_g_major_scale_with_chords}).

% Knocking on heaven's door - bob dylan https://tabs.ultimate-guitar.com/tab/bob-dylan/knockin-on-heavens-door-chords-66559
% America - A Horse With No Name (1972) - Simple, but interesting chords https://tabs.ultimate-guitar.com/tab/america/a-horse-with-no-name-chords-59609
% Outkast - Hey Ya! https://tabs.ultimate-guitar.com/tab/outkast/hey-ya-chords-90698
% Narcotic Chords by Liquido