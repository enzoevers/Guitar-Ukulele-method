\documentclass{book}
\usepackage[10pt]{extsizes}
\usepackage[T1]{fontenc} % higher quality font encoding
\usepackage[sfdefault]{universalis} % load the font and set it to default
% Language setting
% Replace `english' with e.g. `spanish' to change the document language
\usepackage[english]{babel}
\usepackage[toc,page]{appendix}

\usepackage[parfill]{parskip}

\usepackage{titlesec}
\titleformat{\chapter}{\normalfont\huge\bfseries}{\thechapter}{1em}{}

% Set page size and margins
% Replace `letterpaper' with `a4paper' for UK/EU standard size
\usepackage[a4paper,top=2cm,bottom=2cm,left=3cm,right=3cm,marginparwidth=1.75cm]{geometry}

\usepackage{pdfpages}

% Useful packages
\usepackage{amsmath}
\usepackage{graphicx}
\usepackage[colorlinks=false]{hyperref}
\usepackage{subcaption}
\usepackage{wrapfig}
\usepackage{tcolorbox}

\newcommand{\infobox}[1]
{
    \begin{tcolorbox}[colback=orange!5!white, colframe=orange!75!black]
        \begin{minipage}{0.1\textwidth}
            \centering
            \includegraphics[width=\textwidth]{../../Images/attention-43524-svgrepo-com.png}
        \end{minipage}
        \hfill
        \begin{minipage}{0.85\textwidth}
            #1
        \end{minipage}
    \end{tcolorbox}
}

\usepackage{array}
\newcolumntype{P}[1]{>{\centering\arraybackslash}p{#1}}

\usepackage{fontspec}
\usepackage{lilyglyphs}

\usepackage[backend=biber]{biblatex}
\addbibresource{refs.bib}

\title{Strings? Why not!}
\author{Enzo Evers}

\begin{document}
\maketitle

\tableofcontents

\chapter{Getting ready to play}

\section{Sitting/standing position}

\begin{figure}[h]
  \begin{subfigure}[b]{0.45\textwidth}
    \includegraphics[width=\textwidth]{../../Images/Letty_Guitar-Shooting_sitting.jpg}
    \caption{}
    \label{fig:positin_sitting}
  \end{subfigure}
  \hfill
  \begin{subfigure}[b]{0.45\textwidth}
    \includegraphics[width=\textwidth]{../../Images/Letty_Guitar-Shooting_standing.jpg}
    \caption{}
    \label{fig:positin_standing}
  \end{subfigure}
  \caption{\cite{SitStandPosition}}
  \label{fig:positin}
\end{figure}

\infobox{This method assumes a right-handed player. If you are left-handed, replace “right” with “left” and vice versa.}

Even though it may look cooler to place the guitar on your right leg. You will be more comfortable and precise when you are sitting the classical way. The classical way of sitting also translates better to a standing position (see \ref{fig:positin_standing}).

In the classical position you place the guitar on your left leg and the left leg will be slightly raised. You can use a foot stool for this (see the left foot in \ref{fig:positin_sitting}).
\newpage
\section{Stemmen}

Je gitaar moet gestemd zijn. Dit betekent dat elke snaar een bepaalde toonhoogte heeft.

In \autoref{fig:guitar_string_names} zie je de namen (letters) van de dunste (\textit{e}) tot de dikste (\textit{E}) snaar.

\begin{figure}[h]
    \centering
    \includegraphics[width=0.8\textwidth]{../../Images/guitar-neck-string-names.png}
    \caption{Namen van de snaren van de gitaar}
    \label{fig:guitar_string_names}
\end{figure}

Een ezelsbruggetje is (van hoog/dun naar laag/dik):

\begin{minipage}{0.25\textwidth}
    \vspace{3mm}
    \begin{itemize}
        \setlength\itemsep{0em}
        \item[6)] \textbf{E} en
        \item[5)] \textbf{A} ap
        \item[4)] \textbf{D} ie
        \item[3)] \textbf{G} een
        \item[2)] \textbf{B} ananen
        \item[1)] \textbf{e} et 
    \end{itemize}
    \vspace{3mm}
\end{minipage}
\hfill
\begin{minipage}{0.7\textwidth}
    \infobox{Let op: dit is de standaardstemming. Soms wordt de gitaar anders gestemd. Dat wordt dan expliciet vermeld.}
\end{minipage}

\begin{minipage}{0.5\textwidth}
Je gebruikt een stemapparaat om te stemmen (zie \autoref{fig:tuning}). Het stemapparaat geeft een nootnaam aan, en dan moet je omhoog of omlaag stemmen tot de juiste noot op het scherm verschijnt. Of het toont een snaarnummer en je moet de 'wijzer' in het midden krijgen. Dat is afhankelijk van het type stemapparaat dat je hebt.

Wees voorzichtig met het omhoog stemmen (naar een hogere toon). Vooral de dunne snaren kunnen breken als ze te strak staan.
\end{minipage}
\hfill
\begin{minipage}{0.3\textwidth}
    \centering
    \includegraphics[width=\textwidth]{../../Images/guitar-tuning.jpg}
    \captionof{figure}{Een stemapparaat gebruiken op een gitaar}
    \label{fig:tuning}
\end{minipage}

Een andere stemmethode is gebaseerd op het verschil in toonhoogte tussen de snaren. In \autoref{fig:guitar_relative_tuning} zie je welke posities op de hals dezelfde toon hebben als de dunnere snaar eronder.

\begin{figure}[h]
    \centering
    \includegraphics[width=0.4\textwidth]{../../Images/GuitarRelativeTuning.png}
    \caption{Relatief stemmen}
    \label{fig:guitar_relative_tuning}
\end{figure}


\newpage
\chapter{Getting ready to play}

\section{Sitting/standing position}

\begin{figure}[h]
  \begin{subfigure}[b]{0.45\textwidth}
    \includegraphics[width=\textwidth]{../../Images/Letty_Guitar-Shooting_sitting.jpg}
    \caption{}
    \label{fig:positin_sitting}
  \end{subfigure}
  \hfill
  \begin{subfigure}[b]{0.45\textwidth}
    \includegraphics[width=\textwidth]{../../Images/Letty_Guitar-Shooting_standing.jpg}
    \caption{}
    \label{fig:positin_standing}
  \end{subfigure}
  \caption{\cite{SitStandPosition}}
  \label{fig:positin}
\end{figure}

\infobox{This method assumes a right-handed player. If you are left-handed, replace “right” with “left” and vice versa.}

Even though it may look cooler to place the guitar on your right leg. You will be more comfortable and precise when you are sitting the classical way. The classical way of sitting also translates better to a standing position (see \ref{fig:positin_standing}).

In the classical position you place the guitar on your left leg and the left leg will be slightly raised. You can use a foot stool for this (see the left foot in \ref{fig:positin_sitting}).
\newpage
\section{Stemmen}

Je gitaar moet gestemd zijn. Dit betekent dat elke snaar een bepaalde toonhoogte heeft.

In \autoref{fig:guitar_string_names} zie je de namen (letters) van de dunste (\textit{e}) tot de dikste (\textit{E}) snaar.

\begin{figure}[h]
    \centering
    \includegraphics[width=0.8\textwidth]{../../Images/guitar-neck-string-names.png}
    \caption{Namen van de snaren van de gitaar}
    \label{fig:guitar_string_names}
\end{figure}

Een ezelsbruggetje is (van hoog/dun naar laag/dik):

\begin{minipage}{0.25\textwidth}
    \vspace{3mm}
    \begin{itemize}
        \setlength\itemsep{0em}
        \item[6)] \textbf{E} en
        \item[5)] \textbf{A} ap
        \item[4)] \textbf{D} ie
        \item[3)] \textbf{G} een
        \item[2)] \textbf{B} ananen
        \item[1)] \textbf{e} et 
    \end{itemize}
    \vspace{3mm}
\end{minipage}
\hfill
\begin{minipage}{0.7\textwidth}
    \infobox{Let op: dit is de standaardstemming. Soms wordt de gitaar anders gestemd. Dat wordt dan expliciet vermeld.}
\end{minipage}

\begin{minipage}{0.5\textwidth}
Je gebruikt een stemapparaat om te stemmen (zie \autoref{fig:tuning}). Het stemapparaat geeft een nootnaam aan, en dan moet je omhoog of omlaag stemmen tot de juiste noot op het scherm verschijnt. Of het toont een snaarnummer en je moet de 'wijzer' in het midden krijgen. Dat is afhankelijk van het type stemapparaat dat je hebt.

Wees voorzichtig met het omhoog stemmen (naar een hogere toon). Vooral de dunne snaren kunnen breken als ze te strak staan.
\end{minipage}
\hfill
\begin{minipage}{0.3\textwidth}
    \centering
    \includegraphics[width=\textwidth]{../../Images/guitar-tuning.jpg}
    \captionof{figure}{Een stemapparaat gebruiken op een gitaar}
    \label{fig:tuning}
\end{minipage}

Een andere stemmethode is gebaseerd op het verschil in toonhoogte tussen de snaren. In \autoref{fig:guitar_relative_tuning} zie je welke posities op de hals dezelfde toon hebben als de dunnere snaar eronder.

\begin{figure}[h]
    \centering
    \includegraphics[width=0.4\textwidth]{../../Images/GuitarRelativeTuning.png}
    \caption{Relatief stemmen}
    \label{fig:guitar_relative_tuning}
\end{figure}


\newpage
\chapter{Getting ready to play}

\section{Sitting/standing position}

\begin{figure}[h]
  \begin{subfigure}[b]{0.45\textwidth}
    \includegraphics[width=\textwidth]{../../Images/Letty_Guitar-Shooting_sitting.jpg}
    \caption{}
    \label{fig:positin_sitting}
  \end{subfigure}
  \hfill
  \begin{subfigure}[b]{0.45\textwidth}
    \includegraphics[width=\textwidth]{../../Images/Letty_Guitar-Shooting_standing.jpg}
    \caption{}
    \label{fig:positin_standing}
  \end{subfigure}
  \caption{\cite{SitStandPosition}}
  \label{fig:positin}
\end{figure}

\infobox{This method assumes a right-handed player. If you are left-handed, replace “right” with “left” and vice versa.}

Even though it may look cooler to place the guitar on your right leg. You will be more comfortable and precise when you are sitting the classical way. The classical way of sitting also translates better to a standing position (see \ref{fig:positin_standing}).

In the classical position you place the guitar on your left leg and the left leg will be slightly raised. You can use a foot stool for this (see the left foot in \ref{fig:positin_sitting}).
\newpage
\section{Stemmen}

Je gitaar moet gestemd zijn. Dit betekent dat elke snaar een bepaalde toonhoogte heeft.

In \autoref{fig:guitar_string_names} zie je de namen (letters) van de dunste (\textit{e}) tot de dikste (\textit{E}) snaar.

\begin{figure}[h]
    \centering
    \includegraphics[width=0.8\textwidth]{../../Images/guitar-neck-string-names.png}
    \caption{Namen van de snaren van de gitaar}
    \label{fig:guitar_string_names}
\end{figure}

Een ezelsbruggetje is (van hoog/dun naar laag/dik):

\begin{minipage}{0.25\textwidth}
    \vspace{3mm}
    \begin{itemize}
        \setlength\itemsep{0em}
        \item[6)] \textbf{E} en
        \item[5)] \textbf{A} ap
        \item[4)] \textbf{D} ie
        \item[3)] \textbf{G} een
        \item[2)] \textbf{B} ananen
        \item[1)] \textbf{e} et 
    \end{itemize}
    \vspace{3mm}
\end{minipage}
\hfill
\begin{minipage}{0.7\textwidth}
    \infobox{Let op: dit is de standaardstemming. Soms wordt de gitaar anders gestemd. Dat wordt dan expliciet vermeld.}
\end{minipage}

\begin{minipage}{0.5\textwidth}
Je gebruikt een stemapparaat om te stemmen (zie \autoref{fig:tuning}). Het stemapparaat geeft een nootnaam aan, en dan moet je omhoog of omlaag stemmen tot de juiste noot op het scherm verschijnt. Of het toont een snaarnummer en je moet de 'wijzer' in het midden krijgen. Dat is afhankelijk van het type stemapparaat dat je hebt.

Wees voorzichtig met het omhoog stemmen (naar een hogere toon). Vooral de dunne snaren kunnen breken als ze te strak staan.
\end{minipage}
\hfill
\begin{minipage}{0.3\textwidth}
    \centering
    \includegraphics[width=\textwidth]{../../Images/guitar-tuning.jpg}
    \captionof{figure}{Een stemapparaat gebruiken op een gitaar}
    \label{fig:tuning}
\end{minipage}

Een andere stemmethode is gebaseerd op het verschil in toonhoogte tussen de snaren. In \autoref{fig:guitar_relative_tuning} zie je welke posities op de hals dezelfde toon hebben als de dunnere snaar eronder.

\begin{figure}[h]
    \centering
    \includegraphics[width=0.4\textwidth]{../../Images/GuitarRelativeTuning.png}
    \caption{Relatief stemmen}
    \label{fig:guitar_relative_tuning}
\end{figure}


\newpage
\chapter{Getting ready to play}

\section{Sitting/standing position}

\begin{figure}[h]
  \begin{subfigure}[b]{0.45\textwidth}
    \includegraphics[width=\textwidth]{../../Images/Letty_Guitar-Shooting_sitting.jpg}
    \caption{}
    \label{fig:positin_sitting}
  \end{subfigure}
  \hfill
  \begin{subfigure}[b]{0.45\textwidth}
    \includegraphics[width=\textwidth]{../../Images/Letty_Guitar-Shooting_standing.jpg}
    \caption{}
    \label{fig:positin_standing}
  \end{subfigure}
  \caption{\cite{SitStandPosition}}
  \label{fig:positin}
\end{figure}

\infobox{This method assumes a right-handed player. If you are left-handed, replace “right” with “left” and vice versa.}

Even though it may look cooler to place the guitar on your right leg. You will be more comfortable and precise when you are sitting the classical way. The classical way of sitting also translates better to a standing position (see \ref{fig:positin_standing}).

In the classical position you place the guitar on your left leg and the left leg will be slightly raised. You can use a foot stool for this (see the left foot in \ref{fig:positin_sitting}).
\newpage
\section{Stemmen}

Je gitaar moet gestemd zijn. Dit betekent dat elke snaar een bepaalde toonhoogte heeft.

In \autoref{fig:guitar_string_names} zie je de namen (letters) van de dunste (\textit{e}) tot de dikste (\textit{E}) snaar.

\begin{figure}[h]
    \centering
    \includegraphics[width=0.8\textwidth]{../../Images/guitar-neck-string-names.png}
    \caption{Namen van de snaren van de gitaar}
    \label{fig:guitar_string_names}
\end{figure}

Een ezelsbruggetje is (van hoog/dun naar laag/dik):

\begin{minipage}{0.25\textwidth}
    \vspace{3mm}
    \begin{itemize}
        \setlength\itemsep{0em}
        \item[6)] \textbf{E} en
        \item[5)] \textbf{A} ap
        \item[4)] \textbf{D} ie
        \item[3)] \textbf{G} een
        \item[2)] \textbf{B} ananen
        \item[1)] \textbf{e} et 
    \end{itemize}
    \vspace{3mm}
\end{minipage}
\hfill
\begin{minipage}{0.7\textwidth}
    \infobox{Let op: dit is de standaardstemming. Soms wordt de gitaar anders gestemd. Dat wordt dan expliciet vermeld.}
\end{minipage}

\begin{minipage}{0.5\textwidth}
Je gebruikt een stemapparaat om te stemmen (zie \autoref{fig:tuning}). Het stemapparaat geeft een nootnaam aan, en dan moet je omhoog of omlaag stemmen tot de juiste noot op het scherm verschijnt. Of het toont een snaarnummer en je moet de 'wijzer' in het midden krijgen. Dat is afhankelijk van het type stemapparaat dat je hebt.

Wees voorzichtig met het omhoog stemmen (naar een hogere toon). Vooral de dunne snaren kunnen breken als ze te strak staan.
\end{minipage}
\hfill
\begin{minipage}{0.3\textwidth}
    \centering
    \includegraphics[width=\textwidth]{../../Images/guitar-tuning.jpg}
    \captionof{figure}{Een stemapparaat gebruiken op een gitaar}
    \label{fig:tuning}
\end{minipage}

Een andere stemmethode is gebaseerd op het verschil in toonhoogte tussen de snaren. In \autoref{fig:guitar_relative_tuning} zie je welke posities op de hals dezelfde toon hebben als de dunnere snaar eronder.

\begin{figure}[h]
    \centering
    \includegraphics[width=0.4\textwidth]{../../Images/GuitarRelativeTuning.png}
    \caption{Relatief stemmen}
    \label{fig:guitar_relative_tuning}
\end{figure}


\newpage
\chapter{Getting ready to play}

\section{Sitting/standing position}

\begin{figure}[h]
  \begin{subfigure}[b]{0.45\textwidth}
    \includegraphics[width=\textwidth]{../../Images/Letty_Guitar-Shooting_sitting.jpg}
    \caption{}
    \label{fig:positin_sitting}
  \end{subfigure}
  \hfill
  \begin{subfigure}[b]{0.45\textwidth}
    \includegraphics[width=\textwidth]{../../Images/Letty_Guitar-Shooting_standing.jpg}
    \caption{}
    \label{fig:positin_standing}
  \end{subfigure}
  \caption{\cite{SitStandPosition}}
  \label{fig:positin}
\end{figure}

\infobox{This method assumes a right-handed player. If you are left-handed, replace “right” with “left” and vice versa.}

Even though it may look cooler to place the guitar on your right leg. You will be more comfortable and precise when you are sitting the classical way. The classical way of sitting also translates better to a standing position (see \ref{fig:positin_standing}).

In the classical position you place the guitar on your left leg and the left leg will be slightly raised. You can use a foot stool for this (see the left foot in \ref{fig:positin_sitting}).
\newpage
\section{Stemmen}

Je gitaar moet gestemd zijn. Dit betekent dat elke snaar een bepaalde toonhoogte heeft.

In \autoref{fig:guitar_string_names} zie je de namen (letters) van de dunste (\textit{e}) tot de dikste (\textit{E}) snaar.

\begin{figure}[h]
    \centering
    \includegraphics[width=0.8\textwidth]{../../Images/guitar-neck-string-names.png}
    \caption{Namen van de snaren van de gitaar}
    \label{fig:guitar_string_names}
\end{figure}

Een ezelsbruggetje is (van hoog/dun naar laag/dik):

\begin{minipage}{0.25\textwidth}
    \vspace{3mm}
    \begin{itemize}
        \setlength\itemsep{0em}
        \item[6)] \textbf{E} en
        \item[5)] \textbf{A} ap
        \item[4)] \textbf{D} ie
        \item[3)] \textbf{G} een
        \item[2)] \textbf{B} ananen
        \item[1)] \textbf{e} et 
    \end{itemize}
    \vspace{3mm}
\end{minipage}
\hfill
\begin{minipage}{0.7\textwidth}
    \infobox{Let op: dit is de standaardstemming. Soms wordt de gitaar anders gestemd. Dat wordt dan expliciet vermeld.}
\end{minipage}

\begin{minipage}{0.5\textwidth}
Je gebruikt een stemapparaat om te stemmen (zie \autoref{fig:tuning}). Het stemapparaat geeft een nootnaam aan, en dan moet je omhoog of omlaag stemmen tot de juiste noot op het scherm verschijnt. Of het toont een snaarnummer en je moet de 'wijzer' in het midden krijgen. Dat is afhankelijk van het type stemapparaat dat je hebt.

Wees voorzichtig met het omhoog stemmen (naar een hogere toon). Vooral de dunne snaren kunnen breken als ze te strak staan.
\end{minipage}
\hfill
\begin{minipage}{0.3\textwidth}
    \centering
    \includegraphics[width=\textwidth]{../../Images/guitar-tuning.jpg}
    \captionof{figure}{Een stemapparaat gebruiken op een gitaar}
    \label{fig:tuning}
\end{minipage}

Een andere stemmethode is gebaseerd op het verschil in toonhoogte tussen de snaren. In \autoref{fig:guitar_relative_tuning} zie je welke posities op de hals dezelfde toon hebben als de dunnere snaar eronder.

\begin{figure}[h]
    \centering
    \includegraphics[width=0.4\textwidth]{../../Images/GuitarRelativeTuning.png}
    \caption{Relatief stemmen}
    \label{fig:guitar_relative_tuning}
\end{figure}



%Reading notes
%Scales and Chord patterns (learning the fretboard)
%Strumming with fingers
%Basic chord shapes + CAGFD
%Extending the chords
%Finger picking
%Strumming patterns

\printbibliography

\begin{appendices}
\chapter{Getting ready to play}

\section{Sitting/standing position}

\begin{figure}[h]
  \begin{subfigure}[b]{0.45\textwidth}
    \includegraphics[width=\textwidth]{../../Images/Letty_Guitar-Shooting_sitting.jpg}
    \caption{}
    \label{fig:positin_sitting}
  \end{subfigure}
  \hfill
  \begin{subfigure}[b]{0.45\textwidth}
    \includegraphics[width=\textwidth]{../../Images/Letty_Guitar-Shooting_standing.jpg}
    \caption{}
    \label{fig:positin_standing}
  \end{subfigure}
  \caption{\cite{SitStandPosition}}
  \label{fig:positin}
\end{figure}

\infobox{This method assumes a right-handed player. If you are left-handed, replace “right” with “left” and vice versa.}

Even though it may look cooler to place the guitar on your right leg. You will be more comfortable and precise when you are sitting the classical way. The classical way of sitting also translates better to a standing position (see \ref{fig:positin_standing}).

In the classical position you place the guitar on your left leg and the left leg will be slightly raised. You can use a foot stool for this (see the left foot in \ref{fig:positin_sitting}).
\newpage
\section{Stemmen}

Je gitaar moet gestemd zijn. Dit betekent dat elke snaar een bepaalde toonhoogte heeft.

In \autoref{fig:guitar_string_names} zie je de namen (letters) van de dunste (\textit{e}) tot de dikste (\textit{E}) snaar.

\begin{figure}[h]
    \centering
    \includegraphics[width=0.8\textwidth]{../../Images/guitar-neck-string-names.png}
    \caption{Namen van de snaren van de gitaar}
    \label{fig:guitar_string_names}
\end{figure}

Een ezelsbruggetje is (van hoog/dun naar laag/dik):

\begin{minipage}{0.25\textwidth}
    \vspace{3mm}
    \begin{itemize}
        \setlength\itemsep{0em}
        \item[6)] \textbf{E} en
        \item[5)] \textbf{A} ap
        \item[4)] \textbf{D} ie
        \item[3)] \textbf{G} een
        \item[2)] \textbf{B} ananen
        \item[1)] \textbf{e} et 
    \end{itemize}
    \vspace{3mm}
\end{minipage}
\hfill
\begin{minipage}{0.7\textwidth}
    \infobox{Let op: dit is de standaardstemming. Soms wordt de gitaar anders gestemd. Dat wordt dan expliciet vermeld.}
\end{minipage}

\begin{minipage}{0.5\textwidth}
Je gebruikt een stemapparaat om te stemmen (zie \autoref{fig:tuning}). Het stemapparaat geeft een nootnaam aan, en dan moet je omhoog of omlaag stemmen tot de juiste noot op het scherm verschijnt. Of het toont een snaarnummer en je moet de 'wijzer' in het midden krijgen. Dat is afhankelijk van het type stemapparaat dat je hebt.

Wees voorzichtig met het omhoog stemmen (naar een hogere toon). Vooral de dunne snaren kunnen breken als ze te strak staan.
\end{minipage}
\hfill
\begin{minipage}{0.3\textwidth}
    \centering
    \includegraphics[width=\textwidth]{../../Images/guitar-tuning.jpg}
    \captionof{figure}{Een stemapparaat gebruiken op een gitaar}
    \label{fig:tuning}
\end{minipage}

Een andere stemmethode is gebaseerd op het verschil in toonhoogte tussen de snaren. In \autoref{fig:guitar_relative_tuning} zie je welke posities op de hals dezelfde toon hebben als de dunnere snaar eronder.

\begin{figure}[h]
    \centering
    \includegraphics[width=0.4\textwidth]{../../Images/GuitarRelativeTuning.png}
    \caption{Relatief stemmen}
    \label{fig:guitar_relative_tuning}
\end{figure}


\end{appendices}

\end{document}