\chapter{Hoe gebruik je dit boek}

Het idee van deze methode is om je de tools te geven om creatief te zijn met muziek. Concreet betekent dit dat er niet wordt gezegd: "Het D-akkoord speel je met deze vorm", maar: "Een akkoord wordt zo opgebouwd. Dus om een D-akkoord te spelen doe je 'dit' en kom je uit op deze vorm". Waarbij 'dit' kennis is die je zult leren.

Tijdens de methode zul je merken dat je bepaalde constructies/symbolen/etc. tegenkomt die je misschien nog niet kent en die niet direct worden uitgelegd. Dit is met opzet. Het idee is dat je door er vroeg mee in aanraking te komen, terwijl je het nog niet bewust nodig hebt, het later makkelijker leert begrijpen.

In stappen ziet dat er als volgt uit:

\begin{enumerate}
\item Je wordt blootgesteld aan nieuwe concepten zodat je ze hebt gezien, maar nog niet per se begrijpt.
\item Aan de hand van oefeningen en (bestaande) nummers worden de eerder getoonde concepten uitgelegd en hoe ze samenwerken.
\item Je begrijpt de theorie van de concepten en kunt ze toepassen tijdens het spelen.
\item Begin opnieuw bij stap 1 met nieuwe concepten.
\end{enumerate}

\autoref{fig:guitar_book_build_up} illustreert de belangrijkste bouwstenen van deze methode. Het boek begint bij de onderste rij (de basis) en bouwt daarop verder.

\begin{figure}[!b]
\centering
\includegraphics[width=0.85\textwidth]{../../Images/BookBuildup_NL.png}
\caption{Algemene bouwstenen van dit boek}
\label{fig:guitar_book_build_up}
\end{figure}

\newpage

De fretborddiagrammen zijn zo getekend dat de \textbf{dikste snaar onderaan} staat. Dit lijkt op hoe je naar de hals kijkt als je de gitaar vasthoudt.

De akkoorddiagrammen worden getoond met of zonder de open snaren (\autoref{fig:open_chord_chart_example} en \autoref{fig:closed_chord_chart_example} respectievelijk). De getoonde fretnummers onderaan staan op dezelfde plek als de fretindicaties op de gitaarhals.

De \textbf{groene noten} geven de grondtoon aan. De \textbf{oranje noten} zijn de overige noten in het akkoord. De \textbf{grijze "X"} betekent dat je deze snaar niet speelt, of dempt.

\begin{figure}[h]
\begin{subfigure}[b]{0.45\textwidth}
\centering
\includegraphics[height=0.3\textheight]{../../Images/OpenCChordChart.png}
\caption{Open-C akkoorddiagram}
\label{fig:open_chord_chart_example}
\end{subfigure}
\hfill
\begin{subfigure}[b]{0.45\textwidth}
\centering
\includegraphics[height=0.3\textheight]{../../Images/ClosedCChordChart.png}
\caption{Gesloten-C akkoorddiagram}
\label{fig:closed_chord_chart_example}
\end{subfigure}
\caption{Verschillende soorten akkoorddiagrammen gebruikt in dit boek}
\label{fig:kinds_of_chord_charts_in_the_book}
\end{figure}