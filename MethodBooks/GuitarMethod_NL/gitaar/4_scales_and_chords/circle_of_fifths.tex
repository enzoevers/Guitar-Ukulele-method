\section{Kwintencirkel} \label{sec:circle_of_fifths}

\subsection{Wat betekent het?}

In een eerdere sectie werden de volgende tabellen getoond. Ook werd de term "kwintencirkel" genoemd en dat dit te maken heeft met het feit dat er steeds een extra $\sharp$ of $\flat$ wordt toegevoegd.

\begin{table}[h]
	\centering
	\begin{minipage}{0.45\textwidth}
		\begin{NiceTabular}{*{16}{P{0.05mm}}}
			\Block{}{} & \Block{1-2}{\large{W}} & & \Block{1-2}{\large{W}} & & \Block{1-2}{\large{H}} & & \Block{1-2}{\large{W}} & & \Block{1-2}{\large{W}} & & \Block{1-2}{\large{W}} & & \Block{1-2}{\large{H}} & & \Block{}{} \\
			\Block{1-2}{\RomanNumeralCaps{1}} & & \Block{1-2}{\RomanNumeral{2}} & & \Block{1-2}{\RomanNumeral{3}} & & \Block{1-2}{\RomanNumeralCaps{4}} & & \Block{1-2}{\RomanNumeralCaps{5}} & & \Block{1-2}{\RomanNumeral{6}} & & \Block{1-2}{\RomanNumeral{7}\textsuperscript{o}} & & \\
			\Block{1-2}{1} & & \Block{1-2}{2} & & \Block{1-2}{3} & & \Block{1-2}{4} & & \Block{1-2}{5} & & \Block{1-2}{6} & & \Block{1-2}{7} & & \Block{1-2}{8} & \\
			\Block{1-2}{F} & & \Block{1-2}{G} & & \Block{1-2}{A} & & \Block{1-2}{B\flat} & & \Block{1-2}{C} & & \Block{1-2}{D} & & \Block{1-2}{E} & & \Block{1-2}{F} & \\
			\Block{1-2}{C} & & \Block{1-2}{D} & & \Block{1-2}{E} & & \Block{1-2}{F} & & \Block{1-2}{G} & & \Block{1-2}{A} & & \Block{1-2}{B} & & \Block{1-2}{C} & \\
			\Block{1-2}{G} & & \Block{1-2}{A} & & \Block{1-2}{B} & & \Block{1-2}{C} & & \Block{1-2}{D} & & \Block{1-2}{E} & & \Block{1-2}{F\sharp} & & \Block{1-2}{G} & \\
			\Block{1-2}{D} & & \Block{1-2}{E} & & \Block{1-2}{F\sharp} & & \Block{1-2}{G} & & \Block{1-2}{A} & & \Block{1-2}{B} & & \Block{1-2}{C\sharp} & & \Block{1-2}{D} & \\
			\Block{1-2}{A} & & \Block{1-2}{B} & & \Block{1-2}{C\sharp} & & \Block{1-2}{D} & & \Block{1-2}{E} & & \Block{1-2}{F\sharp} & & \Block{1-2}{G\sharp} & & \Block{1-2}{A} & \\
			\Block{1-2}{E} & & \Block{1-2}{F\sharp} & & \Block{1-2}{G\sharp} & & \Block{1-2}{A} & & \Block{1-2}{B} & & \Block{1-2}{C\sharp} & & \Block{1-2}{D\sharp} & & \Block{1-2}{E} & \\
			\Block{1-2}{B} & & \Block{1-2}{C\sharp} & & \Block{1-2}{D\sharp} & & \Block{1-2}{E} & & \Block{1-2}{F\sharp} & & \Block{1-2}{G\sharp} & & \Block{1-2}{A\sharp} & & \Block{1-2}{B} & 
		\end{NiceTabular}
		\caption{Majeur toonladders}
		\label{tab:guitar_major_scales_circle_of_fifths}
	\end{minipage}
	\hfill
	\begin{minipage}{0.45\textwidth}
		\begin{NiceTabular}{*{16}{P{0.05mm}}}
			\Block{}{} & \Block{1-2}{\large{W}} & & \Block{1-2}{\large{H}} & & \Block{1-2}{\large{W}} & & \Block{1-2}{\large{W}} & & \Block{1-2}{\large{H}} & & \Block{1-2}{\large{W}} & & \Block{1-2}{\large{W}} & & \Block{}{} \\
			\Block{1-2}{\RomanNumeral{1}} & & \Block{1-2}{\RomanNumeral{2}\textsuperscript{o}} & & \Block{1-2}{\RomanNumeralCaps{3}} & & \Block{1-2}{\RomanNumeral{4}} & & \Block{1-2}{\RomanNumeral{5}} & & \Block{1-2}{\RomanNumeralCaps{6}} & & \Block{1-2}{\RomanNumeralCaps{7}} & & \\
			\Block{1-2}{1} & & \Block{1-2}{2} & & \Block{1-2}{3$\flat$} & & \Block{1-2}{4} & & \Block{1-2}{5} & & \Block{1-2}{6$\flat$} & & \Block{1-2}{7$\flat$} & & \Block{1-2}{8} & \\
			\Block{1-2}{B} & & \Block{1-2}{C\sharp} & & \Block{1-2}{D} & & \Block{1-2}{E} & & \Block{1-2}{F\sharp} & & \Block{1-2}{G} & & \Block{1-2}{A} & & \Block{1-2}{B} & \\
			\Block{1-2}{E} & & \Block{1-2}{F\sharp} & & \Block{1-2}{G} & & \Block{1-2}{A} & & \Block{1-2}{B} & & \Block{1-2}{C} & & \Block{1-2}{D} & & \Block{1-2}{E} & \\
			\Block{1-2}{A} & & \Block{1-2}{B} & & \Block{1-2}{C} & & \Block{1-2}{D} & & \Block{1-2}{E} & & \Block{1-2}{F} & & \Block{1-2}{G} & & \Block{1-2}{A} & \\
			\Block{1-2}{D} & & \Block{1-2}{E} & & \Block{1-2}{F} & & \Block{1-2}{G} & & \Block{1-2}{A} & & \Block{1-2}{B\flat} & & \Block{1-2}{C} & & \Block{1-2}{D} & \\
			\Block{1-2}{G} & & \Block{1-2}{A} & & \Block{1-2}{B\flat} & & \Block{1-2}{C} & & \Block{1-2}{D} & & \Block{1-2}{E\flat} & & \Block{1-2}{F} & & \Block{1-2}{G} & \\
			\Block{1-2}{C} & & \Block{1-2}{D} & & \Block{1-2}{E\flat} & & \Block{1-2}{F} & & \Block{1-2}{G} & & \Block{1-2}{A\flat} & & \Block{1-2}{B\flat} & & \Block{1-2}{C} & \\
			\Block{1-2}{F} & & \Block{1-2}{G} & & \Block{1-2}{A\flat} & & \Block{1-2}{B\flat} & & \Block{1-2}{C} & & \Block{1-2}{D\flat} & & \Block{1-2}{E\flat} & & \Block{1-2}{F} & 
		\end{NiceTabular}
		\caption{Mineur toonladders}
		\label{tab:guitar_minor_scales_circle_of_fifths}
	\end{minipage}
\end{table}

Bekijk de cirkel in \autoref{fig:guitar_circle_of_fifths}.

\begin{figure}[h]
	\centering
	\includegraphics[width=0.55\textwidth]{../../Images/Circle_of_fifths_deluxe_4.svg.png}
	\caption{Kwintencirkel \cite{CircleOfFifthsImageWikipedia}}
	\label{fig:guitar_circle_of_fifths}
\end{figure}

Let op de volgende punten:
\begin{itemize}
	\item \textbf{Met de klok mee:} Elke noot rechts van een noot is de \textbf{5e graad} in de majeur toonladder van die noot. Je gaat \textbf{een reine kwint omhoog}.
	\item \textbf{Tegen de klok in:} Elke noot links van een noot is de \textbf{4e graad} in de majeur toonladder van die noot. Je gaat \textbf{een reine kwint omlaag}. Om dit te zien, kijk naar \autoref{tab:guitar_major_scales_circle_of_fifths}, begin bij de 8e graad en tel het aantal halve tonen terug naar de 4e graad. Dit zijn 7 halve tonen. Kijk nu naar \autoref{tab:guitar_intervals_in_octave} en zie dat 7 halve tonen een reine kwint is.
\end{itemize}

\begin{itemize}
	\item \textbf{Met de klok mee:} Bij elke nieuwe stap wordt er een $\sharp$ toegevoegd.
	\item \textbf{Tegen de klok in:} Bij elke nieuwe stap wordt er een $\flat$ toegevoegd.
\end{itemize}

De $\sharp$ en $\flat$ die worden toegevoegd volgen beide hetzelfde ezelsbruggetje, maar omgekeerd:

\begin{table}[h]
	\centering
	\begin{NiceTabular}{r | l l l l l l l}
		Met de klok mee ($\sharp$) & \textbf{F}ijne & \textbf{C}harles & \textbf{G}aat & \textbf{D}agelijks & \textbf{A}ppeltaart & \textbf{E}ten & \textbf{B}innen \\
		Tegen de klok in ($\flat$) & \textbf{B}innen & \textbf{E}ten & \textbf{A}ppeltaart & \textbf{D}agelijks & \textbf{G}aat & \textbf{C}harles & \textbf{F}ijne \\
	\end{NiceTabular}
\end{table}

Let op dat de kruizen en mollen ook gescheiden zijn door een reine kwint.

\newpage

\subsection{De kwintencirkel gebruiken}

\subsubsection{Relatieve mineur/majeur toonladders}

Als je de noten in de C-majeur en A-mineur toonladder vergelijkt, zie je dat ze dezelfde noten hebben. Dit betekent dat A-mineur de \textbf{relatieve mineur} is van C-majeur. En andersom is C-majeur de \textbf{relatieve majeur} van A-mineur. De binnenste cirkel in de kwintencirkel (de mineurcirkel) toont de relatieve mineur toonladders bij de buitenste (majeur) cirkel (en andersom).

Zo kun je snel zien dat de relatieve mineur van E$\flat$-majeur C-mineur is, en dat de relatieve majeur van B-mineur D-majeur is.

\subsubsection{Snel de akkoorden in een toonladder tonen}

Aan het begin van dit hoofdstuk heb je al gezien dat er majeur-, mineur- en verminderde akkoorden in een toonladder zitten en hoe je deze herkent. Deze akkoorden worden aangeduid met Romeinse cijfers.

\underline{Akkoorden in de majeur toonladder}

Bekijk \autoref{fig:circle_of_fifths_major_scale_chords}. We beginnen bij een noot op de majeurcirkel.

Er werd al gezegd dat de noot rechts de 5e graad van een toonladder is, en de noot links de 4e graad.

De 6e graad in de majeur toonladder is de relatieve mineur. Dit is de noot onder de \RomanNumeralCaps{1} van de majeurcirkel. Als je een reine kwint (7 halve tonen) omhoog gaat vanaf de 6e graad, kom je uit bij de 3e graad. Ga je een reine kwint omlaag, dan kom je bij de 2e graad.

Het resterende akkoord is het verminderde akkoord. Dit vind je door een reine kwint omhoog te gaan vanaf de 3e graad (één stap naar rechts).

\underline{Akkoorden in de mineur toonladder}

Bekijk \autoref{fig:circle_of_fifths_minor_scale_chords}. We beginnen bij een noot op de mineurcirkel.

De reine kwart en reine kwint komen zowel in de majeur- als in de mineurtoonladder voor. Rechts heb je dus de 5e graad en links de 4e graad in de mineurtoonladder (door respectievelijk een reine kwint omhoog of omlaag te gaan).

De relatieve majeur is de 3e graad. Dit is de noot boven de \RomanNumeral{1} van de mineurcirkel. Een reine kwint omhoog vanaf de 3e graad in de mineurtoonladder brengt je bij de 7e graad. Een reine kwint omlaag vanaf de 3e graad brengt je bij de 6e graad.

Het resterende verminderde akkoord vind je door een reine kwint omhoog te gaan vanaf de 5e graad in de mineurcirkel.

\begin{figure}[h]
	\centering
	\begin{subfigure}{0.47\textwidth}
		\includegraphics[width=\textwidth]{../../Images/CircleOfFifthsChordsInTheMajorScale.png}
		\caption{Kwintencirkel: akkoorden in de majeur toonladder}
		\label{fig:circle_of_fifths_major_scale_chords}
	\end{subfigure}
	\hfill
	\begin{subfigure}{0.47\textwidth}
		\includegraphics[width=\textwidth]{../../Images/CircleOfFifthsChordsInTheMinorScale.png}
		\caption{Kwintencirkel: akkoorden in de mineur toonladder}
		\label{fig:circle_of_fifths_minor_scale_chords}
	\end{subfigure}
	\caption{Draaibare akkoordgraden in de kwintencirkel}
\end{figure}

\newpage

\subsubsection{Startpunt voor het bepalen van de toonsoort van bladmuziek}

Als je aan het begin van de bladmuziek 3 $\sharp$s ziet, zal het lied meestal noten uit de toonsoort A-majeur gebruiken en kun je het dus als A-majeur beschouwen. Het kan echter ook in F$\sharp$-mineur staan (de relatieve mineur van A-majeur). Ze gebruiken dezelfde noten, maar de context van het lied bepaalt of het A-majeur of F$\sharp$-mineur is.

\subsubsection{Zien naar welke toonsoorten je kunt moduleren}

Als je door de cirkel gaat, is er altijd één noot die verschilt (een toegevoegde/verwijderde $\sharp$/$\flat$).

Dus als je wilt moduleren naar een nieuwe toonsoort, kun je de kwintencirkel gebruiken om te bepalen hoe 'groot' je de verandering wilt maken. Of om te zien via welke toonsoorten je kunt gaan om bij de 'eindtoonsoort' te komen.

\underline{Voorbeeld van een abrupte toonsoortwisseling}

In "Livin' On A Prayer" van "Bon Jovi" is er een toonsoortwisseling in het outro-refrein. Het gaat van E-mineur naar G-mineur. Beide gebruiken dezelfde akkoordprogressie.

\begin{table}[h]
	\centering
	\begin{NiceTabular}{|l|c|c|c|c|c|c|}
		\hline
		\backslashbox{Toonsoort}{Akkoord} & \RomanNumeral{1} & \RomanNumeralCaps{6} & \RomanNumeralCaps{7} & \RomanNumeralCaps{3} & \RomanNumeralCaps{6} & \RomanNumeralCaps{7} \\
		\hline
		E mineur & Em & C & D & G & C & D \\
		\hline
		G mineur & Gm & E$\flat$ & F & B$\flat$ & E$\flat$ & F \\
		\hline
	\end{NiceTabular}
\end{table}


\begin{song}[verse/numbered, align-chords=l]{title={Livin' On A Prayer - Bon Jovi}, music={Bon Jovi}}
	\begin{chorus}
		^{Em}Oh, ^{C} we're ^{D}halfway there, ^{G}woah ^{C}oh, ^{D}livin' on a prayer. \\
		^{Em}Take my ^{C}hand, we'll ^{D}make it I swear \\
		^{G}Woah ^{C}oh, ^{D}livin' on a prayer. \\
	\end{chorus}
	
	\begin{outro-chorus}
		^*{Gm}Woooo ^{Eb}oo, we're ^{F}halfway there, ^{Bb}Woah ^{Eb}oh, ^{F}livin' on a prayer. \\
		^{Gm}Take my ^{Eb}hand and we'll ^{F}make it I swear \\
		^{Bb}Woah ^{Eb}oh, ^{F}livin' on a prayer. \\
	\end{outro-chorus}
\end{song}

Als je kijkt naar \autoref{fig:guitar_circle_of_fifths}, zie je dat G-mineur drie stappen links van E-mineur ligt. Dat betekent dat er behoorlijk wat verschillende noten zijn tussen beide toonladders.

Als je ook kijkt naar de toonladder van E-mineur (in \autoref{tab:guitar_minor_scales_circle_of_fifths}), zie je dat G de 3e graad is (een kleine terts). Dat betekent dat de G een belangrijke noot is die de E-mineur toonladder 'mineur' maakt.

\newpage

\underline{Voorbeeld van een draaipuntakkoord/noot}

In "We Are The Champions" van "Queen" zie je een voorbeeld van een draaipuntakkoord/noot. Dit nummer wisselt niet alleen van toonsoort, maar ook van modus. Het gaat van C-mineur (tijdens de coupletten) naar F-majeur (tijdens het refrein).

\autoref{fig:circle_of_fith_overlap_c_minor_f_major} laat zien dat er twee overlappende akkoorden zijn (B$\flat$, G-mineur) en vijf overlappende noten (B$\flat$, C, D, F, G) in deze twee toonsoorten. Merk op dat het D-mineur akkoord niet als overlappend akkoord wordt genoemd. Dit komt omdat in de toonsoort C-mineur de \textit{D} een verminderd akkoord is, terwijl het in F-majeur een mineur akkoord is. In "We Are The Champions" wordt B$\flat$ gebruikt als draaipuntakkoord/noot om C-mineur met F-majeur te verbinden.

Bovendien wordt aan het einde van het couplet een C7-akkoord gespeeld, dat wel in de F-majeur toonladder zit maar niet in de C-mineur toonladder. Het Cm-akkoord bevat \textit{C, E$\flat$, G}. Het C7-akkoord bevat \textit{C, E, G, B$\flat$}. De enige noot in het C7-akkoord die het specifiek bij F-majeur laat horen, is de E.

\begin{figure}[h]
	\centering
	\includegraphics[width=0.45\textwidth]{../../Images/CircleOfFifthsCminorFmajor.png}
	\caption{Overlapping akkoorden in C-mineur (oranje) en F-majeur (paars)}.
	\label{fig:circle_of_fith_overlap_c_minor_f_major}
\end{figure}

\infobox{Sommige van deze akkoordnamen zijn nog niet besproken. Dit wordt in het volgende hoofdstuk uitgelegd. Wil je alvast meespelen, bekijk dan de akkoorddiagrammen op de volgende pagina (\autoref{fig:queen_we_are_the_champions_key_change_chords}).}

\begin{song}[align-chords=l]{title={We Are The Champions - Queen}, music={Queen}} \label{song:we_are_the_champions_queen}
	\begin{verse}
		I've had my ^{Eb}share of sand ^{Bb}kicked in my ^{Cm}face \\
		But ^{F7}I've come ^{Bb}through \\
		And I mean to go ^{Ab/Bb}on, and ^{Bbm7(b5)}on, and ^{Bb7}on, and ^{C7}on \\
	\end{verse}
	
	\begin{chorus}
		^{F}We are the ^{Am}champions, my ^{Dm}friends ^{Bb} ^{C7} \\
		And ^{F}we'll keep on ^{Am}fighting till the ^{Bb}end ^{F#dim7} \\
		^{Gm}We are the ^{C7/G}champions, ^{Bbm}we are the ^{Edim7}champions \\
		^{F}No time for ^{Gm}losers 'cause ^{Ab}we are the ^{Bb}champions ^{C7sus4} \\
		Of the ^{Fm}world ^{Gm7}
 	\end{chorus}
\end{song}

\newpage

\begin{figure}[h]
	\centering
	\includegraphics[width=0.65\textwidth]{../../Images/WeAreTheChampionsQueenKeyChangeChords.png}
	\caption{Enkele akkoorden uit "We Are The Champions" - Queen}.
	\label{fig:queen_we_are_the_champions_key_change_chords}
\end{figure}


\newpage