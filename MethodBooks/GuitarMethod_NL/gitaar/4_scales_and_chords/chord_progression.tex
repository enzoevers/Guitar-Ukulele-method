\section{Akkoordenprogressie} \label{sec:chord_progression}

\subsection{Akkoorden in een toonladder}

Een akkoordenprogressie is een reeks akkoorden die in een bepaalde volgorde worden gespeeld, meestal herhaaldelijk.

Stel, je speelt de akkoorden C en G herhaaldelijk in een (deel van een) nummer, dan is het akkoordenschema C - G. Maar dit akkoordenschema is niet universeel. Stel dat het C - G schema in de toonsoort C-majeur staat, welke akkoorden moet je dan spelen als je dit schema wilt omzetten naar de toonsoort A-majeur? Je kunt hier achter komen door veel te tellen, maar er is een makkelijkere manier. Akkoordenprogressies worden vaak aangeduid met Romeinse cijfers om ze universeel te maken.

Met Romeinse cijfers wordt het akkoordenschema "C - G" \RomanNumeralCaps{1} - \RomanNumeralCaps{5} (1 - 5). De toontreden in de (majeur) toonladder kunnen ook met Romeinse cijfers worden weergegeven. Zie \autoref{tab:guitar_major_scale_chords}. Het akkoord op de 7e index (\RomanNumeral{7}\textsuperscript{o}) heet een verminderde akkoord, maar die negeren we nu. De 8e index krijgt geen akkoord omdat die hetzelfde is als trap 1. Dit kan ook voor de mineur toonladder (\autoref{tab:guitar_minor_scale_chords}).

Twee dingen om op te letten:

\begin{itemize}
	\item \textbf{Hoofdletters} bij Romeinse cijfers betekenen \textbf{majeur} akkoorden
	\item \textbf{Kleine letters} bij Romeinse cijfers betekenen \textbf{mineur} akkoorden
\end{itemize}

\begin{table}[h]
	\begin{minipage}{0.45\textwidth}
		\centering
		\begin{NiceTabular}{*{16}{P{0.05mm}}}
			\Block{}{} & \Block{1-2}{\large{W}} & & \Block{1-2}{\large{W}} & & \Block{1-2}{\large{H}} & & \Block{1-2}{\large{W}} & & \Block{1-2}{\large{W}} & & \Block{1-2}{\large{W}} & & \Block{1-2}{\large{H}} & & \Block{}{} \\
			\Block{1-2}{1} & & \Block{1-2}{2} & & \Block{1-2}{3} & & \Block{1-2}{4} & & \Block{1-2}{5} & & \Block{1-2}{6} & & \Block{1-2}{7} & & \Block{1-2}{8} & \\
			\Block{1-2}{\RomanNumeralCaps{1}} & & \Block{1-2}{\RomanNumeral{2}} & & \Block{1-2}{\RomanNumeral{3}} & & \Block{1-2}{\RomanNumeralCaps{4}} & & \Block{1-2}{\RomanNumeralCaps{5}} & & \Block{1-2}{\RomanNumeral{6}} & & \Block{1-2}{\RomanNumeral{7}\textsuperscript{o}} & &
		\end{NiceTabular}
		\caption{Akkoorden in de majeur toonladder}
		\label{tab:guitar_major_scale_chords}
	\end{minipage}
	\hfill
	\begin{minipage}{0.45\textwidth}
		\centering
		\begin{NiceTabular}{*{16}{P{0.05mm}}}
			\Block{}{} & \Block{1-2}{\large{W}} & & \Block{1-2}{\large{H}} & & \Block{1-2}{\large{W}} & & \Block{1-2}{\large{W}} & & \Block{1-2}{\large{H}} & & \Block{1-2}{\large{W}} & & \Block{1-2}{\large{W}} & & \Block{}{} \\
			\Block{1-2}{1} & & \Block{1-2}{2} & & \Block{1-2}{3$\flat$} & & \Block{1-2}{4} & & \Block{1-2}{5} & & \Block{1-2}{6$\flat$} & & \Block{1-2}{7$\flat$} & & \Block{1-2}{8} & \\
			\Block{1-2}{\RomanNumeral{1}} & & \Block{1-2}{\RomanNumeral{2}\textsuperscript{o}} & & \Block{1-2}{\RomanNumeralCaps{3}} & & \Block{1-2}{\RomanNumeral{4}} & & \Block{1-2}{\RomanNumeral{5}} & & \Block{1-2}{\RomanNumeralCaps{6}} & & \Block{1-2}{\RomanNumeralCaps{7}} & &
		\end{NiceTabular}
		\caption{Akkoorden in de mineur toonladder}
		\label{tab:guitar_minor_scale_chords}
	\end{minipage}
\end{table}

\subsection{Waarom de majeur, mineur en verminderde akkoorden?} \label{sec:building_chords_with_diatonic_scale}

In het begin lijkt het misschien willekeurig waarom sommige toontreden in de toonladder majeurakkoorden zijn en andere mineur. Kijk naar \autoref{tab:guitar_major_chord_buildup_recap} en \autoref{tab:guitar_minor_chord_buildup_recap} (die heb je eerder gezien).

\begin{table}[h]
	\begin{minipage}{0.45\textwidth}
		\centering
		\begin{NiceTabular}{*{16}{P{0.05mm}}}
			\Block{}{} & \Block{1-2}{\large{W}} & & \Block{1-2}{\large{W}} & & \Block{1-2}{\large{H}} & & \Block{1-2}{\large{W}} & & \Block{1-2}{\large{W}} & & \Block{1-2}{\large{W}} & & \Block{1-2}{\large{H}} & & \Block{}{} \\
			\Block[fill=ColorRootNote]{1-2}{1} & & \Block{1-2}{2} & & \Block[fill=ColorOtherNote]{1-2}{3} & & \Block{1-2}{4} & & \Block[fill=ColorOtherNote]{1-2}{5} & & \Block{1-2}{6} & & \Block{1-2}{7} & & \Block{1-2}{8} &
		\end{NiceTabular}
		\caption{Opbouw van een majeurakkoord}
		\label{tab:guitar_major_chord_buildup_recap}
	\end{minipage}
	\hfill
	\begin{minipage}{0.45\textwidth}
		\centering
		\begin{NiceTabular}{*{16}{P{0.05mm}}}
			\Block{}{} & \Block{1-2}{\large{W}} & & \Block{1-2}{\large{H}} & & \Block{1-2}{\large{W}} & & \Block{1-2}{\large{W}} & & \Block{1-2}{\large{H}} & & \Block{1-2}{\large{W}} & & \Block{1-2}{\large{W}} & & \Block{}{} \\
			\Block[fill=ColorRootNote]{1-2}{1} & & \Block{1-2}{2} & & \Block[fill=ColorOtherNote]{1-2}{3$\flat$} & & \Block{1-2}{4} & & \Block[fill=ColorOtherNote]{1-2}{5} & & \Block{1-2}{6$\flat$} & & \Block{1-2}{7$\flat$} & & \Block{1-2}{8} &
		\end{NiceTabular}
		\caption{Opbouw van een mineurakkoord}
		\label{tab:guitar_minor_chord_buildup_recap}
	\end{minipage}
\end{table}

Deze tabellen laten zien dat een akkoord bestaat uit de 1e, 3e en 5e toontreden van een toonladder. Je kunt het ook zien als steeds \textbf{twee stappen vooruit springen in de toonladder} tot je drie noten hebt die samen een akkoord vormen.

In deze sectie wordt vaak verwezen naar intervallen. Kijk nog eens naar \autoref{tab:guitar_intervals_in_octave} als je een opfrisser wilt.

Voor de majeur toonladder:

\begin{itemize}
	\item Er zitten \textbf{4 halve tonen} tussen de 1e en 3e graad. Dit interval heet een \textbf{grote terts}.
	\item Er zitten \textbf{7 halve tonen} tussen de 1e en 5e graad. Dit interval heet een \textbf{reine kwint}.
	\subitem Ook zitten er 3 halve tonen tussen de 3e en 5e graad. Dit interval heet een kleine terts.
\end{itemize}

Voor de mineur toonladder:

\begin{itemize}
	\item Er zitten \textbf{3 halve tonen} tussen de 1e en 3e graad. Dit interval heet een \textbf{kleine terts}.
	\item Er zitten \textbf{7 halve tonen} tussen de 1e en 5e graad. Dit interval heet een \textbf{reine kwint}.
	\subitem Ook zitten er 4 halve tonen tussen de 3e en 5e graad. Dit interval heet een grote terts.
\end{itemize}

Let op: zowel majeur als mineur hebben een reine kwint. Het enige verschil is dat een majeurakkoord een grote terts heeft, terwijl een mineurakkoord een kleine terts heeft.

Hierna volgen enkele voorbeelden. Deze principes kun je toepassen op elke noot in zowel de majeur als de mineur toonladder.

\newpage

\subsubsection{Een mineurakkoord herkennen}
Laten we bepalen of de 3e graad van de majeur toonladder een majeur- of mineurakkoord is. De gekleurde indexen in \autoref{tab:guitar_major_scale_3rd_chord_identify} vind je door te beginnen bij de 3e graad en steeds twee stappen vooruit te springen tot je drie noten hebt.

\begin{table}[h]
	\centering
	\begin{NiceTabular}{*{16}{P{0.05mm}}}
		\Block{}{} & \Block{1-2}{\large{W}} & & \Block{1-2}{\large{W}} & & \Block{1-2}{\large{H}} & & \Block{1-2}{\large{W}} & & \Block{1-2}{\large{W}} & & \Block{1-2}{\large{W}} & & \Block{1-2}{\large{H}} & & \Block{}{} \\
		\Block{1-2}{1} & & \Block{1-2}{2} & & \Block[fill=ColorRootNote]{1-2}{3} & & \Block{1-2}{4} & & \Block[fill=ColorOtherNote]{1-2}{5} & & \Block{1-2}{6} & & \Block[fill=ColorOtherNote]{1-2}{7} & & \Block{1-2}{8} &
	\end{NiceTabular}
	\caption{Intervallen voor het 3e akkoord in de majeur toonladder}
	\label{tab:guitar_major_scale_3rd_chord_identify}
\end{table}

Nu moeten we naar de intervallen tussen de noten kijken.

\begin{itemize}
	\item 3 to 5: 3 halve tonen (kleine terts)
	\item 3 to 7: 7 halve tonen (reine kwint)
\end{itemize}

We hebben een kleine terts en een reine kwint. Dit betekent dat het een mineurakkoord is.

\subsubsection{Een majeurakkoord herkennen}
Laten we bepalen of de 5e graad van de majeur toonladder een majeur- of mineurakkoord is. De gekleurde indexen in \autoref{tab:guitar_major_scale_5th_chord_identify} vind je door te beginnen bij de 5e graad en steeds twee stappen vooruit te springen tot je drie noten hebt.

Let op: deze keer zijn we voor de laatste noot rondgegaan. De 8e noot is namelijk dezelfde als de 1e. Dus vanaf index 7 gingen we eerst een halve toon naar 8 (dezelfde als 1) en daarna een hele toon naar index 2.

\begin{table}[h]
	\centering
	\begin{NiceTabular}{*{16}{P{0.05mm}}}
		\Block{}{} & \Block{1-2}{\large{W}} & & \Block{1-2}{\large{W}} & & \Block{1-2}{\large{H}} & & \Block{1-2}{\large{W}} & & \Block{1-2}{\large{W}} & & \Block{1-2}{\large{W}} & & \Block{1-2}{\large{H}} & & \Block{}{} \\
		\Block{1-2}{1} & & \Block[fill=ColorOtherNote]{1-2}{2} & & \Block{1-2}{3} & & \Block{1-2}{4} & & \Block[fill=ColorRootNote]{1-2}{5} & & \Block{1-2}{6} & & \Block[fill=ColorOtherNote]{1-2}{7} & & \Block{1-2}{8} &
	\end{NiceTabular}
	\caption{Intervallen voor het 5e akkoord in de majeur toonladder}
	\label{tab:guitar_major_scale_5th_chord_identify}
\end{table}

Nu moeten we naar de intervallen tussen de noten kijken.

\begin{itemize}
	\item 5 to 7: 4 halve tonen (grote terts)
	\item 5 to 2: 7 halve tonen (reine kwint)
\end{itemize}

We hebben een grote terts en een reine kwint. Dit betekent dat het een majeurakkoord is.

\subsubsection{Een verminderde akkoord herkennen} \label{sec:identifying_dimished_chords_in_the_scale}

Een verminderde akkoord bestaat uit:

\begin{itemize}
	\item kleine terts: 3 halve tonen
	\item verminderde kwint: 6 halve tonen (ook wel tritonus genoemd)
\end{itemize}

\infobox{Het tritonus-interval wordt vaak als onrustig/dissonant ervaren. Daarom geeft een verminderde akkoord ook dat onrustige gevoel/spanning. De pentatonische toonladder kan hierbij helpen. Meer hierover later.}

De majeur- (en mineur-) toonladder heeft maar één verminderde akkoord. In de majeur toonladder is dat het akkoord op de 7e graad.

\begin{table}[h]
	\centering
	\begin{NiceTabular}{*{16}{P{0.05mm}}}
		\Block{}{} & \Block{1-2}{\large{W}} & & \Block{1-2}{\large{W}} & & \Block{1-2}{\large{H}} & & \Block{1-2}{\large{W}} & & \Block{1-2}{\large{W}} & & \Block{1-2}{\large{W}} & & \Block{1-2}{\large{H}} & & \Block{}{} \\
		\Block{1-2}{1} & & \Block[fill=ColorOtherNote]{1-2}{2} & & \Block{1-2}{3} & & \Block[fill=ColorOtherNote]{1-2}{4} & & \Block{1-2}{5} & & \Block{1-2}{6} & & \Block[fill=ColorRootNote]{1-2}{7} & & \Block{1-2}{8} &
	\end{NiceTabular}
	\caption{Intervallen voor het 7e akkoord in de majeur toonladder}
	\label{tab:guitar_major_scale_7th_chord_identify}
\end{table}

\begin{itemize}
	\item 7 to 2: 3 halve tonen (kleine terts)
	\item 7 to 4: 6 halve tonen (verminderde kwint)
\end{itemize}

We hebben een kleine terts en een verminderde kwint. Dit betekent dat het een verminderde akkoord is.

\newpage

\subsection{Nummers analyseren}
\subsubsection{Analyse van "Knockin' On Heaven's Door - Bob Dylan"}

In een vorige sectie speelde je "Knockin' On Heaven's Door" met de akkoorden G, D, C en Am. Er waren twee afwisselende reeksen:

\begin{itemize}
	\item G - D - Am
	\item G - D - C
\end{itemize}
 
Het nummer staat in de toonsoort G-majeur. De G-majeur toonladder staat in \autoref{tab:guitar_g_major_scale_with_chords}. De gemarkeerde akkoorden zijn die uit "Knockin' On Heaven's Door".

\begin{table}[h]
	\centering
	\begin{NiceTabular}{*{16}{P{0.05mm}}}
		\Block{}{} & \Block{1-2}{\large{W}} & & \Block{1-2}{\large{W}} & & \Block{1-2}{\large{H}} & & \Block{1-2}{\large{W}} & & \Block{1-2}{\large{W}} & & \Block{1-2}{\large{W}} & & \Block{1-2}{\large{H}} & & \Block{}{} \\
		\Block{1-2}{1} & & \Block{1-2}{2} & & \Block{1-2}{3} & & \Block{1-2}{4} & & \Block{1-2}{5} & & \Block{1-2}{6} & & \Block{1-2}{7} & & \Block{1-2}{8} & \\
		\Block{1-2}{\RomanNumeralCaps{1}} & & \Block{1-2}{\RomanNumeral{2}} & & \Block{1-2}{\RomanNumeral{3}} & & \Block{1-2}{\RomanNumeralCaps{4}} & & \Block{1-2}{\RomanNumeralCaps{5}} & & \Block{1-2}{\RomanNumeral{6}} & & \Block{1-2}{\RomanNumeral{7}\textsuperscript{o}} & & \\
		\Block[fill=ColorOtherNote]{1-2}{G} & & \Block[fill=ColorOtherNote]{1-2}{Am} & & \Block{1-2}{Bm} & & \Block[fill=ColorOtherNote]{1-2}{C} & & \Block[fill=ColorOtherNote]{1-2}{D} & & \Block{1-2}{Em} & & \Block{1-2}{F\sharp\textsuperscript{o}} & & \Block{1-2}{G} &
	\end{NiceTabular}
	\caption{G-majeur toonladder met akkoorden}
	\label{tab:guitar_g_major_scale_with_chords}
\end{table}

Met deze kennis kun je de akkoordreeksen beschrijven met Romeinse cijfers in plaats van akkoorden.

\begin{itemize}
	\item G - D - Am: \RomanNumeralCaps{1} - \RomanNumeralCaps{5} - \RomanNumeral{2}
	\item G - D - C: \RomanNumeralCaps{1} - \RomanNumeralCaps{5} - \RomanNumeralCaps{4} 
\end{itemize}

\textbf{Toonsoort veranderen}

Stel dat je zanger zich prettiger voelt in een andere toonsoort, of je wilt het nummer gewoon in een andere toonsoort spelen. Dan kun je de toonsoort van het nummer veranderen.

Als voorbeeld wordt "Knockin' On Heaven's Door" omgezet naar de toonsoort D-majeur. Door de Romeinse cijfers die we hebben bepaald toe te passen op de D-majeur toonladder, krijg je een nieuwe reeks akkoorden.

\begin{itemize}
	\item D - A - Em: \RomanNumeralCaps{1} - \RomanNumeralCaps{5} - \RomanNumeral{2}
	\item D - A - G: \RomanNumeralCaps{1} - \RomanNumeralCaps{5} - \RomanNumeralCaps{4} 
\end{itemize}

\begin{table}[h]
	\centering
	\begin{NiceTabular}{*{16}{P{0.05mm}}}
		\Block{}{} & \Block{1-2}{\large{W}} & & \Block{1-2}{\large{W}} & & \Block{1-2}{\large{H}} & & \Block{1-2}{\large{W}} & & \Block{1-2}{\large{W}} & & \Block{1-2}{\large{W}} & & \Block{1-2}{\large{H}} & & \Block{}{} \\
		\Block{1-2}{1} & & \Block{1-2}{2} & & \Block{1-2}{3} & & \Block{1-2}{4} & & \Block{1-2}{5} & & \Block{1-2}{6} & & \Block{1-2}{7} & & \Block{1-2}{8} & \\
		\Block{1-2}{\RomanNumeralCaps{1}} & & \Block{1-2}{\RomanNumeral{2}} & & \Block{1-2}{\RomanNumeral{3}} & & \Block{1-2}{\RomanNumeralCaps{4}} & & \Block{1-2}{\RomanNumeralCaps{5}} & & \Block{1-2}{\RomanNumeral{6}} & & \Block{1-2}{\RomanNumeral{7}\textsuperscript{o}} & & \\
		\Block[fill=ColorOtherNote]{1-2}{D} & & \Block[fill=ColorOtherNote]{1-2}{Em} & & \Block{1-2}{F\sharp m} & & \Block[fill=ColorOtherNote]{1-2}{G} & & \Block[fill=ColorOtherNote]{1-2}{A} & & \Block{1-2}{Bm} & & \Block{1-2}{C\sharp\textsuperscript{o}} & & \Block{1-2}{D} &
	\end{NiceTabular}
	\caption{D-majeur toonladder met akkoorden}
	\label{tab:guitar_D_major_scale_with_chords}
\end{table}

\subsubsection{Analyse van "Perfect - Ed Sheeran"}

Dit nummer staat in de toonsoort A$\flat$ (\autoref{tab:guitar_a_flat_major_scale_with_chords}).

\begin{table}[h]
	\centering
	\begin{NiceTabular}{*{16}{P{0.05mm}}}
		\Block{}{} & \Block{1-2}{\large{W}} & & \Block{1-2}{\large{W}} & & \Block{1-2}{\large{H}} & & \Block{1-2}{\large{W}} & & \Block{1-2}{\large{W}} & & \Block{1-2}{\large{W}} & & \Block{1-2}{\large{H}} & & \Block{}{} \\
		\Block{1-2}{1} & & \Block{1-2}{2} & & \Block{1-2}{3} & & \Block{1-2}{4} & & \Block{1-2}{5} & & \Block{1-2}{6} & & \Block{1-2}{7} & & \Block{1-2}{8} & \\
		\Block{1-2}{\RomanNumeralCaps{1}} & & \Block{1-2}{\RomanNumeral{2}} & & \Block{1-2}{\RomanNumeral{3}} & & \Block{1-2}{\RomanNumeralCaps{4}} & & \Block{1-2}{\RomanNumeralCaps{5}} & & \Block{1-2}{\RomanNumeral{6}} & & \Block{1-2}{\RomanNumeral{7}\textsuperscript{o}} & & \\
		\Block[fill=ColorOtherNote]{1-2}{A\flat} & & \Block{1-2}{B\flat m} & & \Block{1-2}{Cm} & & \Block[fill=ColorOtherNote]{1-2}{D\flat} & & \Block[fill=ColorOtherNote]{1-2}{E\flat} & & \Block[fill=ColorOtherNote]{1-2}{Fm} & & \Block{1-2}{G\textsuperscript{o}} & & \Block{1-2}{A\flat} &
	\end{NiceTabular}
	\caption{A$\flat$-majeur toonladder met akkoorden}
	\label{tab:guitar_a_flat_major_scale_with_chords}
\end{table}

In het eerste couplet van dit nummer (eerder getoond) wordt het volgende akkoordenschema gebruikt:

\begin{itemize}
	\item A$\flat$ - Fm - D$\flat$ - E$\flat$: \RomanNumeralCaps{1} - \RomanNumeral{6} - \RomanNumeralCaps{4} - \RomanNumeralCaps{5}
\end{itemize}

\newpage

\subsubsection{Afwijken van de toonladderakkoorden}

Natuurlijk zijn de dingen die tot nu toe zijn beschreven een goed startpunt, maar muziek is uiteindelijk een creatief proces. Dus experimenteer gerust.

Het nummer "Hey Ya!" van Outkast deed dat ook een beetje. Het nummer staat in de toonsoort G-majeur. De akkoordvolgorde die het hele nummer wordt gespeeld is:

\begin{itemize}
	\item G - C - D - E
\end{itemize}

Strikt genomen past dit niet in de akkoorden van de G-majeur toonladder. Je zou een Em-akkoord verwachten in plaats van een E (zie \autoref{tab:guitar_g_major_scale_with_chords}).

\subsection{Jouw beurt}

Als je een nummer speelt, probeer dan te kijken of je het akkoordenschema kunt vinden. Als je akkoorden op internet zoekt, zie je vaak ook de toonsoort van het nummer. Staat die er niet bij, dan kun je altijd het internet vragen of het de toonsoort van het nummer weet. Het komt vaak voor dat het eerste en/of laatste akkoord van een nummer hetzelfde is als de toonsoort.

Als je de toonsoort hebt, kijk dan of je de akkoorden in de toonladder van die toonsoort kunt plaatsen.

Als het internet de toonsoort niet kan geven, kijk dan hoeveel halve/hele tonen er tussen de akkoorden in het nummer zitten. Door deze intervallen te gebruiken kun je misschien het patroon in een toonladder herkennen.

% Knocking on heaven's door - bob dylan https://tabs.ultimate-guitar.com/tab/bob-dylan/knockin-on-heavens-door-chords-66559
% America - A Horse With No Name (1972) - Simple, but interesting chords https://tabs.ultimate-guitar.com/tab/america/a-horse-with-no-name-chords-59609
% Outkast - Hey Ya! https://tabs.ultimate-guitar.com/tab/outkast/hey-ya-chords-90698
% Narcotic Chords by Liquido