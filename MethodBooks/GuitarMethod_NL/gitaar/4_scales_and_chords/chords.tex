\section{Akkoorden}

\subsection{Akkoorden opbouwen} \label{sec:building-chords-chords-scales}
In de vorige secties hebben we geleerd over de majeur en mineur toonladders. Deze informatie kun je nu gebruiken om over akkoorden te leren.

\textbf{Een majeur- of mineurakkoord wordt gevormd door de 1e, 3e en 5e toontrap van een toonladder tegelijk te spelen. Dat is alles.}

Zoals te zien is in \autoref{tab:guitar_major_chord_buildup} en \autoref{tab:guitar_minor_chord_buildup}, worden in zowel majeur als mineur toonladders de 1e, 3e en 5e toonladderstap gebruikt. De groene noten zijn de grondtonen. Deze bepalen de naam van het akkoord. De oranje noten zijn de overige noten in het akkoord.

\begin{table}[h]
	\begin{minipage}{0.45\textwidth}
		\centering
		\begin{NiceTabular}{*{16}{P{0.05mm}}}
			\Block{}{} & \Block{1-2}{\large{W}} & & \Block{1-2}{\large{W}} & & \Block{1-2}{\large{H}} & & \Block{1-2}{\large{W}} & & \Block{1-2}{\large{W}} & & \Block{1-2}{\large{W}} & & \Block{1-2}{\large{H}} & & \Block{}{} \\
			\Block[fill=ColorRootNote]{1-2}{1} & & \Block{1-2}{2} & & \Block[fill=ColorOtherNote]{1-2}{3} & & \Block{1-2}{4} & & \Block[fill=ColorOtherNote]{1-2}{5} & & \Block{1-2}{6} & & \Block{1-2}{7} & & \Block{1-2}{8} & 
		\end{NiceTabular}
		\caption{Opbouw van een majeurakkoord}
		\label{tab:guitar_major_chord_buildup}
	\end{minipage}
	\hfill
	\begin{minipage}{0.45\textwidth}
		\centering
		\begin{NiceTabular}{*{16}{P{0.05mm}}}
			\Block{}{} & \Block{1-2}{\large{W}} & & \Block{1-2}{\large{H}} & & \Block{1-2}{\large{W}} & & \Block{1-2}{\large{W}} & & \Block{1-2}{\large{H}} & & \Block{1-2}{\large{W}} & & \Block{1-2}{\large{W}} & & \Block{}{} \\
			\Block[fill=ColorRootNote]{1-2}{1} & & \Block{1-2}{2} & & \Block[fill=ColorOtherNote]{1-2}{3$\flat$} & & \Block{1-2}{4} & & \Block[fill=ColorOtherNote]{1-2}{5} & & \Block{1-2}{6$\flat$} & & \Block{1-2}{7$\flat$} & & \Block{1-2}{8} &
		\end{NiceTabular}
		\caption{Opbouw van een mineurakkoord}
		\label{tab:guitar_minor_chord_buildup}
	\end{minipage}
\end{table}

\autoref{tab:guitar_major_chords_from_scales} en \autoref{tab:guitar_minor_chords_from_scales} laten de opbouw van andere akkoorden zien. De volgorde van de akkoorden in de majeur- en mineurtabellen is hetzelfde, zodat je ze goed kunt vergelijken. Deze noten zie je ook terug in de akkoorddiagrammen in \autoref{fig:guitar_major_minor_chords}.

Let op dat de 1e en 5e noot in zowel majeur als mineur akkoorden hetzelfde zijn. Alleen de 3e noot is altijd een halve toon lager in het mineurakkoord vergeleken met het majeurakkoord.

\begin{table}[h]
	\begin{minipage}{0.45\textwidth}
				\centering
		\begin{NiceTabular}{*{16}{P{0.05mm}}}
			\Block{}{} & \Block{1-2}{\large{W}} & & \Block{1-2}{\large{W}} & & \Block{1-2}{\large{H}} & & \Block{1-2}{\large{W}} & & \Block{1-2}{\large{W}} & & \Block{1-2}{\large{W}} & &  \Block{1-2}{\large{H}} & &  \Block{}{} \\
			\Block{1-2}{1} & & \Block{1-2}{2} & & \Block{1-2}{3} & & \Block{1-2}{4} & & \Block{1-2}{5} & & \Block{1-2}{6} & & \Block{1-2}{7} & & \Block{1-2}{8} \\
			\Block[fill=ColorRootNote]{1-2}{A} & & \Block{1-2}{B} & & \Block[fill=ColorOtherNote]{1-2}{C\sharp} & & \Block{1-2}{D} & & \Block[fill=ColorOtherNote]{1-2}{E} & & \Block{1-2}{F\sharp} & & \Block{1-2}{G\sharp} & & \Block{1-2}{A} \\
			\Block[fill=ColorRootNote]{1-2}{B} & & \Block{1-2}{C\sharp} & & \Block[fill=ColorOtherNote]{1-2}{D\sharp} & & \Block{1-2}{E} & & \Block[fill=ColorOtherNote]{1-2}{F\sharp} & & \Block{1-2}{G\sharp} & & \Block{1-2}{A\sharp} & & \Block{1-2}{B} \\
			\Block[fill=ColorRootNote]{1-2}{C} & & \Block{1-2}{D} & & \Block[fill=ColorOtherNote]{1-2}{E} & & \Block{1-2}{F} & & \Block[fill=ColorOtherNote]{1-2}{G} & & \Block{1-2}{A} & & \Block{1-2}{B} & & \Block{1-2}{C} \\
			\Block[fill=ColorRootNote]{1-2}{D} & & \Block{1-2}{E} & & \Block[fill=ColorOtherNote]{1-2}{F\sharp} & & \Block{1-2}{G} & & \Block[fill=ColorOtherNote]{1-2}{A} & & \Block{1-2}{B} & & \Block{1-2}{C\sharp} & & \Block{1-2}{D} \\
			\Block[fill=ColorRootNote]{1-2}{E} & & \Block{1-2}{F\sharp} & & \Block[fill=ColorOtherNote]{1-2}{G\sharp} & & \Block{1-2}{A} & & \Block[fill=ColorOtherNote]{1-2}{B} & & \Block{1-2}{C\sharp} & & \Block{1-2}{D\sharp} & & \Block{1-2}{E} \\
			\Block[fill=ColorRootNote]{1-2}{F} & & \Block{1-2}{G} & & \Block[fill=ColorOtherNote]{1-2}{A} & & \Block{1-2}{B\flat} & & \Block[fill=ColorOtherNote]{1-2}{C} & & \Block{1-2}{D} & & \Block{1-2}{E} & & \Block{1-2}{F} \\
			\Block[fill=ColorRootNote]{1-2}{G} & & \Block{1-2}{A} & & \Block[fill=ColorOtherNote]{1-2}{B} & & \Block{1-2}{C} & & \Block[fill=ColorOtherNote]{1-2}{D} & & \Block{1-2}{E} & & \Block{1-2}{F\sharp} & & \Block{1-2}{G} \\
		\end{NiceTabular}
		\caption{Majeurakkoorden uit de majeur toonladder}
		\label{tab:guitar_major_chords_from_scales}
	\end{minipage}
	\hfill
	\begin{minipage}{0.45\textwidth}
		\centering
		\begin{NiceTabular}{*{16}{P{0.05mm}}}
			\Block{}{} & \Block{1-2}{\large{W}} & & \Block{1-2}{\large{H}} & & \Block{1-2}{\large{W}} & & \Block{1-2}{\large{W}} & & \Block{1-2}{\large{H}} & & \Block{1-2}{\large{W}} & & \Block{1-2}{\large{W}} & & \Block{}{} \\
			\Block{1-2}{1} & & \Block{1-2}{2} & & \Block{1-2}{3$\flat$} & & \Block{1-2}{4} & & \Block{1-2}{5} & & \Block{1-2}{6$\flat$} & & \Block{1-2}{7$\flat$} & & \Block{1-2}{8} \\
			\Block[fill=ColorRootNote]{1-2}{ A} & & \Block{1-2}{B} & & \Block[fill=ColorOtherNote]{1-2}{ C} & & \Block{1-2}{D} & & \Block[fill=ColorOtherNote]{1-2}{ E} & & \Block{1-2}{F} & & \Block{1-2}{G} & & \Block{1-2}{A} \\
			\Block[fill=ColorRootNote]{1-2}{ B} & & \Block{1-2}{C\sharp} & & \Block[fill=ColorOtherNote]{1-2}{ D} & & \Block{1-2}{E} & & \Block[fill=ColorOtherNote]{1-2}{ F\sharp} & & \Block{1-2}{G} & & \Block{1-2}{A} & & \Block{1-2}{B} \\
			\Block[fill=ColorRootNote]{1-2}{ C} & & \Block{1-2}{D} & & \Block[fill=ColorOtherNote]{1-2}{ E\flat} & & \Block{1-2}{F} & & \Block[fill=ColorOtherNote]{1-2}{ G} & & \Block{1-2}{A\flat} & & \Block{1-2}{B\flat} & & \Block{1-2}{C} \\
			\Block[fill=ColorRootNote]{1-2}{ D} & & \Block{1-2}{E} & & \Block[fill=ColorOtherNote]{1-2}{ F} & & \Block{1-2}{G} & & \Block[fill=ColorOtherNote]{1-2}{ A} & & \Block{1-2}{B\flat} & & \Block{1-2}{C} & & \Block{1-2}{D} \\
			\Block[fill=ColorRootNote]{1-2}{ E} & & \Block{1-2}{F\sharp} & & \Block[fill=ColorOtherNote]{1-2}{ G} & & \Block{1-2}{A} & & \Block[fill=ColorOtherNote]{1-2}{ B} & & \Block{1-2}{C} & & \Block{1-2}{D} & & \Block{1-2}{E} \\
			\Block[fill=ColorRootNote]{1-2}{ F} & & \Block{1-2}{G} & & \Block[fill=ColorOtherNote]{1-2}{ A\flat} & & \Block{1-2}{B\flat} & & \Block[fill=ColorOtherNote]{1-2}{ C} & & \Block{1-2}{D\flat} & & \Block{1-2}{E\flat} & & \Block{1-2}{F} \\
			\Block[fill=ColorRootNote]{1-2}{ G} & & \Block{1-2}{A} & & \Block[fill=ColorOtherNote]{1-2}{ B\flat} & & \Block{1-2}{C} & & \Block[fill=ColorOtherNote]{1-2}{ D} & & \Block{1-2}{E\flat} & & \Block{1-2}{F} & & \Block{1-2}{G} \\
		\end{NiceTabular}
		\caption{Mineurakkoorden uit de mineur toonladder}
		\label{tab:guitar_minor_chords_from_scales}
	\end{minipage}
\end{table}

\newpage

\subsection{Open en barré-akkoorden}

Als een akkoord wordt gespeeld met open snaren, heet dat een "open akkoord" (zie \autoref{fig:guitar-open-c-chord-hand-position}). Als een akkoord zonder open snaren wordt gespeeld, heet dat een "gesloten akkoord" of "barré-akkoord" (zie \autoref{fig:guitar-barre-f-chord-hand-position}). Een barré-akkoord is dus een type gesloten akkoord.

\begin{figure}[h]
	\begin{subfigure}[b]{0.45\textwidth}
		\includegraphics[width=\textwidth]{../../Images/open-c-chord.jpg}
		\caption{Open C akkoord \cite{OpenCChordHand}}
		\label{fig:guitar-open-c-chord-hand-position}
	\end{subfigure}
	\hfill
	\begin{subfigure}[b]{0.45\textwidth}
		\includegraphics[width=\textwidth]{../../Images/F-major-barre-photo-768x768.jpg}
		\caption{Barré F akkoord \cite{BarreFChordHand}}
		\label{fig:guitar-barre-f-chord-hand-position}
	\end{subfigure}
	\caption{}
\end{figure}


Het mooie van gesloten/barré-akkoorden is dat je ze over de hals kunt verschuiven. Op dat moment wordt het meer een vorm dan een specifiek akkoord. Afhankelijk van de positie van de grondtoon krijgt het barré-akkoord een andere naam. Dit zie je later terug in het \textbf{CAGED-systeem}.

Op de volgende pagina in \autoref{fig:guitar_major_minor_chords} zie je alle majeur- en mineurakkoorden. Het akkoord \textbf{C} is een majeurakkoord en \textbf{Cm} is een mineurakkoord. Dit geldt ook voor de andere akkoorden. Onder elk akkoord staan de 1e, 3e en 5e noten uit de betreffende toonladder die het akkoord vormen (zie \autoref{tab:guitar_major_chords_from_scales} en \autoref{tab:guitar_minor_chords_from_scales}).

De groene stippen geven de grondtoon aan. Deze noot bepaalt de naam van het akkoord.

Een paar dingen om op te letten:

\begin{itemize}
	\item De grondtoon en de 5e noot van een toonladder zijn hetzelfde voor zowel de majeur- als de mineurvariant.
	\item De 3e noot van een mineurakkoord is altijd een halve toon lager dan in het majeurakkoord.
\end{itemize}

\newpage

% Billie eilish Therefore I am: Dm A - i V
% Robin Thicke Blurred Lines: G D - I V
% Arctic Monkeys - 505: Dm Em - i ii


\begin{figure}[h]
	\centering
	\includegraphics[height=0.8\textheight]{../../Images/GuitarBasicChords.png}
	\caption{Majeur- en mineurakkoorden}
	\label{fig:guitar_major_minor_chords}
\end{figure}

\clearpage

Laten we wat akkoorden spelen. Het openingsnummer van de serie Adventure Time is een goed begin (\autoref{fig:guitar_adventure_time}). De noten op de notenbalk zijn hier vervangen door ritmenotatie. De duur van de notenvormen blijft hetzelfde, maar nu geeft het alleen het slagritme aan.

Het kleine symbool boven de notenbalk dat eruitziet als een vierkant met een open onderkant betekent een neerwaartse slag. Je speelt het akkoord door je hand/plectrum naar beneden door de snaren te bewegen. Het symbool dat eruitziet als een "V" is een opwaartse slag. Je beweegt je hand/plectrum dan omhoog door de snaren.

\begin{figure}[h]
	\centering
	\includegraphics[width=\textwidth]{../../MuseScore/Guitar/GuitarAdventureTimeTheme.png}
	\caption{Adventure Time openingsnummer}
	\label{fig:guitar_adventure_time}
\end{figure}

\newpage

In het nummer "Knockin' On Heaven's Door" van Bob Dylan worden dezelfde akkoorden gebruikt als in het Adventure Time openingsnummer, plus één extra akkoord: \textbf{Am}.

\begin{song}[verse/numbered, remember-chords, align-chords=l]{title={Knockin' On Heaven's Door - Bob Dylan}, music={Bob Dylan}}
	\begin{intro}	
		^{G}   ^{D}      ^{Am}     ^{G}Oo ^{D}oo-oo ^{C}oo \\
		^{G}Oo ^{D}oo-oo ^{Am}oo   ^{G}Oo ^{D}oo-oo ^{C}oo \\
	\end{intro}
	\begin{verse}
		^{G}Mama, take this ^{D}badge off of me ^{Am} \\
		^{G}I can’t ^{D}use it anymore ^{C} \\
		^{G}It’s gettin’ ^{D}dark, too dark for me to see ^{Am} \\
		^{G}I feel like I’m ^{D}knockin’ on heaven’s door ^{C} \\
	\end{verse}
	\begin{chorus}
		^{G}Knock, knock, ^{D}knockin’ on heaven’s ^{Am}door \\
		^{G}Knock, knock, ^{D}knockin’ on heaven’s ^{C}door \\
		^{G}Knock, knock, ^{D}knockin’ on heaven’s ^{Am}door \\
		^{G}Knock, knock, ^{D}knockin’ on heaven’s ^{C}door \\
	\end{chorus}
	\begin{verse}
		^Mama, put my ^guns in the ground ^ {} \\
		^I can’t ^shoot them anymore ^ {} \\
		^That long black ^cloud is comin’ down ^ {} \\
		^I feel like I’m ^knockin’ on heaven’s door ^ {} \\
	\end{verse}
	\begin{chorus}
		^Knock, knock, ^knockin’ on heaven’s ^door \\
		^Knock, knock, ^knockin’ on heaven’s ^door \\
		^Knock, knock, ^knockin’ on heaven’s ^door \\
		^Knock, knock, ^knockin’ on heaven’s ^door \\
	\end{chorus}
\end{song}

\newpage

Een ander nummer om akkoordwisselingen mee te oefenen is "Hey Ya!" van "Outkast". Dit gebruikt slechts vier akkoorden, en de volgorde blijft hetzelfde gedurende het hele nummer.

Om een idee te krijgen van de akkoorden, is het eerste deel van het nummer hier weergegeven. Je kunt het nummer beluisteren en deze akkoorden voor de rest van het lied spelen.

\begin{song}[verse/numbered, align-chords=l]{title={Hey Ya! - Outkast}, music={Outkast}}
	\begin{intro}
		One, two, three, uh!
	\end{intro}
	\begin{verse}
		^{G}My baby don't ^{C}mess around \\
		Because she loves me so, and this ^{D}I know for ^{E}sure (Uh) \\
		^{G}But does she ^{C}really wanna \\
		But can't stand to see me walk ^{D}out the ^{E}door? (Ah) \\
	\end{verse}
\end{song}

\newpage

There are two (actually 4) more important shapes to learn. The closed barre shapes. These are shown in \autoref{fig:guitar_major_minor_chords} as the F, Fm, B, and Bm chords. For these chords you place your index finger over all the strings, and use the remaining fingers to press the remaining notes. Note that for the B chord you only have to place your index finger over the first 5 strings.

The song "Perfect" by Ed Sheeran is a good song to practice these shapes. This also shows the power of barre chords. The fact that they can be moved up and down the neck to make different chords.

The song uses 4 chords: A$\flat$, Fm, D$\flat$, and E$\flat$. Or if shown with sharps: G$\sharp$, Fm, C$\sharp$, and D$\sharp$.

Only the first verse is shown here to focus on the barre chords themselves. The barre chords to use are shown in \autoref{fig:guitar_barre_chords_perfect_ed_sheeran}. Note the numbers below the shapes. These are the fret numbers.


\begin{figure}[h]
	\centering
	\includegraphics[width=0.8\textwidth]{../../Images/ChordsUsedInPerfectEdSheeran.png}
	\caption{Barré-akkoorden gebruikt in "Perfect - Ed Sheeran"}
	\label{fig:guitar_barre_chords_perfect_ed_sheeran}
\end{figure}


Let op hoe de vormen allemaal erg op elkaar lijken, maar dat de noot op de laagste snaar aangeeft welk akkoord het is.

\infobox{Het is geen regel dat de noot op de laagste snaar altijd het akkoord bepaalt voor alle vormen. De combinatie van noten bepaalt het akkoord. Hier leer je later meer over.}

\begin{song}[verse/numbered, align-chords=l]{title={Perfect - Ed Sheeran}, music={Ed Sheeran}}
	\begin{verse}
		I found a ^{Ab}love for ^{Fm}me \\
		Oh, darling, just ^{Db}dive right in and follow my ^{Eb}lead \\
		Well, I found a ^{Ab}girl, beauti^{Fm}ful and sweet \\
		Oh, I never ^{Db}knew you were the someone waitin' for^{Eb}me \\
	\end{verse}
\end{song}

\subsection{Jouw beurt}

We hebben nu allerlei verschillende akkoorden gespeeld. Het is aan jou om te kijken welk nummer je wilt spelen, zoek de akkoorden op internet en oefen de overgang tussen de akkoorden. Speel gerust de akkoorden op verschillende manieren/vormen. Elke optie geeft een ander geluid, of misschien is de ene optie makkelijker te spelen om snel naar een ander akkoord te gaan dan de andere. Experimenteer gewoon!