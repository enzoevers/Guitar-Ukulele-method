\section{Waarom toonladders leren?}

\begin{itemize}
    \item Akkoorden zijn opgebouwd uit toonladders
    \item Akkoordenprogressies zijn gebaseerd op toonladders
    \item Improviseren
    \item Interessante melodieën maken
    \item Liedjes makkelijker op gehoor leren
    \item En meer...
\end{itemize}

Toonladders leren klinkt misschien saai. En als je het zonder context leert, kan het inderdaad saai zijn. Maar toonladders zijn een raamwerk waarmee je interessante en goed klinkende combinaties van noten kunt maken.

Als een huis geen goede fundering heeft, kan het gemakkelijk worden verwoest door natuurverschijnselen. Maar een goede fundering voor het huis betekent niet dat je geen prachtige sprookjestuin of een mooi schilderij op de muren kunt maken.

Hetzelfde geldt voor muziek. Toonladders geven je een basis en een raamwerk om mee te werken. Een set noten die goed samen klinken in een bepaalde context. Maar je kunt, en moet, creatief zijn met hoe je dit raamwerk gebruikt. Noten buiten de toonladder gebruiken kan ook interessante klanken geven als je ze goed toepast.

\section{Context creeëren: je eerste akkoorden spelen}

Kun je niet wachten om akkoorden te spelen? Kijk dan gerust naar \autoref{fig:guitar_major_minor_chords} om de basisvormen van akkoorden te leren. Als je op internet zoekt naar \textit{"<nummer naam> <artiest> akkoorden"}, vind je meestal een website met de akkoorden die je kunt spelen en die min of meer kloppen.

Alleen al het leren van deze akkoorden brengt je al ver als het gaat om meespelen met liedjes.

Maar zou het niet fijn zijn om ook liedjes op gehoor uit te kunnen zoeken (of foute tabs/akkoorden die je op internet vindt te corrigeren), akkoorden interessanter te laten klinken door ze op verschillende posities te spelen, te kunnen improviseren over een liedje en je eigen akkoordprogressies te maken? Zo ja, ga dan verder met de volgende secties.

\newpage

\section{Toonladders en intervallen}

Een toonladder is een verzameling noten in oplopende volgorde tussen een noot en zijn octaaf. De twee belangrijkste toonladders zijn de diatonische majeur (blije/heldere klank) en diatonische mineur (triste/donkere klank) toonladder.

Bij het beschrijven van toonladders worden vaak de termen "heel" (W van het engelse "whole") en "half" (H) gebruikt. Soms zie je ook de termen "toon" (T) en "halftoon" (S van het engelse "semitone").

Een halve stap omhoog op de gitaar betekent dat je naar de volgende fret gaat (richting de body van de gitaar). Een hele stap omhoog is hetzelfde als twee halve stappen. Een halve stap omlaag betekent één fret terug (richting de kop).

Laten we nog eens kijken naar de intervallen van één halve toon (\autoref{tab:guitar_sharp_flat_intervals_chap_5}). Eén positie naar links of rechts is een halftoon interval. Voor een hele toon neem je gewoon twee halftonen.

\begin{table}[h]
	\centering
	\begin{NiceTabular}{*{12}{P{5mm}}}
		\large{A} & \large{A\sharp} & \large{B} & \large{C} & \large{C\sharp} & \large{D} & \large{D\sharp} & \large{E} & \large{F} & \large{F\sharp} & \large{G} & \large{G\sharp} \\ \\
		\large{A} & \large{B\flat} & \large{B} & \large{C} & \large{D\flat} & \large{D} & \large{E\flat} & \large{E} & \large{F} & \large{G\flat}& \large{G} & \large{A\flat}
	\end{NiceTabular}
	\caption{Kruis- en molintervallen. Elke stap naar links of rechts is een halftoon.}
	\label{tab:guitar_sharp_flat_intervals_chap_5}
\end{table}

Een lijst van intervallen en hun namen staat in \autoref{tab:guitar_intervals_in_octave}. Deze worden gebruikt bij het beschrijven van toonladders en het maken van akkoorden.

\begin{table}[h]
	\centering
	\begin{NiceTabular}{|c|c|c|}
		\hline
		\textbf{Halftonen} & \textbf{Naam} & \textbf{Symbool} \\
		\hline
		\hline
		0 & Reine prime & 1 \\
		\hline
		1 & Kleine secunde & 2$\flat$ \\
		\hline
		2 & Grote secunde & 2 \\
		\hline
		3 & Kleine terts & 3$\flat$ \\
		\hline
		4 & Grote terts & 3 \\
		\hline
		5 & Reine kwart & 4 \\
		\hline
		6 & Overmatige kwart / Verminderde kwint / Tritonus & 4$\sharp$ / 5$\flat$ \\
		\hline
		7 & Reine kwint & 5 \\
		\hline
		8 & Overmatige kwint / Kleine sext & 5$\sharp$ / 6$\flat$ \\
		\hline
		9 & Grote sext / Verminderde septiem & 6 / 7$\flatflat$ \\
		\hline
		10 & Kleine septiem & 7$\flat$ \\
		\hline
		11 & Grote septiem & 7 \\
		\hline
		12 & Octaaf & 8 \\
		\hline
	\end{NiceTabular}
	\caption{Intervallen binnen een octaaf}
	\label{tab:guitar_intervals_in_octave}
\end{table}

Speel \autoref{fig:guitar_intervals_in_octave} om te horen hoe deze intervallen klinken.

\begin{figure}[h]
	\centering
	\includegraphics[width=\textwidth]{../../MuseScore/Guitar/IntervalsInOctave.png}
	\caption{Intervallen binnen een octaaf}
	\label{fig:guitar_intervals_in_octave}
\end{figure}

\infobox{Zie hoe, wanneer je van de laagste naar de hoogste snaar gaat (behalve van G naar B), elk paar snaren een \textbf{reine kwart} interval heeft. Het interval tussen de G- en B-snaren is een \textbf{grote terts}.}

\newpage

\subsection{De majeur toonladder}

Zoals eerder genoemd zijn de meest voorkomende toonladders de majeur en mineur toonladder. Veel muziektheorie is gebaseerd op de diatonische majeur toonladder. Een diatonische toonladder betekent dat deze 7 verschillende noten heeft, gescheiden door vijf hele en twee halve stappen, waarbij elke letter slechts één keer voorkomt. De majeur toonladder is de eerste die we gaan leren.

Elke toonladder heeft een formule. Voor de diatonische majeur toonladder zie je de formule in (\autoref{tab:guitar_major_scale_interval}). Bovenaan zie je de stappen tussen elke noot (de formule zelf). De cijfers geven het interval van de noot in de toonladder aan. Noot 1 en 8 zijn dezelfde noot, maar de 8 is één octaaf hoger dan de 1.

\begin{table}[h]
	\centering
	\begin{NiceTabular}{*{16}{P{0.05mm}}}
		\Block{}{} & \Block{1-2}{\large{W}} & & \Block{1-2}{\large{W}} & & \Block{1-2}{\large{H}} & & \Block{1-2}{\large{W}} & & \Block{1-2}{\large{W}} & & \Block{1-2}{\large{W}} & & \Block{1-2}{\large{H}} & & \Block{}{} \\
		\Block{1-2}{1} & & \Block{1-2}{2} & & \Block{1-2}{3} & & \Block{1-2}{4} & & \Block{1-2}{5} & & \Block{1-2}{6} & & \Block{1-2}{7} & & \Block{1-2}{8} &
	\end{NiceTabular}
	\caption{Majeur toonladder intervallen}
	\label{tab:guitar_major_scale_interval}
\end{table}

Let op dat \autoref{tab:guitar_sharp_flat_intervals_chap_5} 12 verschillende noten/tonen heeft. Tel nu het totale aantal halve stappen in \autoref{tab:guitar_major_scale_interval} (een hele stap is twee halve stappen). Inderdaad, er zijn 12 halve stappen van noot 1 naar dezelfde noot één octaaf hoger (noot 8).

Om bijvoorbeeld de C-majeur toonladder te maken, begin je op de C en volg je gewoon de formule. Een mogelijke manier om dit op de gitaar te spelen zie je in \autoref{fig:guitar_c_major_scale_first_position}. Hier groepeert een kleur de noten in één octaaf.

\begin{table}[h]
	\centering
	\begin{NiceTabular}{*{16}{P{0.05mm}}}
		\Block{}{} & \Block{1-2}{\large{W}} & & \Block{1-2}{\large{W}} & & \Block{1-2}{\large{H}} & & \Block{1-2}{\large{W}} & & \Block{1-2}{\large{W}} & & \Block{1-2}{\large{W}} & & \Block{1-2}{\large{H}} & & \Block{}{} \\
		\Block{1-2}{1} & & \Block{1-2}{2} & & \Block{1-2}{3} & & \Block{1-2}{4} & & \Block{1-2}{5} & & \Block{1-2}{6} & & \Block{1-2}{7} & & \Block{1-2}{8} & \\
		\Block{1-2}{C} & & \Block{1-2}{D} & & \Block{1-2}{E} & & \Block{1-2}{F} & & \Block{1-2}{G} & & \Block{1-2}{A} & & \Block{1-2}{B} & & \Block{1-2}{C} & 
	\end{NiceTabular}
	\caption{C-majeur toonladder}
	\label{tab:guitar_c_major_scale}
\end{table}

\begin{figure}[h]
	\centering
	\includegraphics[height=0.2\textheight]{../../Images/CMajorScaleFirstPosition.png}
	\caption{C-majeur toonladder met notennamen en intervallen}
	\label{fig:guitar_c_major_scale_first_position}
\end{figure}

\iffalse
The G major scale is shown below in (\autoref{tab:guitar_g_major_scale}).

\begin{table}[h]
	\centering
	\begin{NiceTabular}{*{16}{P{0.05mm}}}
		\Block{}{} & \Block{1-2}{\large{W}} & & \Block{1-2}{\large{W}} & & \Block{1-2}{\large{H}} & & \Block{1-2}{\large{W}} & & \Block{1-2}{\large{W}} & & \Block{1-2}{\large{W}} & & \Block{1-2}{\large{H}} & & \Block{}{} \\
		\Block{1-2}{1} & & \Block{1-2}{2} & & \Block{1-2}{3} & & \Block{1-2}{4} & & \Block{1-2}{5} & & \Block{1-2}{6} & & \Block{1-2}{7} & & \Block{1-2}{8} & \\
		\Block{1-2}{G} & & \Block{1-2}{A} & & \Block{1-2}{B} & & \Block{1-2}{C} & & \Block{1-2}{D} & & \Block{1-2}{E} & & \Block{1-2}{F\sharp} & & \Block{1-2}{G} & 
	\end{NiceTabular}
	\caption{G major scale}
	\label{tab:guitar_g_major_scale}
\end{table}
\fi

In \autoref{tab:guitar_natural_note_major_scale} zie je de majeur toonladders van alle natuurlijke noten. Je hoeft deze nu nog niet uit je hoofd te kennen. Je moet wel de formule van de majeur toonladder uit je hoofd leren. Er zijn drie dingen om op te letten:

\begin{enumerate}
	\item \textbf{Elke toonladder heeft alleen unieke letters}. Daarom is de 4e noot in de F-majeur toonladder een B$\flat$ en geen A$\sharp$.
	\item De 5e noot in de toonladder is het begin van de toonladder op de volgende rij. Dat komt nu door de manier van opschrijven van de rijen, maar het is de basis van de "kwintencirkel" waar we later meer over leren (\autoref{sec:circle_of_fifths}).
	\item Elke toonladder onder een andere in deze lijst heeft één $\sharp$ meer dan de vorige. En de noten die een $\sharp$ hebben in één toonladder, hebben dat ook in de toonladders eronder. Dit heeft te maken met de kwintencirkel. Er is een ezelsbruggetje voor de volgorde van de toe te voegen $\sharp$s: \textbf{F}ijne \textbf{C}harles \textbf{G}aat \textbf{D}agelijks \textbf{A}ppeltaart \textbf{E}ten \textbf{B}innen. Meer hierover in \autoref{sec:circle_of_fifths}.
\end{enumerate}

\begin{table}[h]
	\centering
	\begin{NiceTabular}{*{16}{P{0.05mm}}}
		\Block{}{} & \Block{1-2}{\large{W}} & & \Block{1-2}{\large{W}} & & \Block{1-2}{\large{H}} & & \Block{1-2}{\large{W}} & & \Block{1-2}{\large{W}} & & \Block{1-2}{\large{W}} & & \Block{1-2}{\large{H}} & & \Block{}{} \\
		\Block{1-2}{1} & & \Block{1-2}{2} & & \Block{1-2}{3} & & \Block{1-2}{4} & & \Block{1-2}{5} & & \Block{1-2}{6} & & \Block{1-2}{7} & & \Block{1-2}{8} & \\
		\Block{1-2}{F} & & \Block{1-2}{G} & & \Block{1-2}{A} & & \Block{1-2}{B\flat} & & \Block{1-2}{C} & & \Block{1-2}{D} & & \Block{1-2}{E} & & \Block{1-2}{F} & \\
		\Block{1-2}{C} & & \Block{1-2}{D} & & \Block{1-2}{E} & & \Block{1-2}{F} & & \Block{1-2}{G} & & \Block{1-2}{A} & & \Block{1-2}{B} & & \Block{1-2}{C} & \\
		\Block{1-2}{G} & & \Block{1-2}{A} & & \Block{1-2}{B} & & \Block{1-2}{C} & & \Block{1-2}{D} & & \Block{1-2}{E} & & \Block{1-2}{F\sharp} & & \Block{1-2}{G} & \\
		\Block{1-2}{D} & & \Block{1-2}{E} & & \Block{1-2}{F\sharp} & & \Block{1-2}{G} & & \Block{1-2}{A} & & \Block{1-2}{B} & & \Block{1-2}{C\sharp} & & \Block{1-2}{D} & \\
		\Block{1-2}{A} & & \Block{1-2}{B} & & \Block{1-2}{C\sharp} & & \Block{1-2}{D} & & \Block{1-2}{E} & & \Block{1-2}{F\sharp} & & \Block{1-2}{G\sharp} & & \Block{1-2}{A} & \\
		\Block{1-2}{E} & & \Block{1-2}{F\sharp} & & \Block{1-2}{G\sharp} & & \Block{1-2}{A} & & \Block{1-2}{B} & & \Block{1-2}{C\sharp} & & \Block{1-2}{D\sharp} & & \Block{1-2}{E} & \\
		\Block{1-2}{B} & & \Block{1-2}{C\sharp} & & \Block{1-2}{D\sharp} & & \Block{1-2}{E} & & \Block{1-2}{F\sharp} & & \Block{1-2}{G\sharp} & & \Block{1-2}{A\sharp} & & \Block{1-2}{B} & 
	\end{NiceTabular}
	\caption{Majeur toonladders van alle natuurlijke noten}
	\label{tab:guitar_natural_note_major_scale}
\end{table}

\newpage

\subsubsection{Voorbeeld}

In \autoref{fig:guitar_major_scale_fretboard} zie je verschillende vormen waarop de (F\sharp) majeur toonladder gespeeld kan worden. De cijfers komen overeen met het interval in de toonladder. Deze vormen kun je over de hals omhoog en omlaag schuiven, zolang de afstand tussen de frets gelijk blijft. Vorm \autoref{fig:guitar_major_scale_fretboard_single_string} kun je zelfs op verschillende snaren gebruiken. Door de vorm te verplaatsen speel je een andere majeur toonladder. Welke toonladder je speelt wordt bepaald door de grondnoot (de "1" noot). In dit voorbeeld speel je dus de F$\sharp$ majeur toonladder. Als je alle noten één fret opschuift, speel je de G-majeur toonladder.

De verschillende kleuren in \autoref{fig:guitar_major_scale_fretboard} geven verschillende octaven aan. Let op de frets met "8/1", die geven de 8 van het vorige octaaf en de 1 van het volgende octaaf aan.

Er zijn ook andere "vormen" om de majeur toonladder te spelen, maar die beginnen niet op de grondnoot (1). Daar komen we later op terug.

Het uit je hoofd leren van deze vormen maakt het makkelijk om te improviseren over een liedje. Maar belangrijker is om te zien hoe deze vormen zich verhouden tot de intervallen van de majeur toonladder. De makkelijkste vorm hiervoor is \autoref{fig:guitar_major_scale_fretboard_single_string}. Met deze vorm herken je eenvoudig de diatonische majeur formule (w-w-h-w-w-w-h). Alle vormen bevatten dezelfde noten, alleen op een andere plek op de hals en soms in een ander octaaf. De vormen hieronder zijn nog niet alle mogelijkheden.

De F$\sharp$ majeur toonladder heeft de noten zoals te zien in \autoref{tab:guitar_f_sharp_major_scale}. Al die kruizen lijken misschien veel, maar volg gewoon de formule vanaf F$\sharp$ en je komt op dezelfde noten uit.

\begin{table}[h]
	\centering
	\begin{NiceTabular}{*{16}{P{0.05mm}}}
		\Block{}{} & \Block{1-2}{\large{W}} & & \Block{1-2}{\large{W}} & & \Block{1-2}{\large{H}} & & \Block{1-2}{\large{W}} & & \Block{1-2}{\large{W}} & & \Block{1-2}{\large{W}} & & \Block{1-2}{\large{H}} & & \Block{}{} \\
		\Block{1-2}{1} & & \Block{1-2}{2} & & \Block{1-2}{3} & & \Block{1-2}{4} & & \Block{1-2}{5} & & \Block{1-2}{6} & & \Block{1-2}{7} & & \Block{1-2}{8} & \\
		\Block{1-2}{F$\sharp$} & & \Block{1-2}{G$\sharp$} & & \Block{1-2}{A$\sharp$} & & \Block{1-2}{B} & & \Block{1-2}{C$\sharp$} & & \Block{1-2}{D$\sharp$} & & \Block{1-2}{E$\sharp$} & & \Block{1-2}{F$\sharp$} & 
	\end{NiceTabular}
	\caption{F$\sharp$ majeur toonladder}
	\label{tab:guitar_f_sharp_major_scale}
\end{table}

\autoref{tab:guitar_f_sharp_major_scale} laat een belangrijk punt zien dat eerder genoemd is. \textbf{Een toonladder heeft alleen unieke letters}. Let op dat de 7e toontrap een E$\sharp$ is en geen F. Normaal zou je E$\sharp$ niet opschrijven, maar omdat de F$\sharp$ majeur toonladder al de letter F gebruikt voor F$\sharp$, kun je F niet nog een keer gebruiken. Daarom moet je E$\sharp$ gebruiken.

\autoref{fig:guitar_f_sharp_major_scale_score_multi_octave} toont de noten die horen bij \autoref{fig:guitar_major_scale_octaves_over_fretboard}.

\begin{figure}[h]
	\centering
	\includegraphics[width=\textwidth]{../../MuseScore/Guitar/GuitarFSharpMajorMultiOctave.png}
	\caption{Meerdere octaven van de F$\sharp$ majeur toonladder}
	\label{fig:guitar_f_sharp_major_scale_score_multi_octave}
\end{figure}

\newpage

\begin{figure}[h]
	\begin{subfigure}[b]{0.45\textwidth}
		\centering
		\includegraphics[height=0.18\textheight]{../../Images/guitar_major_scale_standard.png}
		\caption{Majeur toonladder op het fretbord (standaard)}
		\label{fig:guitar_major_scale_fretboard_standard}
	\end{subfigure}
	\hfill
	\begin{subfigure}[b]{0.45\textwidth}
		\centering
		\includegraphics[height=0.18\textheight]{../../Images/guitar_major_scale_standard_start_5th_string.png}
		\caption{Majeur toonladder op het fretbord vanaf de 5e (A) snaar}
		\label{fig:guitar_major_scale_fretboard_standard_start_5th_string}
	\end{subfigure}
    
	\vspace{0.5cm}
    
	\begin{subfigure}[b]{\textwidth}
		\centering
		\includegraphics[height=0.18\textheight]{../../Images/guitar_major_scale_single_string.png}
		\caption{Majeur toonladder op het fretbord op één snaar}
		\label{fig:guitar_major_scale_fretboard_single_string}
	\end{subfigure}
    
	\vspace{0.5cm}
    
	\begin{subfigure}[b]{\textwidth}
		\centering
		\includegraphics[height=0.18\textheight]{../../Images/guitar_major_scale_octaves_over_fretboard.png}
		\caption{Majeur toonladder octaven over het hele fretbord}
		\label{fig:guitar_major_scale_octaves_over_fretboard}
	\end{subfigure}
    
	\caption{F$\sharp$ majeur toonladder op het fretbord}
	\label{fig:guitar_major_scale_fretboard}
\end{figure}

\clearpage

Het nummer "Tattoo" van "Loreen" gebruikt noten uit de F$\sharp$ majeur toonladder. De melodie die je op de achtergrond hoort, is te zien in \autoref{fig:guitar_tattoo_loreen_main_backing_melody}. Let ook op dat de akkoorden allemaal binnen de F$\sharp$ majeur toonladder vallen.

\begin{figure}[h]
\centering
\includegraphics[width=\textwidth]{../../MuseScore/Guitar/LoreenTattooSimpleBackingMelody.png}
\caption{"Tattoo - Loreen" melodie op de achtergrond}
\label{fig:guitar_tattoo_loreen_main_backing_melody}
\end{figure}

Als je nu de gebruikte noten, te zien in \autoref{fig:guitar_tattoo_loreen_main_backing_melody_fretboard_major_scale} hieronder, op het fretboard vergelijkt met de majeur toonladder over meerdere octaven op het fretboard van F$\sharp$ in \autoref{fig:guitar_major_scale_octaves_over_fretboard}, zie je dat deze mooi overeenkomen.

De belangrijkste verschillen zijn dat de 'blauwe' 4 nu op de 9e fret van de D-snaar wordt gespeeld, en dat de 'groene' 3 nu op de 6e fret van de hoge E-snaar wordt gespeeld. Probeer zelf te controleren dat dit inderdaad dezelfde noten zijn.

Omdat de cijfers in de frets de toontrap van de majeur toonladder aangeven, zie je dat deze melodie alle noten uit de F$\sharp$ majeur toonladder gebruikt (over twee octaven). Let op hoe alle cijfers van 1 tot en met 8 worden gebruikt.

\begin{figure}[h]
\centering
\includegraphics[height=0.2\textheight]{../../Images/NotesUsedInMainBackingMelodyTattooLoreen.png}
\caption{Noten gebruikt in de melodie van "Tattoo - Loreen"}
\label{fig:guitar_tattoo_loreen_main_backing_melody_fretboard_major_scale}
\end{figure}

\newpage

% C major"
% - hakunamatata: https://www.hooktheory.com/theorytab/view/disney/hakuna-matata#Chorus
% - Animal crossing 5PM: https://www.hooktheory.com/theorytab/view/kazumi-totaka/5pm#Verse
% - WoW Gryphonheart: https://www.hooktheory.com/theorytab/view/jo-blankenburg/gryphonheart#Instrumental
% - Zelda Lost Woods: https://www.hooktheory.com/theorytab/view/the-legend-of-zelda/lost-woods#Instrumental
% - a7x bat country (outro walk): https://www.hooktheory.com/theorytab/view/avenged-sevenfold/bat-country#Outro
% - ET intro: https://www.hooktheory.com/theorytab/view/john-williams/et-the-extra-terrestrial----main-theme#Chorus
% poppy I disagree (chorus): https://www.hooktheory.com/theorytab/view/poppy/i-disagree#Chorus

\subsubsection{Improviseren}

Een andere manier om de majeur toonladder op gitaar te leren is door te improviseren over liedjes en/of akkoordenschema's (hierover leer je meer in \autoref{sec:chord_progression}). Een akkoordenschema is een reeks akkoorden die herhaald wordt.

Het volgende akkoordenschema is in de \textbf{toonsoort C-majeur}. De akkoordgrepen staan in \autoref{fig:c_major_chord_progression}. Je kunt de C-majeur toonladder hierover spelen om een mooie melodie te maken.


\begin{itemize}
	\item C - F - Am - G
\end{itemize}

Dit is een "\RomanNumeralCaps{1} - \RomanNumeralCaps{4} - \RomanNumeral{6} - \RomanNumeralCaps{5}" akkoordenschema (zie \autoref{sec:chord_progression} als je wilt weten wat dit betekent).

\begin{figure}[h]
\centering
\includegraphics[height=0.19\textheight]{../../Images/CMajor_chord_progression.png}
\caption{C-majeur akkoordenschema}
\label{fig:c_major_chord_progression}
\end{figure}

\autoref{fig:guitar_c_major_scales_from_5_6_string_fretboard} toont de C-majeur toonladder vanaf zowel de 5e als de 6e snaar. Let op dat de vakjes rond de frets dezelfde vorm aangeven als in \autoref{fig:guitar_major_scale_fretboard_standard} en \autoref{fig:guitar_major_scale_fretboard_standard_start_5th_string}, maar dan in een andere toonsoort. Ook laten de kleuren noten in dezelfde octaaf zien. Probeer dit zelf te controleren.

\begin{figure}[h]
	\begin{subfigure}[b]{\textwidth}
		\centering
		\includegraphics[height=0.19\textheight]{../../Images/guitar_c_major_scales_from_5_6_string_fretboard.png}
		\caption{C majeur toonladder vanaf zowel de 5e als de 6e snaar met toontrappen}
	\label{fig:guitar_c_major_scales_from_5_6_string_fretboard}
	\end{subfigure}
	
	\vspace{0.5cm}
	
	\begin{subfigure}[b]{\textwidth}
		\centering
		\includegraphics[height=0.19\textheight]{../../Images/guitar_c_major_scales_from_5_6_string_fretboard_note_names.png}
		\caption{C majeur toonladder vanaf zowel de 5e als de 6e snaar met notennamen}
		\label{fig:guitar_c_major_scales_from_5_6_string_fretboard_note_names}
	\end{subfigure}

	\caption{C majeur toonladder op het fretboard}
\end{figure}

YouTube is vol met verschillende akkoordprogressies in verschillende toonsoorten als je ernaar zoekt.

\newpage

\subsection{De mineur toonladder}

De mineur diatonische toonladder heeft de formule die wordt weergegeven in \autoref{tab:guitar_minor_scale_interval}.

\begin{table}[h]
	\centering
	\begin{NiceTabular}{*{16}{P{0.05mm}}}
		\Block{}{} & \Block{1-2}{\large{W}} & & \Block{1-2}{\large{H}} & & \Block{1-2}{\large{W}} & & \Block{1-2}{\large{W}} & & \Block{1-2}{\large{H}} & & \Block{1-2}{\large{W}} & & \Block{1-2}{\large{W}} & & \Block{}{} \\
		\Block{1-2}{1} & & \Block{1-2}{2} & & \Block{1-2}{3$\flat$} & & \Block{1-2}{4} & & \Block{1-2}{5} & & \Block{1-2}{6$\flat$} & & \Block{1-2}{7$\flat$} & & \Block{1-2}{8} & 
	\end{NiceTabular}
	\caption{Intervallen van de mineur toonladder}
	\label{tab:guitar_minor_scale_interval}
\end{table}

Om bijvoorbeeld de C-mineur toonladder te maken, begin je op de C en volg je simpelweg de formule. Een mogelijke manier om deze te spelen is te zien in \autoref{fig:guitar_c_minor_scale_first_position}.

\begin{table}[h]
	\centering
	\begin{NiceTabular}{*{16}{P{0.05mm}}}
		\Block{}{} & \Block{1-2}{\large{W}} & & \Block{1-2}{\large{H}} & & \Block{1-2}{\large{W}} & & \Block{1-2}{\large{W}} & & \Block{1-2}{\large{H}} & & \Block{1-2}{\large{W}} & & \Block{1-2}{\large{W}} & & \Block{}{} \\
		\Block{1-2}{1} & & \Block{1-2}{2} & & \Block{1-2}{3$\flat$} & & \Block{1-2}{4} & & \Block{1-2}{5} & & \Block{1-2}{6$\flat$} & & \Block{1-2}{7$\flat$} & & \Block{1-2}{8} & \\
		\Block{1-2}{C} & & \Block{1-2}{D} & & \Block{1-2}{E\flat} & & \Block{1-2}{F} & & \Block{1-2}{G} & & \Block{1-2}{A\flat} & & \Block{1-2}{B\flat} & & \Block{1-2}{C} & 
	\end{NiceTabular}
	\caption{C-mineur toonladder}
	\label{tab:guitar_c_minor_scale}
\end{table}

\begin{figure}[h]
	\centering
	\includegraphics[height=0.2\textheight]{../../Images/CMinorScaleFirstPosition.png}
	\caption{C-mineur toonladder met notennamen en intervallen}
	\label{fig:guitar_c_minor_scale_first_position}
\end{figure}

\autoref{tab:guitar_natural_note_minor_scale} toont de mineur toonladders van de natuurlijke noten.

\begin{enumerate}
\item \textbf{Elke toonladder heeft alleen unieke letters}. Daarom is de 6e noot in de D-mineur toonladder een B$\flat$ en geen A$\sharp$.
\item De 4e noot in de toonladder is het begin van de toonladder op de volgende rij. Dit komt natuurlijk omdat ze nu zo zijn weergegeven. Let op: bij de majeur toonladder was dit de vijfde noot. Dit heeft allemaal te maken met de "kwintencirkel", die wordt uitgelegd in \autoref{sec:circle_of_fifths}.
\item Elke toonladder onder een andere in deze lijst heeft één extra $\flat$ ten opzichte van de vorige. En de noten die een flat hebben in één toonladder, hebben ook een flat in de toonladders daaronder. Ook dit heeft te maken met de "kwintencirkel". Een geheugensteuntje voor de volgorde van flats: \textbf{B}innen \textbf{E}ten \textbf{A}ppeltaart \textbf{D}agelijks \textbf{G}aat \textbf{C}harles \textbf{F}ijne. Dit is het omgekeerde van het ezelsbruggetje voor de kruizen. Maar meer hierover in \autoref{sec:circle_of_fifths}.
\end{enumerate}

\begin{table}[h]
	\centering
	\begin{NiceTabular}{*{16}{P{0.05mm}}}
		\Block{}{} & \Block{1-2}{\large{W}} & & \Block{1-2}{\large{H}} & & \Block{1-2}{\large{W}} & & \Block{1-2}{\large{W}} & & \Block{1-2}{\large{H}} & & \Block{1-2}{\large{W}} & & \Block{1-2}{\large{W}} & & \Block{}{} \\
		\Block{1-2}{1} & & \Block{1-2}{2} & & \Block{1-2}{3$\flat$} & & \Block{1-2}{4} & & \Block{1-2}{5} & & \Block{1-2}{6$\flat$} & & \Block{1-2}{7$\flat$} & & \Block{1-2}{8} & \\
		\Block{1-2}{B} & & \Block{1-2}{C\sharp} & & \Block{1-2}{D} & & \Block{1-2}{E} & & \Block{1-2}{F\sharp} & & \Block{1-2}{G} & & \Block{1-2}{A} & & \Block{1-2}{B} & \\
		\Block{1-2}{E} & & \Block{1-2}{F\sharp} & & \Block{1-2}{G} & & \Block{1-2}{A} & & \Block{1-2}{B} & & \Block{1-2}{C} & & \Block{1-2}{D} & & \Block{1-2}{E} & \\
		\Block{1-2}{A} & & \Block{1-2}{B} & & \Block{1-2}{C} & & \Block{1-2}{D} & & \Block{1-2}{E} & & \Block{1-2}{F} & & \Block{1-2}{G} & & \Block{1-2}{A} & \\
		\Block{1-2}{D} & & \Block{1-2}{E} & & \Block{1-2}{F} & & \Block{1-2}{G} & & \Block{1-2}{A} & & \Block{1-2}{B\flat} & & \Block{1-2}{C} & & \Block{1-2}{D} & \\
		\Block{1-2}{G} & & \Block{1-2}{A} & & \Block{1-2}{B\flat} & & \Block{1-2}{C} & & \Block{1-2}{D} & & \Block{1-2}{E\flat} & & \Block{1-2}{F} & & \Block{1-2}{G} & \\
		\Block{1-2}{C} & & \Block{1-2}{D} & & \Block{1-2}{E\flat} & & \Block{1-2}{F} & & \Block{1-2}{G} & & \Block{1-2}{A\flat} & & \Block{1-2}{B\flat} & & \Block{1-2}{C} & \\
		\Block{1-2}{F} & & \Block{1-2}{G} & & \Block{1-2}{A\flat} & & \Block{1-2}{B\flat} & & \Block{1-2}{C} & & \Block{1-2}{D\flat} & & \Block{1-2}{E\flat} & & \Block{1-2}{F} & 
	\end{NiceTabular}
	\caption{Mineur toonladders van alle natuurlijke noten}
	\label{tab:guitar_natural_note_minor_scale}
\end{table}

\newpage

\subsubsection{Voorbeeld}

Net als bij de majeur toonladder zijn er verschillende vormen voor de mineur toonladder (\autoref{fig:guitar_minor_scale_fretboard}). De cijfers komen overeen met het toontrede in de toonladder. De toonladder die je speelt wordt bepaald door de grondtoon (de "1" noot). In dit voorbeeld speel je dus de F$\sharp$ mineur toonladder. Als je alle noten één fret opschuift, speel je de G-mineur toonladder.

De verschillende kleuren in \autoref{fig:guitar_minor_scale_fretboard_standard} geven verschillende octaven aan. Dit is de 'standaard'/compacte mineur toonladdervorm. Let op hoe de frets met "8/1" de 8 van de vorige octaaf en de 1 van de volgende octaaf aangeven.

Deze vormen uit je hoofd leren maakt het makkelijk om te improviseren over een liedje. Maar belangrijker is om te zien hoe deze vormen zich verhouden tot de intervallen van de mineur toonladder. De makkelijkste vorm hiervoor is \autoref{fig:guitar_minor_scale_fretboard_single_string}. Met deze vorm herken je eenvoudig de mineur diatonische formule (w-h-w-w-h-w-w). Alle vormen bevatten dezelfde noten, alleen gespeeld op een andere plek op het fretboard en mogelijk in een andere octaaf. De hieronder getoonde vormen zijn nog niet alle mogelijkheden.

De F$\sharp$ mineur toonladder heeft de volgende noten:

\begin{table}[h]
	\centering
	\begin{NiceTabular}{*{16}{P{0.05mm}}}
		\Block{}{} & \Block{1-2}{\large{W}} & & \Block{1-2}{\large{H}} & & \Block{1-2}{\large{W}} & & \Block{1-2}{\large{W}} & & \Block{1-2}{\large{H}} & & \Block{1-2}{\large{W}} & & \Block{1-2}{\large{W}} & & \Block{}{} \\
		\Block{1-2}{1} & & \Block{1-2}{2} & & \Block{1-2}{3$\flat$} & & \Block{1-2}{4} & & \Block{1-2}{5} & & \Block{1-2}{6$\flat$} & & \Block{1-2}{7$\flat$} & & \Block{1-2}{8} & \\
		\Block{1-2}{F$\sharp$} & & \Block{1-2}{G$\sharp$} & & \Block{1-2}{A} & & \Block{1-2}{B} & & \Block{1-2}{C$\sharp$} & & \Block{1-2}{D} & & \Block{1-2}{E} & & \Block{1-2}{F$\sharp$} & 
	\end{NiceTabular}
	\caption{F$\sharp$ mineur toonladder}
	\label{tab:guitar_f_sharp_minor_scale}
\end{table}

\autoref{fig:guitar_f_sharp_minor_scale_score_multi_octave} toont de noten die overeenkomen met \autoref{fig:guitar_minor_scale_octaves_over_fretboard}.

\begin{figure}[h]
	\centering
	\includegraphics[width=\textwidth]{../../MuseScore/Guitar/GuitarFSharpMinorMultiOctave.png}
	\caption{Meerdere octaven van de F$\sharp$ mineur toonladder}
	\label{fig:guitar_f_sharp_minor_scale_score_multi_octave}
\end{figure}

\newpage

\begin{figure}[h]
	\begin{subfigure}[b]{0.45\textwidth}
		\centering
		\includegraphics[height=0.175\textheight]{../../Images/guitar_minor_scale_standard.png}
		\caption{Mineur toonladder op het fretboard (standaard)}
		\label{fig:guitar_minor_scale_fretboard_standard}
	\end{subfigure}
	\hfill
	\begin{subfigure}[b]{0.45\textwidth}
		\centering
		\includegraphics[height=0.175\textheight]{../../Images/guitar_minor_scale_standard_start_5th_string.png}
		\caption{Mineur toonladder op het fretboard vanaf de 5e (A) snaar}
		\label{fig:guitar_minor_scale_fretboard_standard_start_5th_string}
	\end{subfigure}
	
	\vspace{0.5cm}
	
	\begin{subfigure}[b]{\textwidth}
		\centering
		\includegraphics[height=0.175\textheight]{../../Images/guitar_minor_scale_single_string.png}
		\caption{Mineur toonladder op het fretboard op één snaar}
		\label{fig:guitar_minor_scale_fretboard_single_string}
	\end{subfigure}
	
	\vspace{0.5cm}
	
	\begin{subfigure}[b]{\textwidth}
		\centering
		\includegraphics[height=0.175\textheight]{../../Images/guitar_minor_scale_octaves_over_fretboard.png}
		\caption{Mineur toonladder octaven over het hele fretboard}
		\label{fig:guitar_minor_scale_octaves_over_fretboard}
	\end{subfigure}
	
	\caption{F$\sharp$ mineur toonladder op het fretboard}
	\label{fig:guitar_minor_scale_fretboard}
\end{figure}

\clearpage

Het nummer "The Final Countdown" van "Europe" staat in F$\sharp$ mineur. Het intro is te zien in \autoref{fig:guitar_the_final_countdown_europe_intro}.

\begin{figure}[h]
	\centering
	\includegraphics[width=\textwidth]{../../MuseScore/Guitar/GuitarTheFinalCountdownEuropeIntro.png}
	\caption{The Final Countdown - Europe intro}
	\label{fig:guitar_the_final_countdown_europe_intro}
\end{figure}

Als je kijkt naar de gebruikte noten (zie \autoref{fig:guitar_the_final_countdown_europe_fretboard_minor_scale}), zie je dat deze mooi overlappen met de F$\sharp$ mineur toonladder uit \autoref{fig:guitar_minor_scale_fretboard_standard}. Net als in de vorige fretboarddiagrammen geven de cijfers in de cirkels de toontrap aan in de, in dit geval, mineur toonladder.

\begin{figure}[h]
	\centering
	\includegraphics[height=0.2\textheight]{../../Images/NotesUsedInTheFinalCountdownEuropeIntro.png}
	\caption{Noten gebruikt in het intro van "The Final Countdown - Europe"}
	\label{fig:guitar_the_final_countdown_europe_fretboard_minor_scale}
\end{figure}

\newpage

\subsubsection{Improviseren}

Net zoals we deden bij de majeur toonladder, gaan we ook improviseren over een mineur akkoordenschema (meer hierover in \autoref{sec:chord_progression}).

Het volgende akkoordenschema is in de \textbf{toonsoort A-mineur}. Je kunt de A-mineur toonladder hierover spelen om een mooie melodie te maken.

\begin{itemize}
	\item Am - Em - G - Dm
\end{itemize}

Dit is een "\RomanNumeral{1} - \RomanNumeral{5} - \RomanNumeralCaps{7$\flat$} - \RomanNumeral{4}" akkoordenschema (zie \autoref{sec:chord_progression} als je wilt weten wat dit betekent).

\begin{figure}[h]
	\centering
	\includegraphics[height=0.19\textheight]{../../Images/AMinor_chord_progression.png}
	\caption{A-mineur akkoordenschema}
	\label{fig:a_minor_chord_progression}
\end{figure}

\autoref{fig:guitar_a_minor_scales_from_5_6_string_fretboard} toont de A-mineur toonladder vanaf zowel de 5e als de 6e snaar. Let op dat de vakjes rond de frets dezelfde vorm aangeven als in \autoref{fig:guitar_minor_scale_fretboard_standard} en \autoref{fig:guitar_minor_scale_fretboard_standard_start_5th_string}, maar dan in een andere toonsoort. Ook laten de kleuren dezelfde toonhoogte van de noot zien. Probeer dit zelf te controleren.

\begin{figure}[h]
	\begin{subfigure}[b]{\textwidth}
		\centering
		\includegraphics[height=0.19\textheight]{../../Images/guitar_a_minor_scales_from_5_6_string_fretboard.png}
		\caption{A-mineur toonladder vanaf zowel de 5e als de 6e snaar met toonladdertrappen}
		\label{fig:guitar_a_minor_scales_from_5_6_string_fretboard}
	\end{subfigure}
	
	\vspace{0.5cm}
	
	\begin{subfigure}[b]{\textwidth}
		\centering
		\includegraphics[height=0.19\textheight]{../../Images/guitar_a_minor_scales_from_5_6_string_fretboard_note_names.png}
		\caption{A-mineur toonladder vanaf zowel de 5e als de 6e snaar met notennamen}
		\label{fig:guitar_a_minor_scales_from_5_6_string_fretboard_note_names}
	\end{subfigure}
	
	\caption{A-mineur toonladder op het fretboard}
\end{figure}

Op YouTube vind je veel verschillende akkoordenschema's in diverse toonsoorten als je ernaar zoekt.

\newpage

\section{De toonladderformules leren}

De belangrijkste formule om te onthouden is die van de diatonische majeur toonladder. De meeste muziektheorie is hierop gebaseerd. Daarnaast is het goed om de mineur toonladderformule te onthouden. In het begin kun je dit het beste als volgt oefenen:

\begin{itemize}
	\item Schrijf de diatonische majeur (w-w-h-w-w-w-h) en mineur (w-h-w-w-h-w-w) toonladderformules op papier, en schrijf vervolgens de toonladder van elke noot (A tot G) uit door de formule te volgen.
	\item Oefen de compacte vormen vanaf zowel de 6e als de 5e snaar over de hele hals. Denk hierbij aan de notennamen in de toonladder die je speelt.
\end{itemize}

Het kan even duren voordat je dit echt hebt geïnternaliseerd, en het kan wat saai zijn, maar op de lange termijn is het de moeite waard.

In \autoref{chap:empty_guitar_fretboard} staan lege gitaarfretbords. Je kunt deze gebruiken om als extra oefening de notennamen van een bepaalde toonladder op de gitaar in te vullen.

\section{Een metronoom gebruiken (linker- en rechterhand synchroniseren)}

Toonladders op en neer spelen is ook een goed moment om te focussen op handcoördinatie. Zet een metronoom op een comfortabele snelheid en speel de toonladders op en neer. Zorg dat je precies op elke tel van de metronoom een nieuwe noot aanslaat. Voelt het tempo goed? Verhoog dan de BPM een beetje.

% I don't particularly like modes a lot.