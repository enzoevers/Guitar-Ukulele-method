\section{De pentatonische toonladder}

In de vorige sectie, toen we het over het verminderde akkoord hadden, werd al genoemd dat de pentatonische toonladder je helpt de onrust (dissonantie) van het tritonus-interval te vermijden. Maar waarom?

De pentatonische toonladder betekent dat er 5 (penta) noten in de toonladder zitten. In principe kunnen willekeurige 5 noten een pentatonische toonladder vormen. Maar de pentatonische toonladder die wij kennen en gebruiken is de majeur pentatonische toonladder. Dit is de majeur diatonische toonladder zonder de 4e en 7e troontrede. Hierdoor verdwijnt het tritonus-interval uit de toonladder. Maar ook alle halve toonafstanden verdwijnen.

\infobox{Let op dat het woord 'verwijderd' is gebruikt. Dit doet eigenlijk geen recht aan de pentatonische toonladder. Waarschijnlijk bestond de pentatonische toonladder al vóór de majeur (diatonische) toonladder. In 2008 werd bijvoorbeeld een fluit gevonden van 30-40 duizend jaar oud die gestemd was op de pentatonische toonladder. Daarnaast bestudeerde Pythagoras hoe de tonen van de pentatonische toonladder van nature in de natuur voorkomen. \cite{PentatonicFlute}}

Wat de pentatonische toonladder ons hiermee geeft is een 'veiligere' toonladder om mee te improviseren. Door deze twee noten weg te laten, verklein je de kans dat een noot die je gebruikt tijdens het improviseren een tritonus vormt met een noot uit het akkoordenschema.

Natuurlijk mag je altijd buiten de pentatonische toonladder gaan en creatief zijn. Maar vooral in het begin is de pentatonische toonladder een veilige keuze om mee te improviseren.

\newpage

We leggen de majeur pentatonische toonladder naast de majeur (diatonische) toonladder.

\autoref{tab:guitar_major_pentatic_scale} toont de majeur pentatonische toonladder. Hier betekent \textbf{W+} 3 halve tonen (een hele toon (W) + een halve toon (H)). Let op dat de indexen nog steeds 1-7 zijn en niet 1-5. Dit is om het makkelijker te koppelen aan de diatonische toonladders die we eerder hebben geleerd.

\begin{table}[h]
	\centering
	\begin{NiceTabular}{*{12}{P{0.05mm}}}
		\Block{}{} & \Block{1-2}{\large{W}} & & \Block{1-2}{\large{W}} & & \Block{1-2}{\large{W+}} & & \Block{1-2}{\large{W}} & & \Block{1-2}{\large{W+}} & & \Block{}{} \\
		\Block{1-2}{1} & & \Block{1-2}{2} & & \Block{1-2}{3} & & \Block{1-2}{5} & & \Block{1-2}{6} & & \Block{1-2}{8} &
	\end{NiceTabular}
	\caption{Intervallen van de majeur pentatonische toonladder}
	\label{tab:guitar_major_pentatic_scale}
\end{table}

Voor de mineur pentatonische toonladder laten we om dezelfde reden de 2e en 6e toontrap weg. Dit levert de volgende intervallen op.

\begin{table}[h]
	\centering
	\begin{NiceTabular}{*{12}{P{0.05mm}}}
		\Block{}{} & \Block{1-2}{\large{W+}} & & \Block{1-2}{\large{W}} & & \Block{1-2}{\large{W}} & & \Block{1-2}{\large{W+}} & & \Block{1-2}{\large{W}} & & \Block{}{} \\
		\Block{1-2}{1} & & \Block{1-2}{3$\flat$} & & \Block{1-2}{4} & & \Block{1-2}{5} & & \Block{1-2}{7$\flat$} & & \Block{1-2}{8} & 
	\end{NiceTabular}
	\caption{Intervallen van de mineur pentatonische toonladder}
	\label{tab:guitar_minor_pentatic_scale}
\end{table}

Dat is allemaal leuk en aardig, maar hoe gebruik je dit? \autoref{fig:guitar_major_minor_pentatonic_shapes} toont de vorm van zowel de majeur als de mineur pentatonische toonladder. Dit zijn in feite dezelfde vormen als de diatonische vormen die je eerder hebt geleerd, maar met enkele noten weggelaten (4 en 7 voor majeur, 2 en 6$\flat$ voor mineur).

\begin{figure}[h]
	\begin{subfigure}[b]{0.45\textwidth}
		\centering
		\includegraphics[height=0.175\textheight]{../../Images/guitar_major_pentatonic_standard.png}
		\caption{Majeur pentatonische vorm vanaf de 6e snaar}
		\label{fig:guitar_major_pentatonic_shape}
	\end{subfigure}
	\hfill
	\begin{subfigure}[b]{0.45\textwidth}
		\centering
		\includegraphics[height=0.175\textheight]{../../Images/guitar_major_pentatonic_standard_start_5th_string.png}
		\caption{Majeur pentatonische vorm vanaf de 5e snaar}
		\label{fig:guitar_major_pentatonic_shape_start_5th_string}
	\end{subfigure}
	
	\vspace{0.5cm}
	
	\begin{subfigure}[b]{0.45\textwidth}
		\centering
		\includegraphics[height=0.175\textheight]{../../Images/guitar_minor_pentatonic_standard.png}
		\caption{Mineur pentatonische vorm vanaf de 6e snaar}
		\label{fig:guitar_minor_pentatonic_shape}
	\end{subfigure}
	\hfill
	\begin{subfigure}[b]{0.45\textwidth}
		\centering
		\includegraphics[height=0.175\textheight]{../../Images/guitar_minor_pentatonic_standard_start_5th_string.png}
		\caption{Mineur pentatonische vorm vanaf de 5e snaar}
		\label{fig:guitar_minor_pentatonic_shape_start_5th_string}
	\end{subfigure}
	\caption{}
	\label{fig:guitar_major_minor_pentatonic_shapes}
\end{figure}

\subsection{Jouw beurt}

Zoek een nummer dat je leuk vindt, zoek de toonsoort op en gebruik de pentatonische toonladder om erover te improviseren. De toonsoort kun je vaak ook op internet vinden.

Let op: sommige nummers veranderen van toonsoort tijdens het nummer. Hierdoor kan de pentatonische toonladder soms wat vreemd klinken. Als dat gebeurt, kun je de nieuwe toonsoort opzoeken en verder improviseren in de pentatonische toonladder van die toonsoort.

Wat je ook kunt proberen is op internet te zoeken naar backing tracks in een bepaalde toonsoort.