\section{Fretbord} \label{sec:fretboard_introduction}

\begin{minipage}{0.45\textwidth}
Elk vakje op het fretbord heeft een andere toon. De metalen strips op de hals heten de \textbf{frets}. Let op: dezelfde toon van een enkele noot kun je ook op andere snaren vinden (denk terug aan het stemmen van de gitaar door een andere snaar als referentie te gebruiken).

Als iemand vraagt om de 2e fret op de 3e snaar in te drukken, dan druk je je vinger op de plek van de groene stip. Direct naast de fret. Zie \autoref{fig:guitar_string_fretting}.
\end{minipage}
\hfill
\begin{minipage}{0.44\textwidth}
    \centering
    \includegraphics[width=\textwidth]{../../Images/guitar-neck-fretting.png}
	\captionof{figure}{De groene stip geeft de vingerpositie voor de 2e fret op de 3e snaar aan}
    \label{fig:guitar_string_fretting}
\end{minipage}

In muziek zijn er 12 verschillende noten voordat je weer 'terug bent'. Als je weer bij dezelfde nootletter komt, ben je een octaaf hoger. De 12 verschillende noten staan hieronder.

\begin{table}[h]
\centering
\begin{NiceTabular}{*{12}{P{5mm}}}
\large{A} & \large{A\sharp} & \large{B} & \large{C} & \large{C\sharp} & \large{D} & \large{D\sharp} & \large{E} & \large{F} & \large{F\sharp} & \large{G} & \large{G\sharp}
\end{NiceTabular}
\end{table}

Je ziet dat er maar \textbf{7} verschillende letters zijn en \textbf{5} letters met een \textbf{\sharp}. Deze $\sharp$ symbolen heten \textbf{kruizen}. Op het fretbord betekent een $\sharp$ dat je één fret omhoog gaat (naar de body van de gitaar).

In \autoref{fig:guitar_string_a_octave_single_string_sharps} zie je een \textbf{notenbalk} met daaronder \textbf{tabulatuur (TAB)}. In het volgende hoofdstuk leer je de noten lezen. Voor nu kun je proberen eerst de TAB te lezen om het te spelen.

Elke lijn in het TAB-gedeelte stelt een snaar voor, met de 6e (dikste) snaar onderaan. De cijfers geven aan welke fret je moet indrukken (een 0 betekent een open snaar). Dus de TAB in \autoref{fig:guitar_string_a_octave_single_string_sharps} zegt dat je eerst de open A-snaar speelt, en dan elke fret omhoog tot de 12e fret.

\begin{figure}[h]
    \centering
    \includegraphics[width=\textwidth]{../../MuseScore/Guitar/PitchesSharpsSingleString.png}
	\caption{Een octaaf van A tot A op de 5e A-snaar met kruizen}
    \label{fig:guitar_string_a_octave_single_string_sharps}
\end{figure}

\newpage

Eerder werd genoemd dat dezelfde toon op meerdere snaren te vinden is. Dit wordt geïllustreerd in \autoref{fig:guitar_string_a_octave_multi_string_sharps}. Dit zijn de oplopende noten/toonhoogtes zoals in \autoref{fig:guitar_string_a_octave_single_string_sharps}, maar gespeeld op een andere snaar.

Dit geeft ook een groot verschil aan tussen tabs en noten. Met noten wordt het verwachte geluid (de toonhoogte) beschreven. Je bent vrij om te bepalen waar je dit op het fretbord speelt. Tabs laten één mogelijke positie zien om de noten te spelen. Je mag natuurlijk altijd de positie veranderen zolang de resulterende toonhoogte hetzelfde is. Maar hiervoor moet je weten waar elke noot op het fretbord zit.

\begin{figure}[h]
	\centering
	\includegraphics[width=\textwidth]{../../MuseScore/Guitar/PitchesSharpsMultiString.png}
	\caption{Een octaaf van A tot A op meerdere snaren met kruizen}
	\label{fig:guitar_string_a_octave_multi_string_sharps}
\end{figure}


Naast kruizen zijn er ook mollen. Een mol (\flat) betekent dat je een halve toon (één fret) omlaag gaat. Als je \autoref{fig:guitar_string_a_octave_multi_string_sharps} herschrijft met mollen, krijg je \autoref{fig:guitar_string_a_octave_multi_string_flats}.


In \autoref{fig:guitar_string_a_octave_multi_string_flats} zie je ook een nieuw symbool: het herstellingsteken (\natural). Dit betekent dat de noot waarop een $\flat$ of $\sharp$ stond, weer 'normaal' is. Wanneer een $\flat$ of $\sharp$ aan een noot wordt toegevoegd, blijft dit geldig voor die noot tot het einde van de maat. Wat een 'maat' is, wordt later uitgelegd.

\begin{figure}[h]
	\centering
	\includegraphics[width=\textwidth]{../../MuseScore/Guitar/PitchesFlatsMultiString.png}
	\caption{Een octaaf van A tot A op meerdere snaren met mollen en herstellingstekens}
	\label{fig:guitar_string_a_octave_multi_string_flats}
\end{figure}