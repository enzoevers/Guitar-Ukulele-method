\section{Vingernamen}

Bij het gitaarspelen krijgen je vingers een naam. Dit maakt het makkelijker om in muzieknotatie aan te geven welke vinger je moet gebruiken. De namen staan in \autoref{fig:finger_names}.

\begin{figure}[h]
    \centering
    \includegraphics[width=0.7\textwidth]{../../Images/guitar-finger-tips_pima.jpg}
  \caption{Namen van de vingers \cite{FingerNames}}
    \label{fig:finger_names}
\end{figure}

\autoref{fig:left_hand_position} laat zien hoe je je linkerhand positioneert. Het belangrijkste is dat je vingertoppen bijna loodrecht op de snaar staan. Zo krijg je een zuiver geluid en raak je niet per ongeluk andere snaren aan of demp je een andere snaar.

Soms is er echter een reeks noten waarbij je één vinger over twee of meer snaren moet leggen. Maar in het algemeen geldt: houd je vingers loodrecht op de snaren.

\begin{figure}[h]
	\begin{subfigure}[b]{0.45\textwidth}
		\includegraphics[width=\textwidth]{../../Images/Brad-left-finger-position-front-2016.jpg}
		\caption{}
		\label{fig:}
	\end{subfigure}
	\hfill
	\begin{subfigure}[b]{0.45\textwidth}
		\includegraphics[width=\textwidth]{../../Images/Bradford-Left-hand-player-2016.jpg}
		\caption{}
		\label{fig:}
	\end{subfigure}
  \caption{Linkerhandpositie \cite{LeftHandPositionBradlyWerner}}
	\label{fig:left_hand_position}
\end{figure}

\newpage

\section{open en vallende slag}

Bij een open aanslag houd je je rechterhand ontspannen boven de snaren (zie \autoref{fig:free_stoke_hand_position}). Om een snaar te spelen beweeg je je vinger door de snaar zonder het bovenste deel van je vinger op te tillen. Je vinger krult lichtjes in je hand. Nadat je het geluid hebt gemaakt, beweeg je je vinger terug naar de ontspannen positie.

Het is nu de kunst om de andere snaren niet te raken en de snaar niet te plukken/trekken.

\begin{figure}[h]
  \begin{subfigure}[b]{0.45\textwidth}
    \includegraphics[width=\textwidth]{../../Images/Bradford-right-hand-close-2016.jpg}
    \caption{}
    \label{fig:}
  \end{subfigure}
  \hfill
  \begin{subfigure}[b]{0.45\textwidth}
    \includegraphics[width=\textwidth]{../../Images/brad-right-stroke-2016.jpg}
    \caption{}
    \label{fig:}
  \end{subfigure}
  \caption{Open slag positie \cite{FreeStrokePositionBradlyWerner}}
  \label{fig:free_stoke_hand_position}
\end{figure}

Een vallende slag klinkt vaak wat harder (maar met wat oefening kan een vrije slag net zo hard klinken). Zoals de naam al zegt, betekent een vallende slag dat je je vinger door een snaar beweegt om deze te spelen, maar nu laat je je vinger rusten op de volgende snaar.

\section{Spelen met een plectrum}

Naast je vingers kun je ook gitaar spelen met een plectrum (ook wel pick genoemd).

Er is niet één juiste manier om een plectrum vast te houden, maar \autoref{fig:how_to_hold_a_pick} is een goed uitgangspunt. Het belangrijkste is dat je de plectrum niet te los en niet te strak vasthoudt. Probeer ook je duim niet naar binnen te buigen, maar houd hem recht of een beetje naar buiten gebogen.

Om soepel door de snaar te spelen: leg de plectrum niet plat op de snaar, maar houd hem onder een kleine hoek zodat de snaar een beetje over de rand van de plectrum glijdt. Een kleine hoek is al voldoende.

\begin{figure}[h]
	\begin{subfigure}[b]{0.45\textwidth}
		\includegraphics[width=\textwidth]{../../Images/JustinGuitarPickhold1.jpg}
		\caption{}
		\label{fig:}
	\end{subfigure}
	\hfill
	\begin{subfigure}[b]{0.45\textwidth}
		\includegraphics[width=\textwidth]{../../Images/JustinGuitarPickhold2.jpg}
		\caption{}
		\label{fig:}
	\end{subfigure}
  \caption{Hoe houd je een plectrum vast \cite{JustinGuitarPickHolding}}
	\label{fig:how_to_hold_a_pick}
\end{figure}

\newpage

\section{\textnormal{O}efeningen}

In de onderstaande oefeningen zie je enkele symbolen boven de noten. De cijfers met een cirkel erom geven aan op welke snaar de noot gespeeld moet worden. De \textit{i} en \textit{m} geven aan welke rechterhandvinger je moet gebruiken om de noot te spelen.

Probeer \autoref{fig:exercise_rest_free_stroke} te spelen met een open slag, vallende slag en met een plectrum om de verschillen te horen. Als je met een plectrum speelt, kun je de \textit{i} en \textit{m} aanduidingen negeren.

Als je met een plectrum speelt, probeer dan één keer alleen met downstrokes te spelen en één keer met afwisselende slagen. Downstrokes betekenen dat je de noten speelt door de plectrum op een snaar te plaatsen, door te duwen en de plectrum weer op de snaar te plaatsen. Afwisselend spelen begint hetzelfde als een enkele downstroke, maar in plaats van weer bovenop de snaar te gaan na de eerste noot, speel je de noot door de plectrum omhoog door de snaar te duwen. Daarna begin je weer met een downstroke en blijf je afwisselen.

\begin{figure}[h]
    \centering
    \includegraphics[width=\textwidth]{../../MuseScore/Guitar/OpenEnVallendeAanslag.png}
  \caption{\textnormal{O}efening: rust- en vrije slag}
    \label{fig:exercise_rest_free_stroke}
\end{figure}


Deze tweede oefening (\autoref{fig:exercise_i_m_string_change}) lijkt op \autoref{fig:exercise_rest_free_stroke}, maar is iets uitdagender.

\begin{figure}[h]
    \centering
    \includegraphics[width=\textwidth]{../../MuseScore/Guitar/TwoStringAlternating.png}
  \caption{\textnormal{O}efening: van snaar wisselen met \textit{i}- en \textit{m}-vingers}
    \label{fig:exercise_i_m_string_change}
\end{figure}


Om alle PIMA-vingers te gebruiken, kun je proberen de intro van \textit{Nothing Else Matters} van \textit{Metallica} te spelen (\autoref{fig:exercise_nothing_else_matters_metallica_intro_pima}). Je kunt dit ook proberen met een plectrum.

\begin{figure}[h]
    \centering
    \includegraphics[width=\textwidth]{../../MuseScore/Guitar/NothinElseMatters_Metallica_Intro.png}
  \caption{\textnormal{O}efening: PIMA met Nothing Else Matters - Metallica intro}
    \label{fig:exercise_nothing_else_matters_metallica_intro_pima}
\end{figure}

\newpage


In \autoref{fig:guitar_finger_exercise_perfect_ed_sheeran} gebruik je ook je linkerhand. De cijfers boven de noten geven aan welke linkerhandvinger je moet gebruiken om de fret in te drukken. Speel deze oefening met afwisselend \textit{i}- en \textit{m}-vingers. Ook hier kun je een plectrum gebruiken.

Focus nu op de tabs en negeer de andere symbolen.

\begin{figure}[h]
    \centering
    \includegraphics[width=\textwidth]{../../MuseScore/Guitar/GuitarPerfectEdSheeranSingleNotesFirstVerse.png}
  \caption{Vingeroefening met Perfect - Ed Sheeran}
    \label{fig:guitar_finger_exercise_perfect_ed_sheeran}
\end{figure}