\chapter{Vinger oefeningen} \label{chap:finger_exercises}

\section{Fret hand}

\subsection{Speel nummers en toonladders langzaam}

Door gewoon een nummer langzaam en met een metronoom te oefenen, kun je je concentreren op een goede vingerplaatsing. Focus op een helder geluid van de noot.

\subsection{Spider oefening}

Er zijn veel variaties van de spider oefening. Maar \autoref{fig:guitar_spider_exercise} richt zich op een rechte vingerplaatsing. Als je vingers niet goed op de snaren staan, demp je de lagere snaar, wat resulteert in onduidelijke noten wanneer je noten op die snaar speelt.

\begin{figure}[h]
	\centering
	\includegraphics[width=\textwidth]{../../MuseScore/Guitar/guitar_basic_spider_exercise.png}
	\caption{Spider oefening}
	\label{fig:guitar_spider_exercise}
\end{figure}

Je kunt natuurlijk op een andere fret beginnen dan de 5e. En je zou ook bijvoorbeeld elke andere snaar kunnen overslaan. \autoref{fig:guitar_spider_exercise} is gewoon de basis.

\section{Strumming \& picking hand}
TODO