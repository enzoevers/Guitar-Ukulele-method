\section{Modi (kerktoonsoort)} \label{sec:modes}

\subsection{diatonische modi}

Tot nu toe hebben we de majeur- en mineurtoonladders gezien. Zowel de diatonische als de pentatonische versies. En allemaal beginnend vanaf de 6e of 5e snaar. Hiermee kun je een mooie melodie improviseren over een nummer.

Je hebt ook geleerd over de relatieve mineur- en majeurtoonladders bij de kwintencirkel. In \autoref{tab:guitar_mode_intervals} zie je dat als je begint op de 6e graad van de majeurtoonladder (Ionische modus), je de intervallen van de natuurlijke mineurtoonladder krijgt (ook wel \textnormal{A}eolische modus genoemd), terwijl je nog steeds dezelfde noten hebt.

Betekent dit dat je kunt improviseren met de noten van de A-mineur toonladder over een nummer in C-majeur? JA! Het zijn dezelfde noten.

Betekent dit dat C-majeur en A-mineur hetzelfde zijn? \textbf{NEE!} Dat betekent ook dat je alleen kunt zeggen dat een nummer in C-majeur, A-mineur, E-frygisch, enz. staat (die dezelfde noten hebben) door te kijken naar het \mbox{\textbf{tonale centrum}} (de toon die het meest prominent is) van de muzikale context. Let op dat de intervallen vanaf de grondtoon van elke modus verschillend zijn.

\underline{Ezelsbruggetje voor de volgorde van de modale namen: \textbf{I} \textbf{D} \textbf{P} \textbf{L} \textbf{M} \textbf{A} \textbf{L}.} % TODO

\begin{table}[h]
	\centering
	\begin{NiceTabular}{*{30}{c}}
		\Block{}{Ionische (majeur)} & \Block{}{} & \Block{1-2}{\large{W}} & & \Block{1-2}{\large{W}} & & \Block{1-2}{\large{H}} & & \Block{1-2}{\large{W}} & & \Block{1-2}{\large{W}} & & \Block{1-2}{\large{W}} & & \Block{1-2}{\large{H}} & & \Block{}{} \\
		\Block{}{\textit{Akkoorden}} & \Block{1-2}{\RomanNumeralCaps{1}} & & \Block{1-2}{\RomanNumeral{2}} & & \Block{1-2}{\RomanNumeral{3}} & & \Block{1-2}{\RomanNumeralCaps{4}} & & \Block{1-2}{\RomanNumeralCaps{5}} & & \Block{1-2}{\RomanNumeral{6}} & & \Block{1-2}{\RomanNumeral{7}\textsuperscript{o}} & & \\
		\Block{}{\textit{Intervallen}} & \Block{1-2}{1} & & \Block{1-2}{2} & & \Block{1-2}{3} & & \Block{1-2}{4} & & \Block{1-2}{5} & & \Block{1-2}{6} & & \Block{1-2}{7} & & \\
		\hline \\
		\Block{}{Dorisch} & \Block{1-3}{} & & & \Block{1-2}{\large{W}} & & \Block{1-2}{\large{H}} & & \Block{1-2}{\large{W}} & & \Block{1-2}{\large{W}} & & \Block{1-2}{\large{W}} & & \Block{1-2}{\large{H}} & & \Block{1-2}{\large{W}} & & \\
		\Block{}{} & \Block{1-2}{} & & \Block{1-2}{\RomanNumeral{1}} & & \Block{1-2}{\RomanNumeral{2}} & & \Block{1-2}{\RomanNumeralCaps{3}} & & \Block{1-2}{\RomanNumeralCaps{4}} & & \Block{1-2}{\RomanNumeral{5}} & & \Block{1-2}{\RomanNumeral{6}\textsuperscript{o}} & & \Block{1-2}{\RomanNumeralCaps{7}} & & \\
		\Block{}{} & \Block{1-2}{} & & \Block{1-2}{1} & & \Block{1-2}{2} & & \Block{1-2}{3\flat} & & \Block{1-2}{4} & & \Block{1-2}{5} & & \Block{1-2}{6} & & \Block{1-2}{7\flat} & & \\
		\hline \\
		\Block{}{Frygisch} & \Block{1-5}{} & & & & & \Block{1-2}{\large{H}} & & \Block{1-2}{\large{W}} & & \Block{1-2}{\large{W}} & & \Block{1-2}{\large{W}} & & \Block{1-2}{\large{H}} & & \Block{1-2}{\large{W}} & & \Block{1-2}{\large{W}} & & \\
		\Block{}{} & \Block{1-4}{} & & & & \Block{1-2}{\RomanNumeral{1}} & & \Block{1-2}{\RomanNumeralCaps{2}} & & \Block{1-2}{\RomanNumeralCaps{3}} & & \Block{1-2}{\RomanNumeral{4}} & & \Block{1-2}{\RomanNumeral{5}\textsuperscript{o}} & & \Block{1-2}{\RomanNumeralCaps{6}} & & \Block{1-2}{\RomanNumeral{7}} & & \\
		\Block{}{} & \Block{1-4}{} & & & & \Block{1-2}{1} & & \Block{1-2}{2\flat} & & \Block{1-2}{3\flat} & & \Block{1-2}{4} & & \Block{1-2}{5} & & \Block{1-2}{6\flat} & & \Block{1-2}{7\flat} & & \\
		\hline \\
		\Block{}{Lydisch} & \Block{1-7}{} & & & & & & & \Block{1-2}{\large{W}} & & \Block{1-2}{\large{W}} & & \Block{1-2}{\large{W}} & & \Block{1-2}{\large{H}} & & \Block{1-2}{\large{W}} & & \Block{1-2}{\large{W}} & & \Block{1-2}{\large{H}} & & \\
		\Block{}{} & \Block{1-6}{} & & & & & & \Block{1-2}{\RomanNumeralCaps{1}} & & \Block{1-2}{\RomanNumeralCaps{2}} & & \Block{1-2}{\RomanNumeral{3}} & & \Block{1-2}{\RomanNumeral{4}\textsuperscript{o}} & & \Block{1-2}{\RomanNumeralCaps{5}} & & \Block{1-2}{\RomanNumeral{6}} & & \Block{1-2}{\RomanNumeral{7}} & & \\
		\Block{}{} & \Block{1-6}{} & & & & & & \Block{1-2}{1} & & \Block{1-2}{2} & & \Block{1-2}{3} & & \Block{1-2}{4\sharp} & & \Block{1-2}{5} & & \Block{1-2}{6} & & \Block{1-2}{7} & & \\
		\hline \\
		\Block{}{Mixolydisch} & \Block{1-9}{} & & & & & & & & & \Block{1-2}{\large{W}} & & \Block{1-2}{\large{W}} & & \Block{1-2}{\large{H}} & & \Block{1-2}{\large{W}} & & \Block{1-2}{\large{W}} & & \Block{1-2}{\large{H}} & & \Block{1-2}{\large{W}} & & \\
		\Block{}{} & \Block{1-8}{} & & & & & & & & \Block{1-2}{\RomanNumeralCaps{1}} & & \Block{1-2}{\RomanNumeral{2}} & & \Block{1-2}{\RomanNumeral{3}\textsuperscript{o}} & & \Block{1-2}{\RomanNumeralCaps{4}} & & \Block{1-2}{\RomanNumeral{5}} & & \Block{1-2}{\RomanNumeral{6}} & & \Block{1-2}{\RomanNumeralCaps{7}} & & \\
		\Block{}{} & \Block{1-8}{} & & & & & & & &  \Block{1-2}{1} & & \Block{1-2}{2} & & \Block{1-2}{3} & & \Block{1-2}{4} & & \Block{1-2}{5} & & \Block{1-2}{6} & & \Block{1-2}{7\flat} & & \\
		\hline \\
		\Block{}{\textnormal{A}eolisch (natuurlijk mineur)} & \Block{1-11}{} & & & & & & & & & & & \Block{1-2}{\large{W}} & & \Block{1-2}{\large{H}} & & \Block{1-2}{\large{W}} & & \Block{1-2}{\large{W}} & & \Block{1-2}{\large{H}} & & \Block{1-2}{\large{W}} & & \Block{1-2}{\large{W}} & & \\
		\Block{}{} & \Block{1-10}{} & & & & & & & & & & \Block{1-2}{\RomanNumeral{1}} & & \Block{1-2}{\RomanNumeral{2}\textsuperscript{o}} & & \Block{1-2}{\RomanNumeralCaps{3}} & & \Block{1-2}{\RomanNumeral{4}} & & \Block{1-2}{\RomanNumeral{5}} & & \Block{1-2}{\RomanNumeralCaps{6}} & & \Block{1-2}{\RomanNumeralCaps{7}} & & \\
		\Block{}{} & \Block{1-10}{} & & & & & & & & & &  \Block{1-2}{1} & & \Block{1-2}{2} & & \Block{1-2}{3\flat} & & \Block{1-2}{4} & & \Block{1-2}{5} & & \Block{1-2}{6\flat} & & \Block{1-2}{7\flat} & & \\
		\hline \\
		\Block{}{Locrisch} & \Block{1-13}{} & & & & & & & & & & & & & \Block{1-2}{\large{H}} & & \Block{1-2}{\large{W}} & & \Block{1-2}{\large{W}} & & \Block{1-2}{\large{H}} & & \Block{1-2}{\large{W}} & & \Block{1-2}{\large{W}} & & \Block{1-2}{\large{W}} & & \\
		\Block{}{} & \Block{1-12}{} & & & & & & & & & & & & \Block{1-2}{\RomanNumeral{1}\textsuperscript{o}} & & \Block{1-2}{\RomanNumeralCaps{2}} & & \Block{1-2}{\RomanNumeral{3}} & & \Block{1-2}{\RomanNumeral{4}} & & \Block{1-2}{\RomanNumeralCaps{5}} & & \Block{1-2}{\RomanNumeral{6}} & & \Block{1-2}{\RomanNumeral{7}} & & \\
		\Block{}{} & \Block{1-12}{} & & & & & & & & & & & &  \Block{1-2}{1} & & \Block{1-2}{2\flat} & & \Block{1-2}{3\flat} & & \Block{1-2}{4} & & \Block{1-2}{5\flat} & & \Block{1-2}{6\flat} & & \Block{1-2}{7\flat} & & \\
	\end{NiceTabular}
	\caption{Modi intervallen}
	\label{tab:guitar_mode_intervals}
\end{table}

\newpage

\subsubsection{Modi gebruiken om een ander gevoel te geven}

Tot nu toe hebben we naar modi gekeken als varianten van de majeurtoonladder. Dit betekent dat als een nummer in C-majeur staat, je noten kunt spelen uit alle verschillende modi die van C-majeur zijn afgeleid (D-dorisch, E-frygisch, etc.). Let op: dit gaat ervan uit dat het tonale centrum, de noot waar het nummer steeds naar terug wil, C is. Door noten en akkoorden uit de C-majeur toonladder te spelen, krijgt het de majeurklank.

Om het gevoel van een andere modus te krijgen, wil je meestal beginnen op het akkoord van jouw modus (dus het E-mineur akkoord voor E-frygisch), en dan akkoorden spelen die de karakteristieke noten van de modus benadrukken.

Wat zijn de karakteristieke noten van een modus? Daarvoor moeten we eerst bepalen of een modus majeur of mineur is. Dit doe je door naar de 3e graad te kijken: is het een grote terts (3) of een kleine terts (3$\flat$)? Vervolgens vergelijk je de intervallen van de modus met die van de majeur- of mineurtoonladder en kijk je welke verschillen. Dit is te zien in \autoref{tab:guitar_mode_characteristic_notes}.

\begin{table}[h]
	\centering
	\begin{NiceTabular}{*{8}{c}}
		\textbf{Ionisch (majeur)} & 1 & 2 & 3 & 4 & 5 & 6 & 7 \\
		Lydisch (\textit{majeur}) & 1 & 2 & 3 & \Block[fill=ColorOtherNote]{}{4$\sharp$} & 5 & 6 & 7 \\
		Mixolydisch (\textit{majeur}) & 1 & 2 & 3 & 4 & 5 & 6 & \Block[fill=ColorOtherNote]{}{7$\flat$} \\
		\\
		\textbf{\textnormal{A}eolisch (natuurlijke mineur)} & 1 & 2 & 3$\flat$ & 4 & 5 & 6$\flat$ & 7$\flat$ \\
		Dorisch (\textit{mineur}) & 1 & 2 & 3$\flat$ & 4 & 5 & \Block[fill=ColorOtherNote]{}{6} & 7$\flat$ \\
		Frygisch (\textit{mineur}) & 1 & \Block[fill=ColorOtherNote]{}{2$\flat$} & 3$\flat$ & 4 & 5 & 6$\flat$ & 7$\flat$ \\
		Locrisch (\textit{verlaagd}) & 1 & \Block[fill=ColorOtherNote]{}{2$\flat$} & 3$\flat$ & 4 & \Block[fill=ColorOtherNote]{}{5$\flat$} & 6$\flat$ & 7$\flat$ \\
	\end{NiceTabular}
	\caption{Modi karakteriserende noten}
	\label{tab:guitar_mode_characteristic_notes}
\end{table}

De volgende stap is om te kijken welke akkoorden in de modus die je wilt spelen de karakteristieke noot bevatten. Dit is belangrijk omdat we juist die karakteristieke noot willen benadrukken om het juiste 'gevoel' te geven. In \autoref{sec:building_chords_with_diatonic_scale} heb je geleerd hoe je akkoorden bouwt uit de diatonische toonladders. De akkoorden in \autoref{tab:guitar_mode_intervals} zijn op dezelfde manier gemaakt. Met deze kennis kan tabel \autoref{tab:guitar_mode_characteristic_chords} worden gemaakt, die laat zien welke akkoorden van de modus karakteristiek zijn voor die modus. Deze tabel is beperkt tot de triades en septiemakkoorden. Maar je kunt natuurlijk een akkoord aanpassen, bijvoorbeeld naar een sus2-akkoord, zodat de karakteristieke noot in het akkoord komt.

\begin{table}[h]
	\centering
	\begin{NiceTabular}{*{5}{c}}
		\textbf{Ionisch (majeur)} &  \\
		Lydisch (\textit{majeur}) & \RomanNumeralCaps{2} & \RomanNumeral{4}\textsuperscript{o} & \RomanNumeralCaps{5}\textsuperscript{M7} & \RomanNumeral{7}  \\
		Mixolydisch (\textit{majeur}) & \RomanNumeralCaps{1}\textsuperscript{7} & \RomanNumeral{3}\textsuperscript{o} & \RomanNumeral{5} & \RomanNumeralCaps{7} \\
		\\
		\textbf{\textnormal{A}eolisch (natuurlijke mineur)} & \\
		Dorisch (\textit{mineur}) &  \RomanNumeral{2} & \RomanNumeralCaps{4} & \RomanNumeral{6}\textsuperscript{o} & \RomanNumeralCaps{7}\textsuperscript{M7} \\
		Frygisch (\textit{mineur}) &  \RomanNumeralCaps{2} & \RomanNumeralCaps{3}\textsuperscript{7} & \RomanNumeral{4} & \RomanNumeralCaps{6} \\
		Locrisch (\textit{verlaagd}) & all \\
	\end{NiceTabular}
	\caption{Modi karakteriserende akkoorden}
	\label{tab:guitar_mode_characteristic_chords}
\end{table}

Als je nu een akkoordprogressie maakt voor een modus, is het goed om ten minste één akkoord uit \autoref{tab:guitar_mode_characteristic_chords} te gebruiken, naast het eerste akkoord van de modus.

\subsubsection{The modi toonladders spelen}

We hebben al vormen gezien voor de majeur- en mineurtoonladders. \autoref{fig:guitar_mode_scale_shapes} toont de vorm voor alle modi. De cijfers geven het interval van de noot aan.

\newpage

\begin{figure}[h]
	\begin{subfigure}[b]{0.31\textwidth}
		\centering
		\includegraphics[height=0.175\textheight]{../../Images/IonianScaleShape.png}
		\caption{Ionisch}
		\label{fig:guitar_mode_shape_ionian}
	\end{subfigure}
	\hfill
	\begin{subfigure}[b]{0.31\textwidth}
		\centering
		\includegraphics[height=0.175\textheight]{../../Images/DorianScaleShape.png}
		\caption{Dorisch}
		\label{fig:guitar_mode_shape_dorian}
	\end{subfigure}
	\hfill
	\begin{subfigure}[b]{0.31\textwidth}
		\centering
		\includegraphics[height=0.175\textheight]{../../Images/PhrygianScaleShape.png}
		\caption{Frygisch}
		\label{fig:guitar_mode_shape_phrygian}
	\end{subfigure}
	
	\vspace{0.5cm}
	
	\begin{subfigure}[b]{0.31\textwidth}
		\centering
		\includegraphics[height=0.175\textheight]{../../Images/LydianScaleShape.png}
		\caption{Lydisch}
		\label{fig:guitar_mode_shape_lydian}
	\end{subfigure}
	\hfill
	\begin{subfigure}[b]{0.31\textwidth}
		\centering
		\includegraphics[height=0.175\textheight]{../../Images/MixolydianScaleShape.png}
		\caption{Mixolydisch}
		\label{fig:guitar_mode_shape_mixolydian}
	\end{subfigure}
	\hfill
	\begin{subfigure}[b]{0.31\textwidth}
		\centering
		\includegraphics[height=0.175\textheight]{../../Images/AeolianScaleShape.png}
		\caption{\textnormal{A}eolisch}
		\label{fig:guitar_mode_shape_aeolian}
	\end{subfigure}
	
	\vspace{0.5cm}
	
	\begin{subfigure}[b]{0.31\textwidth}
		\centering
		\includegraphics[height=0.175\textheight]{../../Images/LocrianScaleShape.png}
		\caption{Locrisch}
		\label{fig:guitar_mode_shape_locrian}
	\end{subfigure}
	
	\caption{F$\sharp$ toonladder in alle verschillende modi}
	\label{fig:guitar_mode_scale_shapes}
\end{figure}

\newpage

TODO: Geef voorbeelden van nummers in verschillende modi

\newpage

\subsubsection{Gebruik van modi-vormen om het fretbord meer te gebruiken}

Modus-vorm is een lastig woord dat verkeerd begrepen kan worden. Vormen zijn gewoon vormen. De muzikale context bepaalt de modus.

\autoref{fig:guitar_diatonic_modes_on_guitar} toont de verschillende modus-vormen. In dit geval beginnen we met de F$\sharp$ Ionische (majeur) toonladder. Maar onthoud dat deze omhoog of omlaag kunnen worden verschoven om in een andere toonsoort te spelen.

De grijze frets met cijfers op de 6e snaar tonen de F$\sharp$ majeur toonladder op de 6e snaar. Dit wordt gebruikt om de verbinding te laten zien tussen de startpositie van de verschillende modi en de majeurtoonladder-intervallen. Omdat we vanuit het perspectief van de majeurtoonladder kijken, komen de cijfers in de frets van een vorm ook overeen met de intervallen in de majeurtoonladder.

Net als eerder geven de verschillende kleuren verschillende octaven in de vorm aan. Dit betekent dat elke vorm 2 en een beetje octaven laat zien.

\infobox
{
	\begin{itemize}
		\item De modale namen in \autoref{fig:guitar_diatonic_modes_on_guitar} komen overeen met de vorm in dit geval.
		\item De relatieve positie van deze vormen dekt eenvoudig het hele fretbord met noten van een bepaalde toonladder/modus.
		\item Alle vormen die worden weergegeven in \autoref{fig:guitar_diatonic_modes_on_guitar} zijn relatief ten opzichte van de F$\sharp$ majeur toonladder (ionische modus). Als ze relatief zouden zijn ten opzichte van de G$\sharp$ dorische modus, zouden de vormen hetzelfde zijn, maar de nummers (intervallen) zouden anders zijn.
	\end{itemize}
}

\newpage

\begin{figure}[h]
	\centering
	\includegraphics[width=0.9\textwidth]{../../Images/guitar_mode_all.png}
	\caption{Diatonische modi op de gitaar}
	\label{fig:guitar_diatonic_modes_on_guitar}
\end{figure}

\clearpage

\subsection{Pentatonische modi}
TODO

\newpage

\section{CAGED}

Het CAGED-systeem heet zo vanwege de akkoordvormen die het gebruikt (\autoref{fig:guitar_caged_open_chords}). Tot nu toe zijn dit gewoon akkoorden die in een bepaalde volgorde worden weergegeven, zodat ze CAGED spellen.

\begin{figure}[h]
	\centering
	\includegraphics[width=0.95\textwidth]{../../Images/CagedOpenChords.png}
	\caption{Open akkoordvormen die worden gebruikt in het CAGED-systeem}
	\label{fig:guitar_caged_open_chords}
\end{figure}

Wat deze volgorde zo handig maakt, is hoe de (meest rechtse) grondtoon van een akkoord op dezelfde snaar staat als de (meest linkse) grondtoon van het volgende akkoord. Zie de groene pijlen in \autoref{fig:guitar_caged_open_chord_shaped_root_connnection}.

\begin{figure}[h]
	\centering
	\includegraphics[width=0.95\textwidth]{../../Images/CagedOpenChordShapesRootConnection.png}
	\caption{Connectie van de grondtonen in de CAGED akkoordvormen}
	\label{fig:guitar_caged_open_chord_shaped_root_connnection}
\end{figure}

Het CAGED-systeem kan worden gebruikt om elk akkoord met elke vorm te spelen. De F- en B-barre-akkoorden worden op deze manier gemaakt. Zie \autoref{fig:guitar_caged_e_shape_to_f} en \autoref{fig:guitar_caged_a_shape_to_b}.

\begin{figure}[h]
	\centering
	\begin{subfigure}{0.37\textwidth}
		\includegraphics[width=\textwidth]{../../Images/CagedEShapeToFChord.png}
		\caption{Het maken van het F-barre-akkoord met de E-vorm}
		\label{fig:guitar_caged_e_shape_to_f}
	\end{subfigure}
	\hfill
	\begin{subfigure}{0.37\textwidth}
		\includegraphics[width=\textwidth]{../../Images/CagedAShapeToBChord.png}
		\caption{Het maken van het B-barre-akkoord met de A-vorm}
		\label{fig:guitar_caged_a_shape_to_b}
	\end{subfigure}
	\label{fig:guitar_caged_create_main_f_b_barre_chords}
	\caption{Creëer de F- en B-akkoorden met het CAGED-systeem}
\end{figure}

\newpage

Hetzelfde kan worden gedaan met de mineurvormen.

\begin{figure}[h]
	\centering
	\includegraphics[width=0.95\textwidth]{../../Images/CagedOpenMinorChords.png}
	\caption{Open mineurakkoordvormen die worden gebruikt in het CAGED-systeem}
	\label{fig:guitar_caged_open_minor_chords}
\end{figure}

\begin{figure}[h]
	\centering
	\includegraphics[width=0.95\textwidth]{../../Images/CagedOpenMinorChordShapesRootConnection.png}
	\caption{Connectie van de grondtonen in de CAGED mineurakkoordvormen}
	\label{fig:guitar_caged_open_minor_chord_shaped_root_connnection}
\end{figure}

\begin{figure}[h]
	\centering
	\begin{subfigure}{0.37\textwidth}
		\includegraphics[width=\textwidth]{../../Images/CagedEmShapeToFmChord.png}
		\caption{Het maken van het Fm-barre-akkoord met de Em-vorm}
		\label{fig:guitar_caged_em_shape_to_fm}
	\end{subfigure}
	\hfill
	\begin{subfigure}{0.37\textwidth}
		\includegraphics[width=\textwidth]{../../Images/CagedAmShapeToBmChord.png}
		\caption{Het maken van het Bm-barre-akkoord met de Am-vorm}
		\label{fig:guitar_caged_am_shape_to_bm}
	\end{subfigure}
	\label{fig:guitar_caged_create_main_fm_bm_barre_chords}
	\caption{Creëer de Fm- en Bm-akkoorden met het CAGED-systeem}
\end{figure}

Je kunt je voorstellen dat hetzelfde geldt voor open verminderde/verhoogde, sus-, septiemakkoorden, etc. Het belangrijkste van het CAGED-systeem is dat je de grondtonen in akkoordvormen leert, en deze gebruikt als ankerpunten tijdens het spelen.

\newpage

\autoref{fig:guitar_caged_c_chord_over_fretboard} demonstreert hoe de verschillende akkoordvormen je in staat stellen om hetzelfde akkoord over de hele hals te spelen.

\begin{figure}[h]
	\centering
	\includegraphics[width=\textwidth]{../../Images/CagedCChordOverFretboard.png}
	\caption{Het spelen van het C-akkoord met alle CAGED-vormen over de hals}
	\label{fig:guitar_caged_c_chord_over_fretboard}
\end{figure}

\infobox{Het is belangrijk om te benadrukken dat je niet altijd de volledige vormen van elk akkoord hoeft te spelen. Het CAGED-systeem moet worden gebruikt als een kaart om te zien waar de noten van een akkoord op de hals liggen, waarbij de grondtonen als ankerpunten dienen.}

Probeer nu te beginnen met bijvoorbeeld het open G-akkoord, en ga dan door alle CAGED-vormen heen. Dus in volgorde zouden de vormen zijn: (open) G, E, D, C, A.

Probeer ook bijvoorbeeld de E-vorm op de 7e fret te spelen, en ga dan naar beneden (richting de kop) met de CAGED-vormen. De volgorde van de vormen naar beneden zou zijn: E, G, A. Je zult merken dat je niet genoeg 'ruimte' hebt als je naar de C-vorm probeert te gaan. Maar je kunt nog steeds de noten op de 5e, 4e en 2e snaar spelen om een gedeeltelijke C-vorm te spelen.

\newpage

\subsection{Voorbeelden in nummers}

% https://www.youtube.com/watch?v=5nG7a5PdNyA
% Under The Bridge  - RHCP ()
	% Snow (Hey Oh)  - RHCP (E. C, G)
	% Pray for plagues - BMTH (sweep solo)
% all right now - Free
% wind cries mery - jimi hendrix
% The Bannered Mare - D major for F
	% island in the sun - weezer
% From so Far Away - All Shall Perish

\subsubsection{Island In The Sun - Weezer}

Een nummer als voorbeeld is "Island In The Sun" van "Weezer" (\autoref{fig:guitar_island_in_the_sun_weezer_intro}). Dit gebruikt de D-mineur, E-mineur, D- en E-vormen. Let op dat alleen de hoge 3 snaren van de E- en Em-vormen worden gebruikt.

\begin{itemize}
	\setlength\itemsep{0em}
	\item \writechord{Em}: Dm vorm
	\item \writechord{Am}: Em vorm
	\item \writechord{D}: D vorm
	\item \writechord{G}: E vorm
\end{itemize}


\begin{figure}[h]
	\centering
	\includegraphics[width=\textwidth]{../../MuseScore/Guitar/IslandInTheSunWeezerIntro.png}
	\caption{Intro van "Island In The Sun - Weezer"}
	\label{fig:guitar_island_in_the_sun_weezer_intro}
\end{figure}

\newpage

\subsubsection{Snow(Hey Oh) - Red Hot Chili Peppers}

Het nummer "Snow (Hey Oh)" van "Red Hot Chili Peppers" (\autoref{fig:guitar_snow_rhcp_intro}) gebruikt de Em-, C-, G- en E-eerste-inversie-vormen. Maar er wordt slechts een deel van elke vorm gespeeld.

\begin{itemize}
	\setlength\itemsep{0em}
	\item \writechord{G#m}: Em vorm
	\item \writechord{E} \& \writechord{F#}: C vorm
	\item \writechord{B}: G vorm
	\item \writechord{F#/A#} (\writechord{F#} eerste inversie): E eerste inversie vorm
\end{itemize}

\begin{figure}[h]
	\centering
	\includegraphics[width=\textwidth]{../../MuseScore/Guitar/Snow_RHCP_Intro.png}
	\caption{Intro van "Snow (Hey Oh) - Red Hot Chili Peppers"}
	\label{fig:guitar_snow_rhcp_intro}
\end{figure}