\section{Power/fifth akkoorden}

Het powerakkoord, of formeel een vijfde-akkoord genoemd, is een akkoord waarbij je alleen de grondtoon en de reine kwint speelt. Powerakkoorden bevatten geen terts, waardoor ze zowel niet majeur als mineur zijn.

Powerakkoorden worden vaak gebruikt op de elektrische gitaar met distortion. Als je het volledige akkoord (met de terts) zou spelen, kan het afhankelijk van de hoeveelheid distortion erg modderig klinken.

% https://www.hooktheory.com/blog/power-chords/

\subsection{Notatie}
Een powerakkoord kan worden aangegeven met (als voorbeeld het C-akkoord):

\begin{itemize}
	\item C5
	\item C(no 3)
\end{itemize}

Beide geven aan dat het akkoord geen terts bevat.

\subsection{Voorbeeld}

\begin{figure}[h]
	\centering
	\includegraphics[height=0.16\textheight]{../../Images/ChordsUsedInBasketCaseGreenDayVerse1.png}
	\caption{Akkoorden gebruikt in couplet 1 van "Basket Case - Green Day"}
	\label{fig:guitar_chords_verse_1_basket_case_green_day}
\end{figure}

Let op: op de meeste plekken op internet vind je alleen de akkoordnamen zonder de "5" of "(no 3)". In die gevallen kan je het beste naar het nummer luisteren. In de meeste gevallen, bij punk, rock, metal, enzovoort, zijn powerakkoorden een veilige keuze.

\begin{song}[verse/numbered, align-chords=l]{title={Basket Case - Green Day (verse 1)}, music={Green Day}}
	\begin{verse}
		^{E$\flat$5}Do you have the ^{B$\flat$5}time to ^{C5}listen to me ^{G5}whine? \\
		^{A$\flat$5}About nothing and ^{E$\flat$5}everything, all ^{B$\flat$5}at once \\
		^{E$\flat$5}I am one of ^{B$\flat$5}those ^{C5}melodramatic ^{G5}fools \\
		^{A$\flat$5}Neurotic to the ^{E$\flat$5}bone, no doubt about ^{B$\flat$5}it \\
	\end{verse}
\end{song}

\newpage

\section{Akkoord inversies \& open/gesloten ligging}

Tot nu toe hebben we geleerd dat een basisakkoord bestaat uit de \textbf{1e/grondtoon}, \textbf{3e} en \textbf{5e} graad van een (majeur of mineur) toonladder, gestapeld boven elkaar.

Stel dat we alle drie de noten \textbf{binnen één octaaf} spelen (dus de 1e, 3e en 5e graad zijn oplopend). Dit betekent dat de ligging van het akkoord \textbf{gesloten} is, omdat alles binnen één octaaf valt.

\autoref{fig:c_major_inversions} toont de 1e inversie (C/E) en 2e inversie (C/G). De 1e inversie betekent dat je de 3e graad (E) als basnoot speelt. De 2e inversie betekent dat je de 5e graad (G) als basnoot speelt. Om het akkoord in een gesloten ligging (binnen één octaaf) te houden, verplaats je de huidige basnoot simpelweg één octaaf omhoog zodat deze nu de hoogste toon in het akkoord is.


\begin{figure}[h]
	\centering
	\includegraphics[height=0.13\textheight]{../../MuseScore/Guitar/CMajorInversions.png}
	\caption{Inversies van het C-majeur akkoord/triade op gitaar}
	\label{fig:c_major_inversions}
\end{figure}

Voorbeelden van open ligging (noten over 2 of meer octaven) zijn te zien in \autoref{fig:c_major_open_voicings}.

\begin{figure}[h]
	\centering
	\includegraphics[height=0.13\textheight]{../../MuseScore/Guitar/CMajorOpenVoicings.png}
	\caption{Voorbeelden van open ligging voor het C-majeur akkoord}
	\label{fig:c_major_open_voicings}
\end{figure}

\newpage

\subsection{Voorbeelden}

In een eerdere sectie is al laten zien dat "We Are The Champions" van "Queen" akkoord inversies gebruikt.

Een ander voorbeeld komt uit "Thinking out loud" van Ed Sheeran. Hier wordt de D-akkoord 1e inversie (D/F\sharp) gespeeld. De akkoorddiagrammen staan in \autoref{fig:guitar_chords_verse_1_thinking_out_loud_ed_sheeran}.

\begin{figure}[h]
	\centering
	\includegraphics[height=0.16\textheight]{../../Images/EdSheeranThinkingOutLoudVerseChords.png}
	\caption{Akkoorden gebruikt in couplet 1 van "Thinking Out Loud - Ed Sheeran"}
	\label{fig:guitar_chords_verse_1_thinking_out_loud_ed_sheeran}
\end{figure}


\begin{song}[verse/numbered, align-chords=l]{title={Thinking out loud - Ed Sheeran (verse 1)}, music={Ed Sheeran}}
	\begin{verse}
		^{D} ^{D/F#}When your legs don't work like they ^{G}used to before ^{A} ^{D} \\
		^{D/F#}And I can't sweep you off of your ^{G}feet ^{A} ^{D} \\
		^{D/F#}Will your mouth still remember the ^{G}taste of my love ^{A} ^{D} \\
		^{D/F#}Will your eyes still smile from your ^{G}cheeks ^{A} \\
	\end{verse}
\end{song}

% Cadential 6 - 4

\newpage

\section{Verminderde \& verhoogde akkoorden}

Verminderde en verhoogde akkoorden hebben een wat onrustige klank. Dit maakt ze ideaal als doorgangsakkoord om wat spanning toe te voegen.

Waar majeur- en mineurakkoorden worden gekenmerkt door hun terts, worden verminderde en verhoogde akkoorden gekenmerkt door hun kwint.

Kort samengevat:

\begin{itemize}
	\item \textbf{Verminderd} akkoord: \textbf{mineur} akkoord met de \textbf{reine kwint verlaagd} met een halve toon (1 halve stap)
	\item \textbf{Verhoogd} akkoord: \textbf{majeur} akkoord met de \textbf{reine kwint verhoogd} met een halve toon (1 halve stap)
\end{itemize}

\autoref{tab:guitar_diminished_intervals} en \autoref{tab:guitar_augmented_intervals} laten zien hoe deze zich verhouden tot de majeur en mineur toonladders (en de akkoorden die daaruit zijn voortgekomen).

\begin{table}[h]
	\centering
	\begin{NiceTabular}{*{17}{c}}
		\Block{}{} & \Block{}{} & \Block{1-2}{\large{W}} & & \Block{1-2}{\large{H}} & & \Block{1-2}{\large{W}} & & \Block{1-2}{\large{W}} & & \Block{1-2}{\large{H}} & & \Block{1-2}{\large{W}} & & \Block{1-2}{\large{W}} & & \Block{}{} \\
		\Block{}{Mineur intervallen} & \Block[fill=ColorRootNote]{1-2}{1} & & \Block{1-2}{2} & & \Block[fill=ColorOtherNote]{1-2}{3$\flat$} & & \Block{1-2}{4} & & \Block[fill=ColorOtherNote]{1-2}{5} & & \Block{1-2}{6$\flat$} & & \Block{1-2}{7$\flat$} & & \Block{1-2}{8} & \\
		\Block{}{Verlaagd akkoord uit de mineur toonladder} & \Block[fill=ColorRootNote]{1-2}{1} & & \Block{1-2}{} & & \Block[fill=ColorOtherNote]{1-2}{3$\flat$} & & \Block{}{} & \Block[fill=ColorOtherNote]{1-2}{5$\flat$} & & \Block{}{} & \Block{1-2}{} & & \Block{1-2}{} & & \Block{1-2}{} &
	\end{NiceTabular}
	\caption{Intervallen van verminderde akkoorden}
	\label{tab:guitar_diminished_intervals}
\end{table}

\begin{table}[h]
	\centering
	\begin{NiceTabular}{*{17}{c}}
		\Block{}{} & \Block{}{} & \Block{1-2}{\large{W}} & & \Block{1-2}{\large{W}} & & \Block{1-2}{\large{H}} & & \Block{1-2}{\large{W}} & & \Block{1-2}{\large{W}} & & \Block{1-2}{\large{W}} & & \Block{1-2}{\large{H}} & & \Block{}{} \\
		\Block{}{Majeur intervallen} & \Block[fill=ColorRootNote]{1-2}{1} & & \Block{1-2}{2} & & \Block[fill=ColorOtherNote]{1-2}{3} & & \Block{1-2}{4} & & \Block[fill=ColorOtherNote]{1-2}{5} & & \Block{1-2}{6} & & \Block{1-2}{7} & & \Block{1-2}{8} & \\
		\Block{}{Verhoogd akkoord uit de majeur toonladder} & \Block[fill=ColorRootNote]{1-2}{1} & & \Block{1-2}{} & & \Block[fill=ColorOtherNote]{1-2}{3} & & \Block{1-2}{} & & \Block{}{} & \Block[fill=ColorOtherNote]{1-2}{5$\sharp$} & & \Block{}{} & \Block{1-2}{} & & \Block{1-2}{} &
	\end{NiceTabular}
	\caption{Intervallen van verhoogde akkoorden}
	\label{tab:guitar_augmented_intervals}
\end{table}

In \autoref{sec:building_chords_with_diatonic_scale} heb je het volgende al geleerd:

\begin{minipage}{0.5\textwidth}
	\begin{itemize}
		\item Mineur akkoord: 1 - 3$\flat$ - 5:
			\subitem Interval 1 - 3$\flat$: kleine terts
			\subitem Interval 3$\flat$ - 5: grote terts
	\end{itemize}
\end{minipage}
\hfill
\begin{minipage}{0.5\textwidth}
	\begin{itemize}
		\item Majeur akkoord: 1 - 3 - 5:
			\subitem Interval 1 - 3: grote terts
			\subitem Interval 3 - 5: kleine terts
	\end{itemize}
\end{minipage}

Let op dat zowel majeur- als mineurakkoorden een grote en kleine terts boven elkaar gestapeld hebben. Verminderde akkoorden hebben alleen kleine tertsen gestapeld en verhoogde akkoorden alleen grote tertsen:

\begin{minipage}{0.5\textwidth}
	\begin{itemize}
		\item Diminished chord: 1 - 3$\flat$ - 5$\flat$:
			\subitem Interval 1 - 3$\flat$: minor 3rd
			\subitem Interval 3$\flat$ - 5$\flat$: minor 3rd
	\end{itemize}
\end{minipage}
\hfill
\begin{minipage}{0.5\textwidth}
	\begin{itemize}
		\item Augmented chord: 1 - 3 - 5$\sharp$:
			\subitem Interval 1 - 3: major 3rd
			\subitem Interval 3 - 5$\sharp$: major 3rd
	\end{itemize}
\end{minipage}

Eerder in dit boek, bij \autoref{sec:identifying_dimished_chords_in_the_scale}, is uitgelegd dat een verminderd akkoord een tritonus bevat (oftewel 6 halve tonen). Dit komt door het stapelen van twee kleine tertsen.

\subsection{Notatie}
\begin{minipage}{0.4\textwidth}
	Verminderde akkoorden kunnen worden aangegeven met (als voorbeeld C als grondtoon):
	
	\begin{itemize}
		\item Cdim
		\item C($\flat$5)
		\item C\textsuperscript{o}
	\end{itemize}
\end{minipage}
\hfill
\begin{minipage}{0.4\textwidth}
	Verhoogde akkoorden kunnen worden aangegeven met (als voorbeeld C als grondtoon):
	
	\begin{itemize}
		\item Caug
		\item C($\sharp$5)
		\item C+
	\end{itemize}
\end{minipage}

\newpage

\subsection{Voorbeelden}

% Diminished allstart + don't look back in anger - oasis

Het nummer "All Stars" van "Smash Mouth" gebruikt het verminderde akkoord om wat spanning toe te voegen (\autoref{fig:guitar_chords_in_all_stars_smash_mouth}).
 
\begin{figure}[h]
	\centering
	\includegraphics[height=0.16\textheight]{../../Images/ChordsInChorusAllStarsSmashMouth.png}
	\caption{Akkoorden gebruikt in het refrein van "All Stars" van "Smash Mouth"}
	\label{fig:guitar_chords_in_all_stars_smash_mouth}
\end{figure}

\begin{song}[verse/numbered, align-chords=l]{title={All Stars - Smash Mouth (chorus)}, music={Smash Mouth}}
	\begin{chorus}
		^{F#}Hey now you're an ^{B}All Star get your ^{Cdim}game on, go ^{B}play \\
		^{F#}Hey now you're a ^{B}Rock Star get the ^{Cdim}show on get ^{B}paid \\
		And ^{F#}all that ^{B}glitters is ^{Cdim}gold \\
	 	^{B}Only shooting ^{F#}stars ^{E}break the ^{B}mold \\
	\end{chorus}
\end{song}


% Augmented Impossible Year Chords by Panic! At the Disco
% Augmented Stairway to heaven
% Augmented Life on mars - david bowie

Het nummer "Impossible Year" van "Panic! At The Disco" gebruikt verhoogde akkoorden om een zogenaamde "line cliché" te creëren. Dit is wanneer de grondtoon of de reine kwint van een akkoord een paar keer een halve toon omhoog (of omlaag) wordt verschoven, terwijl de rest van de noten hetzelfde blijft. Je ziet dat hier gebeuren met de reine kwint van het F-akkoord (zie \autoref{fig:guitar_chords_used_in_impossible_year_panic_at_the_disco}).

\begin{figure}[h]
	\centering
	\includegraphics[height=0.16\textheight]{../../Images/ChordsUsedInImpossibleYearPanicAtTheDisco.png}
	\caption{Enkele akkoorden uit "Impossible Year - Panic! At The Disco"}
	\label{fig:guitar_chords_used_in_impossible_year_panic_at_the_disco}
\end{figure}

\begin{song}[verse/numbered, align-chords=l]{title={Impossible Year - Panic! At The Disco (Part of Verse 1)}, music={Panic! At The Disco}}
	\begin{chorus}
		^{F}There's no sunshine ^{Faug} \\
		This ^{Dm/F}impossible year ^{Faug} \\
	\end{chorus}
\end{song}

\newpage

\section{Septiemakkoorden}

Eerst een herinnering aan welke akkoorden in de majeur- en mineurtoonladder majeur, mineur of verminderd zijn.

\begin{table}[h]
	\begin{minipage}{0.45\textwidth}
		\centering
		\begin{NiceTabular}{*{16}{P{0.05mm}}}
			\Block{}{} & \Block{1-2}{\large{W}} & & \Block{1-2}{\large{W}} & & \Block{1-2}{\large{H}} & & \Block{1-2}{\large{W}} & & \Block{1-2}{\large{W}} & & \Block{1-2}{\large{W}} & & \Block{1-2}{\large{H}} & & \Block{}{} \\
			\Block{1-2}{1} & & \Block{1-2}{2} & & \Block{1-2}{3} & & \Block{1-2}{4} & & \Block{1-2}{5} & & \Block{1-2}{6} & & \Block{1-2}{7} & & \Block{1-2}{8} & \\
			\Block{1-2}{\RomanNumeralCaps{1}} & & \Block{1-2}{\RomanNumeral{2}} & & \Block{1-2}{\RomanNumeral{3}} & & \Block{1-2}{\RomanNumeralCaps{4}} & & \Block{1-2}{\RomanNumeralCaps{5}} & & \Block{1-2}{\RomanNumeral{6}} & & \Block{1-2}{\RomanNumeral{7}\textsuperscript{o}} & &
		\end{NiceTabular}
		\caption{Akkoorden in de majeur toonladder}
		\label{tab:guitar_major_scale_chords_sec_7th_chords}
	\end{minipage}
	\hfill
	\begin{minipage}{0.45\textwidth}
		\centering
		\begin{NiceTabular}{*{16}{P{0.05mm}}}
			\Block{}{} & \Block{1-2}{\large{W}} & & \Block{1-2}{\large{H}} & & \Block{1-2}{\large{W}} & & \Block{1-2}{\large{W}} & & \Block{1-2}{\large{H}} & & \Block{1-2}{\large{W}} & & \Block{1-2}{\large{W}} & & \Block{}{} \\
			\Block{1-2}{1} & & \Block{1-2}{2} & & \Block{1-2}{3$\flat$} & & \Block{1-2}{4} & & \Block{1-2}{5} & & \Block{1-2}{6$\flat$} & & \Block{1-2}{7$\flat$} & & \Block{1-2}{8} & \\
			\Block{1-2}{\RomanNumeral{1}} & & \Block{1-2}{\RomanNumeral{2}\textsuperscript{o}} & & \Block{1-2}{\RomanNumeralCaps{3}} & & \Block{1-2}{\RomanNumeral{4}} & & \Block{1-2}{\RomanNumeral{5}} & & \Block{1-2}{\RomanNumeralCaps{6}} & & \Block{1-2}{\RomanNumeralCaps{7}} & &
		\end{NiceTabular}
		\caption{Akkoorden in de mineur toonladder}
		\label{tab:guitar_minor_scale_chords_sec_7th_chords}
	\end{minipage}
\end{table}

Nu voegen we bij elk akkoord de noot toe die twee toonladdertreden boven de 5e graad ligt.

\subsection{Majeur en mineur septiemakkoorden}

Een voorbeeld voor het grondakkoord van de toonladder is te zien in \autoref{tab:guitar_major_7th_chord_buildup} en \autoref{tab:guitar_minor_7th_chord_buildup}.

\begin{minipage}{0.43\textwidth}
	In \autoref{tab:guitar_major_7th_chord_buildup} (het majeurakkoord) een grote terts (4 halve tonen) wordt bovenop de 5e graad gestapeld. Dit levert een grote septiem op vanaf de grondtoon. Dit type akkoord (een majeurakkoord met een grote septiem) wordt genoteerd als \textbf{M7} of \textbf{maj7}. Bijvoorbeeld \textbf{CM7} of \textbf{Cmaj7}.
\end{minipage}
\hfill
\begin{minipage}{0.43\textwidth}
	In \autoref{tab:guitar_minor_7th_chord_buildup} (het mineurakkoord) een kleine terts (3 halve tonen) wordt bovenop de 5e graad gestapeld. Dit levert een kleine septiem op vanaf de grondtoon. Dit type akkoord (een mineurakkoord met een kleine septiem) wordt genoteerd als \textbf{m7} of \textbf{min7}. Bijvoorbeeld \textbf{Cm7} of \textbf{Cmin7}.
\end{minipage}


\begin{table}[h]
	\begin{minipage}{0.45\textwidth}
		\centering
		\begin{NiceTabular}{*{16}{P{0.05mm}}}
			\Block{}{} & \Block{1-2}{\large{W}} & & \Block{1-2}{\large{W}} & & \Block{1-2}{\large{H}} & & \Block{1-2}{\large{W}} & & \Block{1-2}{\large{W}} & & \Block{1-2}{\large{W}} & & \Block{1-2}{\large{H}} & & \Block{}{} \\
			\Block[fill=ColorRootNote]{1-2}{1} & & \Block{1-2}{2} & & \Block[fill=ColorOtherNote]{1-2}{3} & & \Block{1-2}{4} & & \Block[fill=ColorOtherNote]{1-2}{5} & & \Block{1-2}{6} & & \Block[fill=ColorOtherNote]{1-2}{7} & & \Block{1-2}{8} &
		\end{NiceTabular}
		\caption{Opbouw van een majeur septiemakkoord}
		\label{tab:guitar_major_7th_chord_buildup}
	\end{minipage}
	\hfill
	\begin{minipage}{0.45\textwidth}
		\centering
		\begin{NiceTabular}{*{16}{P{0.05mm}}}
			\Block{}{} & \Block{1-2}{\large{W}} & & \Block{1-2}{\large{H}} & & \Block{1-2}{\large{W}} & & \Block{1-2}{\large{W}} & & \Block{1-2}{\large{H}} & & \Block{1-2}{\large{W}} & & \Block{1-2}{\large{W}} & & \Block{}{} \\
			\Block[fill=ColorRootNote]{1-2}{1} & & \Block{1-2}{2} & & \Block[fill=ColorOtherNote]{1-2}{3$\flat$} & & \Block{1-2}{4} & & \Block[fill=ColorOtherNote]{1-2}{5} & & \Block{1-2}{6$\flat$} & & \Block[fill=ColorOtherNote]{1-2}{7$\flat$} & & \Block{1-2}{8} &
		\end{NiceTabular}
		\caption{Opbouw van een mineur septiemakkoord}
		\label{tab:guitar_minor_7th_chord_buildup}
	\end{minipage}
\end{table}

\subsection{Dominant septiemakkoord}

Maar wat als je een majeurakkoord met een kleine septiem hebt (een kleine terts bovenop de 5e graad)? Dit heet een \textbf{dominant 7} akkoord. Dit wordt genoteerd met alleen een \textbf{7}. Bijvoorbeeld \textbf{C7}.

Hier is wat nuance nodig. Een dominant septiemakkoord is een majeur triade met een kleine septiem. Maar een akkoord met dominante functie is het 5e akkoord in de toonladder. In de majeurtoonladder is het 5e akkoord ook een dominant septiemakkoord. In de mineurtoonladder is het 7e akkoord een dominant septiem en het 5e akkoord een mineurakkoord (met dominante functie). Een akkoord met dominante functie leidt naar de tonica (de grondtoon van de toonladder). Daarom zie je vaak dat in een mineurtoonladder het 5e akkoord wordt vervangen door een dominant septiemakkoord (in plaats van een mineurakkoord).

\infobox{Je kunt ook denken aan een mineurakkoord met een grote septiem. Dit komt niet van nature voor in de diatonische toonladder, maar wel als je een septiemakkoord maakt vanaf de grondtoon in de \textbf{harmonische mineurtoonladder}. Deze toonladder is de diatonische/natuurlijke mineurtoonladder die je hebt geleerd, maar met de kleine septiem een halve toon verhoogd tot een grote septiem.}

\subsection{Half-verminderde \& verminderde septiemakkoorden}

Er is ook het verminderde akkoord/triade. Dit heeft de grondtoon, een kleine terts en een verlaagde kwint. In de diatonische toonladder vind je alleen een verminderde triade met kleine septiem op de 7e toonladdergraad en dat heet een \textbf{half-verminderd} akkoord. Dit wordt genoteerd met \textbf{\o7} (let op de schuine streep) of met \textbf{m7\flat5}. Bijvoorbeeld \textbf{C\textsuperscript{\o7}} of \textbf{Cm7\flat5}

Als dat een half-verminderd akkoord heet, wat is dan een volledig verminderd septiemakkoord? Dat is een verminderde triade met een verminderde septiem vanaf de grondtoon. Een verminderde septiem is 1 halve toon lager dan een kleine septiem (hetzelfde als een grote sext). Dit wordt aangegeven met \textbf{o7} of \textbf{dim7}. Bijvoorbeeld \textbf{C\textsuperscript{o7}} of \textbf{Cdim7}. Dit komt niet voor in de diatonische toonladder, maar wel als het akkoord op de 7e toonladdergraad in de harmonische mineurtoonladder.

\newpage 

\subsection{Alles bij elkaar}

Met deze kennis kunnen de verschillende, natuurlijk voorkomende, septiemakkoorden in de diatonische majeur- en mineurtoonladder worden aangegeven in \autoref{tab:guitar_major_scale_7th_chords} en \autoref{tab:guitar_minor_scale_7th_chords}. Let op dat er Romeinse cijfers worden gebruikt. Hier betekent \textbf{M7} dat er een grote septiem wordt toegevoegd en \textbf{\o7} betekent half-verminderd, dus een kleine septiem bij een verminderde triade. Tot slot betekent \textbf{7} dat er een kleine septiem wordt toegevoegd. Dit laatste kan zowel bij een majeur (hoofdletter Romeins cijfer) als een mineur (kleine letter Romeins cijfer) akkoord worden toegepast. Bij een majeurakkoord is het een dominant septiemakkoord, bij een mineurakkoord een mineur septiemakkoord.

\begin{table}[h]
	\begin{minipage}{0.45\textwidth}
		\centering
		\begin{NiceTabular}{*{16}{P{0.05mm}}}
			\Block{}{} & \Block{1-2}{\large{W}} & & \Block{1-2}{\large{W}} & & \Block{1-2}{\large{H}} & & \Block{1-2}{\large{W}} & & \Block{1-2}{\large{W}} & & \Block{1-2}{\large{W}} & & \Block{1-2}{\large{H}} & & \Block{}{} \\
			\Block{1-2}{1} & & \Block{1-2}{2} & & \Block{1-2}{3} & & \Block{1-2}{4} & & \Block{1-2}{5} & & \Block{1-2}{6} & & \Block{1-2}{7} & & \Block{1-2}{8} & \\
			\Block{1-2}{\RomanNumeralCaps{1}\textsuperscript{M7}} & & \Block{1-2}{\RomanNumeral{2}\textsuperscript{7}} & & \Block{1-2}{\RomanNumeral{3}\textsuperscript{7}} & & \Block{1-2}{\RomanNumeralCaps{4}\textsuperscript{M7}} & & \Block{1-2}{\RomanNumeralCaps{5}\textsuperscript{7}} & & \Block{1-2}{\RomanNumeral{6}\textsuperscript{7}} & & \Block{1-2}{\RomanNumeral{7}\textsuperscript{\o7}} & &
		\end{NiceTabular}
		\caption{Septiemakkoorden in de majeur toonladder}
		\label{tab:guitar_major_scale_7th_chords}
	\end{minipage}
	\hfill
	\begin{minipage}{0.45\textwidth}
		\centering
		\begin{NiceTabular}{*{16}{P{0.05mm}}}
			\Block{}{} & \Block{1-2}{\large{W}} & & \Block{1-2}{\large{H}} & & \Block{1-2}{\large{W}} & & \Block{1-2}{\large{W}} & & \Block{1-2}{\large{H}} & & \Block{1-2}{\large{W}} & & \Block{1-2}{\large{W}} & & \Block{}{} \\
			\Block{1-2}{1} & & \Block{1-2}{2} & & \Block{1-2}{3$\flat$} & & \Block{1-2}{4} & & \Block{1-2}{5} & & \Block{1-2}{6$\flat$} & & \Block{1-2}{7$\flat$} & & \Block{1-2}{8} & \\
			\Block{1-2}{\RomanNumeral{1}\textsuperscript{7}} & & \Block{1-2}{\RomanNumeral{2}\textsuperscript{\o7}} & & \Block{1-2}{\RomanNumeralCaps{3}\textsuperscript{M7}} & & \Block{1-2}{\RomanNumeral{4}\textsuperscript{7}} & & \Block{1-2}{\RomanNumeral{5}\textsuperscript{7}} & & \Block{1-2}{\RomanNumeralCaps{6}\textsuperscript{M7}} & & \Block{1-2}{\RomanNumeralCaps{7}\textsuperscript{7}} & &
		\end{NiceTabular}
		\caption{Septiemakkoorden in de mineur toonladder}
		\label{tab:guitar_minor_scale_7th_chords}
	\end{minipage}
\end{table}

\autoref{tab:guitar_intervals_7th_chords} geeft een overzicht van de verschillende septiemakkoord-intervallen.

\begin{table}[h]
	\centering
	\begin{NiceTabular}{|l|l|l|}
		\hline
		\textbf{Chord} & \textbf{Intervals} & \textbf{Symbol} \\
		\hline
		\hline
		Major 7th & 1 - 3 - 5 - 7 & maj7, M7 \\
		\hline
		Minor 7th & 1 - 3\flat - 5 - 7\flat & min7, m7, (just 7 if using a lower-case roman numeral) \\
		\hline
		Dominant 7th & 1 - 3 - 5 - 7\flat & 7 \\
		\hline
		Half-diminished 7th & 1 - 3\flat - 5\flat - 7\flat & \o7, m7\flat5 \\
		\hline
		Diminished 7th & 1 - 3\flat - 5\flat - 7\flatflat & o7, dim7 \\
		\hline
	\end{NiceTabular}
	\caption{Intervallen voor septiemakkoorden}
	\label{tab:guitar_intervals_7th_chords}
\end{table}


\newpage

\subsection{Voorbeelden}

% Valerie - Amy Winehouse https://tabs.ultimate-guitar.com/tab/amy-winehouse/valerie-chords-490669
% Interstate love song - 
% Fly Me To The Moon - Frank Sinatra https://tabs.ultimate-guitar.com/tab/frank-sinatra/fly-me-to-the-moon-chords-1193296

Een nummer dat majeur-, mineur- en dominant septiemakkoorden bevat is "Valerie" van "Amy Winehouse". De akkoorddiagrammen staan in \autoref{fig:guitar_chords_valerie_amy_winehouse}. Dit nummer staat in de toonsoort \writechord{Eb}. Probeer te ontdekken waarom \writechord{Bb7} hier een dominant septiemakkoord is.

\begin{figure}[h]
	\centering
	\includegraphics[width=0.55\textwidth]{../../Images/ChordsInValerieAmyWinehouse.png}
	\caption{Akkoorden gebruikt in "Valerie- Amy Winehouse"}
	\label{fig:guitar_chords_valerie_amy_winehouse}
\end{figure}

\begin{song}[align-chords=l]{title={Valerie - Amy Winehouse (first verse + chorus)}, music={Amy Winehouse}}
	\begin{verse}
		^{Ebmaj7}Well, sometimes I go out by myself, and I look across the ^{Fm7}water \\
		And I ^{Ebmaj7}think of all the things, what you're doin', and in my head, I paint a ^{Fm7}picture \\
	\end{verse}
	\begin{chorus}
		^{Abmaj7}Since I've come on home, well, ^{Gm}my body's been a mess \\
		And I've ^{Abmaj7}missed your ginger hair, and ^{Gm}the way you like to dress \\
		^{Abmaj7}Won't you come on over? \\
		^{Gm}Stop makin' a fool out of ^{Bb}me \\
		^{Bb7}Why don't you come on over, Val^{Ebmaj7}erie? \\
		Val^{Fm7}erie, yeah \\
		Val^{Ebmaj7}erie, Val^{Fm7}erie \\
	\end{chorus}
\end{song}

\newpage

Een nummer waarin alle vier de verschillende septiemakkoorden voorkomen die in de diatonische toonladder voorkomen, is "Fly Me To The Moon" van "Frank Sinatra". De akkoorddiagrammen staan in \autoref{fig:guitar_chords_fly_me_to_the_moon_frank_sinatra}.

Let op dat er twee versies van het \writechord{Am7} akkoord zijn hier. De gesloten/barre versie wordt alleen gebruikt voor het eerste akkoord van het nummer. Daarna wordt de open vorm gebruikt. De reden hiervoor is de akkoorden die volgen en/of voorafgaan aan het \writechord{Am7} akkoord. Het eerste \writechord{Am7} akkoord wordt gevolgd door een barre-versie van het \writechord{Dm7} akkoord. Beide liggen op de positie van de 5e fret. Het spelen van de barre-versies van deze akkoorden geeft een iets helderder geluid door de gemeenschappelijke hoge A-noot op de eerste E-snaar. Je kunt natuurlijk ook de open versie van \writechord{Am7} spelen en dan de barre- of open versie van het \writechord{Dm7} akkoord. Geen van beide is goed of fout. Het is een kwestie van wat jij het mooist vindt klinken. Voor de rest van het couplet wordt de open versie van het \writechord{Am7} gespeeld. Ook hier komt dat doordat de omliggende akkoorden qua klank dichter bij het open \writechord{Am7} liggen dan bij het barre-akkoord. Een tweede reden is dat de akkoorden dichter bij elkaar gespeeld worden, waardoor je minder handbeweging hebt. Experimenteer gerust met de opties.

\begin{figure}[h]
	\centering
	\includegraphics[width=0.65\textwidth]{../../Images/ChordsInFlyMeToTheMoonFrankSinatra.png}
	\caption{Akkoorden gebruikt in "Fly Me To The Moon - Frank Sinatra"}
	\label{fig:guitar_chords_fly_me_to_the_moon_frank_sinatra}
\end{figure}

\begin{song}[align-chords=l]{title={Fly Me To The Moon - Frank Sinatra (first verse)}, music={Frank Sinatra}}
	\begin{verse}
		^{Am7}Fly me to the ^{Dm7}moon \\
		Let me ^{G7}play among the ^{Cmaj7}stars ^{C7}  \\
		And ^{Fmaj7}let me see what ^{Bm7b5}spring is like \\
		On a-^{E7}Jupiter and ^{Am7}Mars ^{A7}  \\
		^{Dm7}In other words, ^{G7} hold my ^{Cmaj7}hand ^{F7} ^{Em7} ^{A7}  \\
		^{Dm7}In other words, ^{G7} baby, ^{Cmaj7}kiss me ^{Bm7b5} ^{E7} \\
	\end{verse}
\end{song}

\newpage

\section{Zwevende (sus) akkoorden}

Een sus akkoord vervangt de terts (3e graad) van het akkoord door een grote secunde (sus2) of een reine kwart (sus4).

Hierdoor ontstaat een akkoord dat zowel niet majeur als mineur is. Let op dat zowel de majeur- als de mineurtoonladder een grote secunde en een reine kwart bevatten. Sus akkoorden worden vaak kort gespeeld als variatie om de muziek wat levendiger te maken, terwijl de grondtoon hetzelfde blijft.

De onderstaande tabellen tonen de structuur van sus2 en sus4 akkoorden.

\begin{table}[h]
	\begin{minipage}{0.45\textwidth}
		\centering
		\begin{NiceTabular}{*{16}{P{0.05mm}}}
			\Block{}{} & \Block{1-2}{\large{W}} & & \Block{1-2}{\large{W}} & & \Block{1-2}{\large{H}} & & \Block{1-2}{\large{W}} & & \Block{1-2}{\large{W}} & & \Block{1-2}{\large{W}} & & \Block{1-2}{\large{H}} & & \Block{}{} \\
			\Block[fill=ColorRootNote]{1-2}{1} & & \Block[fill=ColorOtherNote]{1-2}{2} & & \Block{1-2}{3} & & \Block{1-2}{4} & & \Block[fill=ColorOtherNote]{1-2}{5} & & \Block{1-2}{6} & & \Block{1-2}{7} & & \Block{1-2}{8} &
		\end{NiceTabular}
		\caption{Building a sus2 chord from the major scale}
		\label{tab:guitar_sus2_chord_buildup_major_scale}
	\end{minipage}
	\hfill
	\begin{minipage}{0.45\textwidth}
		\centering
		\begin{NiceTabular}{*{16}{P{0.05mm}}}
			\Block{}{} & \Block{1-2}{\large{W}} & & \Block{1-2}{\large{W}} & & \Block{1-2}{\large{H}} & & \Block{1-2}{\large{W}} & & \Block{1-2}{\large{W}} & & \Block{1-2}{\large{W}} & & \Block{1-2}{\large{H}} & & \Block{}{} \\
			\Block[fill=ColorRootNote]{1-2}{1} & & \Block{1-2}{2} & & \Block{1-2}{3} & & \Block[fill=ColorOtherNote]{1-2}{4} & & \Block[fill=ColorOtherNote]{1-2}{5} & & \Block{1-2}{6} & & \Block{1-2}{7} & & \Block{1-2}{8} &
		\end{NiceTabular}
		\caption{Building a sus4 chord from the major scale}
		\label{tab:guitar_sus4_chord_buildup_major_scale}
	\end{minipage}
	
	\begin{minipage}{0.45\textwidth}
		\centering
		\begin{NiceTabular}{*{16}{P{0.05mm}}}
			\Block{}{} & \Block{1-2}{\large{W}} & & \Block{1-2}{\large{H}} & & \Block{1-2}{\large{W}} & & \Block{1-2}{\large{W}} & & \Block{1-2}{\large{H}} & & \Block{1-2}{\large{W}} & & \Block{1-2}{\large{W}} & & \Block{}{} \\
			\Block[fill=ColorRootNote]{1-2}{1} & & \Block[fill=ColorOtherNote]{1-2}{2} & & \Block{1-2}{3$\flat$} & & \Block{1-2}{4} & & \Block[fill=ColorOtherNote]{1-2}{5} & & \Block{1-2}{6$\flat$} & & \Block{1-2}{7$\flat$} & & \Block{1-2}{8} &
		\end{NiceTabular}
		\caption{Building a sus2 chord from the minor scale}
		\label{tab:guitar_sus2_chord_buildup_minor_scale}
	\end{minipage}
	\hfill
	\begin{minipage}{0.45\textwidth}
		\centering
		\begin{NiceTabular}{*{16}{P{0.05mm}}}
			\Block{}{} & \Block{1-2}{\large{W}} & & \Block{1-2}{\large{H}} & & \Block{1-2}{\large{W}} & & \Block{1-2}{\large{W}} & & \Block{1-2}{\large{H}} & & \Block{1-2}{\large{W}} & & \Block{1-2}{\large{W}} & & \Block{}{} \\
			\Block[fill=ColorRootNote]{1-2}{1} & & \Block{1-2}{2} & & \Block{1-2}{3$\flat$} & & \Block[fill=ColorOtherNote]{1-2}{4} & & \Block[fill=ColorOtherNote]{1-2}{5} & & \Block{1-2}{6$\flat$} & & \Block{1-2}{7$\flat$} & & \Block{1-2}{8} &
		\end{NiceTabular}
		\caption{Building a sus4 chord from the minor scale}
		\label{tab:guitar_sus4_chord_buildup_minor_scale}
	\end{minipage}
\end{table}

\subsection{Notation}

Sus akkoorden worden als volgt genoteerd (met C als grondtoon):

\begin{itemize}
	\item Csus2
	\item Csus4
\end{itemize}

\subsection{Inversies van sus akkoorden}

Een interessante eigenschap van sus akkoorden is dat de inversie ervan resulteert in een ander sus-akkoord.

In de onderstaande diagrammen geven de cijfers aan van welke en naar welke toonladdergraad de pijlen lopen. De cijfers op de pijlen geven het aantal halve tonen in het interval aan.

\begin{tikzcd}
	graad \ar[r,"interval"] & graad \ar[r,"interval"] & graad  \ar[ll,bend left,dashed,"interval"]
\end{tikzcd}

\autoref{tab:guitar_sus_inversions} laat de inversies van sus2 en sus4 akkoorden zien.

Hier wordt ook de term \textit{kwartakkoord} genoemd. Dit is een akkoord dat is opgebouwd door reine kwarten op elkaar te stapelen.

\begin{table}[h]
	\centering
	\begin{NiceTabular}{P{4.5cm}|P{4.5cm}|P{4.5cm}}
		\Block{1-3}{\textit{sus2}} & & \\ 
		\hline
		Grondtoon & Eerste inversie & Tweede inversie \\
		\textbf{sus2} & Kwartakkoord met 2 als grondtoon & \textbf{sus4} met 5 als grondtoon \\
		\hline
		{
			\begin{tikzcd}
				1 \ar[r,"2"] & 2 \ar[r,"5"] & 5 \ar[ll,bend left,dashed,"5"]
			\end{tikzcd}
		}
	 	& {
	 		\begin{tikzcd}
	 			2 \ar[r,"5"] & 5 \ar[r,"5"] & 1 \ar[ll,bend left,dashed,"2"]
	 		\end{tikzcd}
 		} & {
 			\begin{tikzcd}
	 			5 \ar[r,"5"] & 1 \ar[r,"2"] & 2 \ar[ll,bend left,dashed,"5"]
 			\end{tikzcd}} \\
		\hline
		\Block{1-3}{} & & \\
		\Block{1-3}{\textit{sus4}} & & \\
		\hline
		Grondtoon & Eerste inversie & Tweede inversie \\
		\textbf{sus4} & \textbf{sus2} met 4 als grondtoon & Kwartakkoord met 5 als grondtoon \\
		\hline
		{
			\begin{tikzcd}
				1 \ar[r,"5"] & 4 \ar[r,"2"] & 5 \ar[ll,bend left,dashed,"5"]
			\end{tikzcd}
		}
		& {
			\begin{tikzcd}
				4 \ar[r,"2"] & 5 \ar[r,"5"] & 1 \ar[ll,bend left,dashed,"5"]
			\end{tikzcd}
		} & {
			\begin{tikzcd}
				5 \ar[r,"5"] & 1 \ar[r,"5"] & 4 \ar[ll,bend left,dashed,"2"]
		\end{tikzcd}} \\
	\end{NiceTabular}
	\caption{Intervallen van sus akkoord inversies}
	\label{tab:guitar_sus_inversions}
\end{table}

\newpage

\subsection{Voorbeelden}

In het nummer "SummerOf'69" van "Bryan Adams" worden de sus2- en sus4-akkoorden op het A- en D-akkoord gebruikt aan het einde van de refreinen en in het laatste couplet.

\begin{figure}[h]
	\centering
	\includegraphics[width=0.65\textwidth]{../../Images/ChordsInSummerOf69BryanAdams.png}
	\caption{Akkoorden gebruikt in "SummerOf'69 - Bryan Adams"}
	\label{fig:guitar_chords_summer_of_69_bryan_adams}
\end{figure}

\begin{song}[align-chords=l]{title={SummerOf'69 - Bryan Adams (last verse + chorus)}, music={Bryan Adams}}
	\begin{verse}
		^{Dsus2} ^{D}And now^{Dsus4} the times^{D} are^{Dsus2} chan^{D}gin' {} \\
		^{Asus2} Look^{A} at every^{Asus4}thing that's come^{A} and ^{Asus2}gone^{A} {} \\
		^{Dsus2} ^{D}Some^{Dsus4}times, when^{D} I play that old ^{Dsus2}six-^{D}string {} \\
		^{Asus2} Think^{A} about you, ^{Asus4}wonder what^{A} went ^{Asus2}wrong^{A} \\
	\end{verse}
	\begin{chorus}
		^{Bm} Standin' on your ^{A}mama's porch, ^{D} You told me that it'd ^{G}last forever \\
		^{Bm} Oh, and when you ^{A}held my hand, ^{D} I knew that it was ^{G}now or never \\
		^{Bm} Those were the ^{A}best days of my life \\
		^{Dsus2} ^{D} ^{Dsus4} ^{D} ^{Dsus2} Oh^{D}, yeah, \\
		^{Asus2} ^{A} ^{Asus4} ^{A}back in the ^{Asus2}summer ^{A}of ^{Dsus2}sixty ^{D}nine, ^{Dsus4} ^{D} ^{Dsus2} ^{D} \\
	\end{chorus}
\end{song}

\newpage

Sus akkoorden kunnen natuurlijk ook in een melodie gebruikt worden. Het nummer "Breath" van "Rioghan" doet dit (zie \autoref{fig:guitar_breath_rioghan_intro}).

\begin{figure}[h]
	\centering
	\includegraphics[width=\textwidth]{../../MuseScore/Guitar/GuitarBreathRioghanIntro.png}
	\caption{Intro van "Breath - Rioghan"}
	\label{fig:guitar_breath_rioghan_intro}
\end{figure}


% Summer of 69 - Bryan Adams
% Breath - Rioghan
% Cold as ice - Foreigner
% Wonderboy - tenacious D
% Inversions

\newpage

\section{Extended \& Add akkoorden}
TODO
