\section{Stemmen}

Je gitaar moet gestemd zijn. Dit betekent dat elke snaar een bepaalde toonhoogte heeft.

In \autoref{fig:guitar_string_names} zie je de namen (letters) van de dunste (\textit{e}) tot de dikste (\textit{E}) snaar.

\begin{figure}[h]
    \centering
    \includegraphics[width=0.8\textwidth]{../../Images/guitar-neck-string-names.png}
    \caption{Namen van de snaren van de gitaar}
    \label{fig:guitar_string_names}
\end{figure}

Een ezelsbruggetje is (van hoog/dun naar laag/dik):

\begin{minipage}{0.25\textwidth}
    \vspace{3mm}
    \begin{itemize}
        \setlength\itemsep{0em}
        \item[6)] \textbf{E} en
        \item[5)] \textbf{A} ap
        \item[4)] \textbf{D} ie
        \item[3)] \textbf{G} een
        \item[2)] \textbf{B} ananen
        \item[1)] \textbf{e} et 
    \end{itemize}
    \vspace{3mm}
\end{minipage}
\hfill
\begin{minipage}{0.7\textwidth}
    \infobox{Let op: dit is de standaardstemming. Soms wordt de gitaar anders gestemd. Dat wordt dan expliciet vermeld.}
\end{minipage}

\begin{minipage}{0.5\textwidth}
Je gebruikt een stemapparaat om te stemmen (zie \autoref{fig:tuning}). Het stemapparaat geeft een nootnaam aan, en dan moet je omhoog of omlaag stemmen tot de juiste noot op het scherm verschijnt. Of het toont een snaarnummer en je moet de 'wijzer' in het midden krijgen. Dat is afhankelijk van het type stemapparaat dat je hebt.

Wees voorzichtig met het omhoog stemmen (naar een hogere toon). Vooral de dunne snaren kunnen breken als ze te strak staan.
\end{minipage}
\hfill
\begin{minipage}{0.3\textwidth}
    \centering
    \includegraphics[width=\textwidth]{../../Images/guitar-tuning.jpg}
    \captionof{figure}{Een stemapparaat gebruiken op een gitaar}
    \label{fig:tuning}
\end{minipage}

Een andere stemmethode is gebaseerd op het verschil in toonhoogte tussen de snaren. In \autoref{fig:guitar_relative_tuning} zie je welke posities op de hals dezelfde toon hebben als de dunnere snaar eronder.

\begin{figure}[h]
    \centering
    \includegraphics[width=0.4\textwidth]{../../Images/GuitarRelativeTuning.png}
    \caption{Relatief stemmen}
    \label{fig:guitar_relative_tuning}
\end{figure}
