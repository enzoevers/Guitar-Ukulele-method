\section{Zit-/sta-positie}

\begin{figure}[h]
  \begin{subfigure}[b]{0.45\textwidth}
    \includegraphics[width=\textwidth]{../../Images/Letty_Guitar-Shooting_sitting.jpg}
    \caption{}
    \label{fig:positin_sitting}
  \end{subfigure}
  \hfill
  \begin{subfigure}[b]{0.45\textwidth}
    \includegraphics[width=\textwidth]{../../Images/Letty_Guitar-Shooting_standing.jpg}
    \caption{}
    \label{fig:positin_standing}
  \end{subfigure}
  \caption{\cite{SitStandPosition}}
  \label{fig:positin}
\end{figure}

\infobox{Deze methode gaat uit van een rechtshandige speler. Ben je linkshandig, vervang dan “rechts” door “links” en omgekeerd.}

\textbf{Allereerst: er is niet één juiste manier om gitaar te spelen. Afhankelijk van wat jij het prettigst vindt, of de stijl die je speelt, kun je een andere houding aannemen. Of je wisselt simpelweg tussen verschillende posities bij verschillende stukken.}

Voor stukken die technischer en/of sneller zijn zul je waarschijnlijk comfortabeler en nauwkeuriger spelen als je zit in de klassieke houding. De klassieke zithouding vertaalt ook beter naar een staande positie (zie \autoref{fig:positin_standing}).

In de klassieke houding plaats je de gitaar op je linkerbeen en is het linkerbeen iets verhoogd. Je kunt hiervoor een voetenbankje gebruiken (zie de linkervoet in \autoref{fig:positin_sitting}).