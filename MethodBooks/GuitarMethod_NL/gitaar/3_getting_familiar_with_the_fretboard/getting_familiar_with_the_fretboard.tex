\section{Kan ik al het nummer spelen dat ik wil?}

Het antwoord is: "soort van". Tot nu toe heb je je eerste klanken gemaakt en mee gespeeld met een bekend nummer.

Door alleen tabs te lezen kun je al een eind komen. Maar je zult waarschijnlijk gaan merken dat je vast komt te zitten in bepaalde patronen, en nummers mee kan spelen zonder echt te weten wat er muzikaal gebeurt. En \textbf{als dat voldoende voor je is, is dat helemaal prima}.

Echter, door vertrouwd te raken met het fretbord, weet waar welke noot zit, en wat theorie leert om het allemaal te verbinden, kun je veel creatiever zijn met muziek.

Later leer je ook over toonladders, waarmee je kunt improviseren. Je leert over verschillende soorten akkoorden en waar je deze op het fretbord kunt vinden. Om te weten welke fretten (of akkoorden) je moet spelen, moet je weten welke noten bij de fretten horen.

De doelen van dit hoofdstuk zijn je vertrouwd maken met het lezen van noten op een notenbalk, de snelheid van verschillende noten, en te leren waar de noten op het fretbord zijn.

\section{Muzieknotatie}

Je hebt de muzieknotatie van \autoref{fig:music_note_names_on_staff} al gezien in de vorige oefeningen. Maar de betekenis ervan was nog niet uitgelegd.

\begin{itemize}
	\item De letters A-G op de notenbalk geven aan welke lijn op de notenbalk welke nootnaam heeft.
	\item De noten gaan alfabetisch omhoog per stap (begin weer met A na G). 
	\item Noten tussen de lijnen spellen mooi \textbf{"FACE"}, wat het gemakkelijk maakt om te onthouden.
	\item Noten op de lijnen kunnen worden onthouden met het ezelsbruggetje "\textbf{E}elke \textbf{G}oede \textbf{B}oer \textbf{D}ie \textbf{F}ietst".
	\item Het meest linkse symbool (\clefG) wordt de G-sleutel genoemd. Merk op dat de krul van de G-sleutel (tweede lijn van onder) aangeeft waar de G zich bevindt. Meer specifiek, de G4.
	\subitem Merk op dat de G-sleutel die in \autoref{fig:music_note_names_on_staff} wordt weergegeven, iets anders is dan de sleutels die in eerdere oefeningen zijn gezien. Voor gitaarnotatie zie je soms een klein 8'tje onder de sleutel. Dit betekent dat alle noten een octaaf lager klinken. Dus \textbf{de krul van een G-sleutel die een 8'tje eronder heeft, zou een G3 aangeven in plaats van een G4}. Wanneer een stuk voor gitaar is geschreven, wordt het 8'tje soms weggelaten.
	\item De verticale lijn in het midden geeft het begin/einde van een maat aan.
	\item De dikkere verticale lijn aan het einde geeft het einde van het stuk aan.
\end{itemize}

\begin{figure}[h]
	\centering
	\includegraphics[width=0.55\textwidth]{../../Images/MusicNotation_MeasureNoteNames.png}
	\caption{Noten namen op de notenbalk in twee maten}
	\label{fig:music_note_names_on_staff}
\end{figure}

\newpage

\section{Tellen}

Tot nu toe hebben we maar één type noot gezien. De kwartnoot (\quarterNote). Er zijn echter meer. Zie \autoref{fig:note_duration_basic}. In dezelfde tijd dat een hele noot wordt gespeeld, kunnen twee halve noten worden gespeeld. En in dezelfde duur als een halve noot, kunnen twee kwartnoten worden gespeeld, etc.

De \lilyTimeSignature{4}{4} betekent dat er 4 (bovenste getal) kwartnoten (onderste getal) in een maat passen.

\textbf{Belangrijk}: Een hele noot (\wholeNote) is gelijk aan 4 kwartnoten (\quarterNote). Het is \textbf{niet} gelijk aan een hele maat. Maar omdat de maatsoort zegt dat er 4 kwartnoten in een maat staan, neemt de hele noot hier een hele maat in beslag.\newline

De "\quarterNote = 60" geeft het tempo aan en zegt hoeveel noten van de aangegeven duur in een minuut passen (de BPM). In dit geval passen 60 kwartnoten in een minuut. Dus het is 60 BPM, wat betekent dat een kwartnoot gelijk is aan 1 seconde. "\quarterNote = 80" zou 80 BPM zijn, en een kwartnoot zou 0,75 seconde zijn.

\begin{figure}[h]
	\centering
	\includegraphics[width=\textwidth]{../../MuseScore/Guitar/MusicNotation/NoteDurations_Basic.png}
	\caption{Nootduur}
	\label{fig:note_duration_basic}
\end{figure}

Er zijn ook andere maatsoorten. De bovenste waarde geeft aan hoeveel noten van de duur van de onderste waarde in een maat passen. Dus een \lilyTimeSignature{3}{4} maatsoort kan 3 kwartnoten per maat bevatten. En een \lilyTimeSignature{6}{8} maatsoort kan 6 achtste noten per maat bevatten. Merk op dat \lilyTimeSignature{3}{4} en \lilyTimeSignature{6}{8} dezelfde duur per maat aangeven, maar ze geven een ander gevoel. Dit wordt gedemonstreerd in \autoref{fig:time_signatures}.

In \autoref{fig:time_signatures} zie je ook een nieuwe duurnotatie. In de eerste maat met \lilyTimeSignature{6}{8} timing, zijn er puntjes naast de noten (\quarterNoteDottedDown). Dit betekent dat de noot een duur heeft van 1,5 keer de oorspronkelijke duur.

Het ">"-symbool betekent dat deze noot met een krachtiger accent moet worden gespeeld. De \textbf{vetgedrukte} cijfers boven de noten geven de telling van de noten aan. Een vetgedrukt cijfer betekent dat je er een accent op legt, maar minder geaccentueerd dan de noten waar ook een ">"-symbool bij staat.

\begin{figure}[h]
	\centering
	\includegraphics[width=\textwidth]{../../MuseScore/Guitar/MusicNotation/TimeSignature.png}
	\caption{Maatsoorten}
	\label{fig:time_signatures}
\end{figure}

Herinner je je \autoref{fig:exercise_nothing_else_matters_metallica_intro_pima} (Metallica - Nothing else matters (intro))? Dat is ook in \lilyTimeSignature{6}{8}.

\newpage

Waar noten aangeven wanneer je een geluid moet maken, geven rusten aan wanneer je stil moet zijn. In \autoref{fig:guitar_rests} zijn de meest voorkomende rustduren weergegeven.

\begin{figure}[h]
	\centering
	\includegraphics[width=\textwidth]{../../MuseScore/Guitar/GuitarRests.png}
	\caption{Rust notaties van verschillende duur}
	\label{fig:guitar_rests}
\end{figure}

In \autoref{fig:guitar_rests_exercise} is een oefening gegeven om de rusten te tellen. Vergeet niet om dit langzaam te doen en bewust te zijn van de tellen. Als hulp is het tempo ingesteld op 60 kwartnoten per minuut (BPM). Op deze manier is elke kwartnoot 1 seconde. Maar voel je vrij om het langzamer te spelen.

\begin{figure}[h]
	\centering
	\includegraphics[width=\textwidth]{../../MuseScore/Guitar/GuitarRestsExercise.png}
	\caption{Rust notaties van verschillende duur}
	\label{fig:guitar_rests_exercise}
\end{figure}


\newpage

\section{De belangrijkste noten leren} \label{sec:learning_main_notes}

Als eerste nummer dat meerdere nootduren gebruikt, en om de eerste noten op de gitaar te leren, zullen we Jingle Bells spelen (\autoref{fig:jingle_bells}). De noten die voor dit nummer worden gebruikt, zie je in \autoref{fig:notes_in_jingle_bells}.

\begin{figure}[h]
	\centering
	\includegraphics[height=0.12\textheight]{../../MuseScore/Guitar/NotesInJingleBells.png}
	\caption{Noten gebruikt in Jingle Bells}
	\label{fig:notes_in_jingle_bells}
\end{figure}

Nu kan Jingle Bells worden gespeeld.

\begin{figure}[h]
	\centering
	\includegraphics[width=\textwidth]{../../MuseScore/Guitar/GuitarJingleBells.png}
	\caption{Jingle Bells}
	\label{fig:jingle_bells}
\end{figure}

% Ma Meilleure Ennemie - Stromea
% BOB - NOFX
% Another One Bites The dust - Queen

\newpage

Om een paar noten meer te leren, zullen we de "Tetris" melodie spelen. De noten van \autoref{fig:notes_for_tetris_first_part} worden in dit deuntje gebruikt. De enige nieuwe noten zijn A en B.

\begin{figure}[h]
	\centering
	\includegraphics[height=0.12\textheight]{../../MuseScore/Guitar/NotesUsedInTetris_FirstPart.png}
	\caption{Noten in het eerste deel van de Tetris melodie}
	\label{fig:notes_for_tetris_first_part}
\end{figure}

In \autoref{fig:tetris_simple_first_part} is het eerste deel van de Tetris melodie geschreven. De volledige melodie vereist dat we het hebben over kruis- en mollen. Dus we zullen de volledige melodie later spelen. De akkoorden boven de notenbalk kunnen voorlopig worden genegeerd.

\begin{figure}[h]
	\centering
	\includegraphics[width=\textwidth]{../../MuseScore/Guitar/GuitarTetrisFirstPart.png}
	\caption{Eerste deel van de Tetris melodie}
	\label{fig:tetris_simple_first_part}
\end{figure}

\infobox{De "Tetris" melodie is afgeleid van een Russisch volkslied genaamd "Korobeiniki", dat is gebaseerd op een gedicht met een vergelijkbare naam geschreven door Nikolay Nekrasov. \cite{KorobeinikiWiki}}

\newpage

Het nummer "Shake It Off" van "Taylor Swift" (\autoref{fig:shake_it_off_taylor_swift}) gebruikt slechts drie noten voor het hele nummer. Deze noten zijn weergegeven in \autoref{fig:notes_for_shake_it_off_taylor_swift}. Je leert hier de nieuwe lage G-noot.

\begin{figure}[h]
	\centering
	\includegraphics[height=0.12\textheight]{../../MuseScore/Guitar/NotesUsedInShakeIfOffTaylorSwift.png}
	\caption{Noten gebruikt voor "Shake It Off - Taylor Swift"}
	\label{fig:notes_for_shake_it_off_taylor_swift}
\end{figure}

Er is een nieuw symbool geïntroduceerd. Het herhalingssymbool zoals je in \autoref{fig:guitar_repeat_symbol} ziet. Wanneer je aan het einde van de maat komt waar het rechterherhalingssymbool staat, ga je terug naar het linkse herhalingssymbool. Als er geen linkse herhalingssymbool is (zoals in \autoref{fig:shake_it_off_taylor_swift}), dan begin je vanaf het begin van het nummer. Wanneer je weer bij het rechterherhalingssymbool komt, speel je deze keer gewoon verder.

\begin{figure}[h]
	\centering
	\includegraphics[height=0.05\textheight]{../../MuseScore/Guitar/GuitarRepeatSymbol.png}
	\caption{Het herhalingssymbool}
	\label{fig:guitar_repeat_symbol}
\end{figure}

\begin{figure}[h]
	\centering
	\includegraphics[width=\textwidth]{../../MuseScore/Guitar/GuitarShakeItOffTaylorSwift.png}
	\caption{Herhaald deel voor "Shake It Off - Taylor Swift"}
	\label{fig:shake_it_off_taylor_swift}
\end{figure}

\newpage

We hebben nu alle niet-kruis/moll-noten gespeeld. Maar elke noot kan op verschillende plaatsen worden gespeeld, en met verschillende tonen.

Laten we de melodie van "Memory" uit de musical "Cats" nemen \autoref{fig:memory_cats}. Het gebruikt de meeste noten die we al geleerd hebben, maar het gebruikt ook een lagere G, F, en E (\autoref{fig:notes_g_f_e_3}).

\begin{figure}[h]
	\centering
	\includegraphics[height=0.12\textheight]{../../MuseScore/Guitar/NotesUsedInMemoryCats.png}
	\caption{De G, F, en G, noten op de 3e en 4e snaar}
	\label{fig:notes_g_f_e_3}
\end{figure}

Het gebruikt ook een \textbf{nieuw symbool}. Het \textbf{verbindingsteken} (die wordt gebruikt om de noten van maat 5 en 6 te verbinden in \autoref{fig:memory_cats}). Dit symbool geeft aan dat de duur van de eerste noot die de verbinding begint, de opgetelde duur heeft van alle opeenvolgende identiek genoteerde noten die aan elkaar zijn verbonden. Alle verbonden noten na de noot die de verbinding begint, worden daarom niet gespeeld. Alleen geteld.

\begin{figure}[h]
	\centering
	\includegraphics[width=\textwidth]{../../MuseScore/Guitar/GuitarMemoryCats.png}
	\caption{Memory uit de musical Cats}
	\label{fig:memory_cats}
\end{figure}

Een ander nummer dat je kent en dat alle noten gebruikt die we tot nu toe geleerd hebben, is Happy birthday (\autoref{fig:happy_birthday}). 

\begin{figure}[h]
	\centering
	\includegraphics[width=\textwidth]{../../MuseScore/Guitar/GuitarHappyBirthday.png}
	\caption{Happy birthday}
	\label{fig:happy_birthday}
\end{figure}

\newpage

Op de volgende pagina zie je een versie van het nummer "Golden" uit de serie "KPop Demon Hunters". Dit nummer introduceert de lage D-noot. De noten die in dit nummer worden gebruikt, zijn te zien in \autoref{fig:notes_for_golden_kpop_demon_hunters}.

\begin{figure}[h]
	\centering
	\includegraphics[height=0.12\textheight]{../../MuseScore/Guitar/NotesUsedInGoldenKPopDemonHunters.png}
	\caption{Noten gebruikt voor "Golden" uit de serie "KPop Demon Hunters"}
	\label{fig:notes_for_golden_kpop_demon_hunters}
\end{figure}

Dit nummer introduceert ook een extra optie voor het herhalingssymbool zoals gezien in \autoref{fig:different_measure_on_second_repeat}. Wat dit betekent is dat de eerste keer dat je de maten tussen de herhalingssymbolen speelt, je gewoon de maten 17, 18, 19, \textbf{20} speelt. Maar de tweede keer speel je in plaats daarvan de maten 17, 18, 19, \textbf{21}.

\begin{figure}[h]
	\centering
	\includegraphics[width=\textwidth]{../../MuseScore/Guitar/GuiterGoldenKPopDemonHunters_Simple_DifferentRepeatMeasure.png}
	\caption{Extra optie voor het herhalingssymbool}
	\label{fig:different_measure_on_second_repeat}
\end{figure}

Nog één ding voordat we het nummer spelen. Bij sommige herhalingssymbolen staat het aantal keren dat je moet spelen (zie \autoref{fig:extra_repeat_count}). Normaal gesproken herhaal je een sectie maar één keer (dus je speelt het twee keer). Maar in dit nummer speel je sommige secties meer dan twee keer. In die gevallen wordt het aantal keren dat je de sectie speelt aangegeven.

\begin{figure}[h]
	\centering
	\includegraphics[height=0.05\textheight]{../../MuseScore/Guitar/GuiterGoldenKPopDemonHunters_Simple_MultiRepeat.png}
	\caption{Meerdere herhalingen indicatie}
	\label{fig:extra_repeat_count}
\end{figure}

\includepdf[pages=-,pagecommand={\thispagestyle{headings}}]{../../MuseScore/Guitar/GuiterGoldenKPopDemonHunters_Simple.pdf}

\newpage

Tot nu toe zijn er alleen melodieën met enkele noten gespeeld. Maar soms wil je ook wat meer bastonen laten doorklinken om een voller geluid te krijgen. Een manier om dat te doen is door arpeggio's te spelen. Een arpeggio is in feite het spelen van een akkoord (groep noten) één noot tegelijk. Akkoorden worden later in dit boek veel uitgebreider uitgelegd.

Op de volgende pagina worden verschillende akkoorden gebruikt om arpeggio's te spelen. De vakjes boven de notenbalk laten de vorm van het akkoord zien die op het fretboard gespeeld moet worden. De naam van het akkoord staat boven het fretboardvakje. Je hoeft nu nog niet te weten wat deze akkoorden betekenen. Kijk gewoon naar de noten/tabs om de arpeggio's te spelen.

De cijfers boven de noten geven aan welke vinger van de linkerhand je moet gebruiken. Laat de vingers op de snaren staan zodat de noten blijven doorklinken nadat ze zijn gespeeld.

De letters \textit{p}, \textit{i}, \textit{m} en \textit{a} geven aan welke vingers van de rechterhand gebruikt moeten worden (ervan uitgaande dat je rechtshandig speelt).

Zie \autoref{fig:finger_names} voor een herinnering voor welke letter bij welke vinger hoort.

\includepdf[pages=-,pagecommand={\thispagestyle{headings}}]{../../MuseScore/Guitar/ArpeggiosInCMajor_1.pdf}

\newpage

In het volgende nummer zul je de lage C-noot leren.

\begin{figure}[h]
	\centering
	\includegraphics[height=0.12\textheight]{../../MuseScore/Guitar/GuitarNotesUsedInVogeltjesdans.png}
	\caption{Noten gebruikt voor het nummer "De Vogeltjesdans"}
	\label{fig:guitar_notes_for_vogeltjesdans}
\end{figure}

\infobox{In \autoref{fig:guitar_notes_for_vogeltjesdans} zie je niet alleen de noten die in het nummer worden gebruikt, maar je ziet ook de C-majeur toonladder. Later zullen we meer over toonladders praten.}

\begin{figure}[h]
	\centering
	\includegraphics[width=\textwidth]{../../MuseScore/Guitar/GuitarVogeltjesdansDeElectronicas.png}
	\caption{De vogeltjesdans - De Electronica's}
	\label{fig:guitar_vogeltjesdans}
\end{figure}

\infobox{Hoewel de meeste mensen dit kennen als het Nederlands getitelde "De vogeltjesdans". Het is gebaseerd op het originele nummer genaamd "Der Ententanz" geschreven door Werner Thomas. \cite{DeVogeltjesDansWiki}}

\newpage

In het volgende nummer worden de lage B, A, G, en E noten geïntroduceerd.

\begin{figure}[h]
	\centering
	\includegraphics[height=0.12\textheight]{../../MuseScore/Guitar/GuitarNotesUsedInSevenNationArmy.png}
	\caption{Noten gebruikt voor het nummer "Seven Nation Army"}
	\label{fig:guitar_notes_for_seven_nation_army}
\end{figure}

Voordat we \autoref{fig:guitar_seven_nation_army} spelen. Laten we eens kijken hoe deze noten werken die onder de normale lijnen liggen. Aan het begin van dit hoofdstuk zijn de namen van de noten die overeenkomen met de lijnen van de notenbalk getoond (\autoref{fig:music_note_names_on_staff}). Merk op dat elke lijn en ruimte tussen de lijnen de volgorde "A, B, C, D, E, F, G, A, B, etc." heeft als je omhoog gaat op de lijnen van de notenbalk (en de andere richting als je omlaag gaat op de lijnen van de notenbalk). Deze volgorde gaat gewoon verder onder en boven de normale lijnen van de notenbalk. Lijnen door een noot die boven of onder de hoofdnotenbalk liggen, worden \textbf{hulplijnen} genoemd. Ze helpen om bij te houden hoeveel lijnen de noot onder of boven de hoofdnotenbalk ligt.

Speel het nummer eens met behulp van de tabs die zijn aangegeven in \autoref{fig:guitar_notes_for_seven_nation_army}.

\begin{figure}[h]
	\centering
	\includegraphics[width=\textwidth]{../../MuseScore/Guitar/GuitarSevenNationArmyTheWhiteStripes.png}
	\caption{Seven Nation Army - The White Stripes}
	\label{fig:guitar_seven_nation_army}
\end{figure}

\newpage

Dit nummer (Seven Nation Army) wordt typisch gespeeld met power akkoorden (besproken in een later hoofdstuk). Wanneer je dat doet, is het handiger om de (grondtoon) noten voornamelijk op de 5e en 6e snaar te spelen. \autoref{fig:guitar_seven_nation_army_alternative_tabs} toont alternatieve tabs die handiger zijn bij het spelen van power akkoorden. Probeer dit ooks eens.

\begin{figure}[h]
	\centering
	\includegraphics[width=\textwidth]{../../MuseScore/Guitar/GuitarSevenNationArmyTheWhiteStripes-singlestring-tab.png}
	\caption{Seven Nation Army - The White Stripes (alternatieve tabs)}
	\label{fig:guitar_seven_nation_army_alternative_tabs}
\end{figure}

Om de laatste niet-kruis/moll noot binnen de eerste 3 fretten te introduceren, zullen we het eerste deel van "Californication" van "Red Hot Chili Peppers" spelen. Dit introduceert de lage F-noot.

\begin{figure}[h]
	\centering
	\includegraphics[height=0.12\textheight]{../../MuseScore/Guitar/GuitarNotesUsedInCalifornication.png}
	\caption{Noten gebruikt voor het nummer "Californication"}
	\label{fig:guitar_notes_for_californication}
\end{figure}

Let op de vingerzetting in \autoref{fig:guitar_californication}. In dit stuk, houd je vingers op de fretten voor de duur van de maat nadat je ze hebt gespeeld zodat ze kunnen doorklinken.

\begin{figure}[h]
	\centering
	\includegraphics[width=\textwidth]{../../MuseScore/Guitar/GuitarCalifornicationRedHotChiliPeppers.png}
	\caption{Californication - Red Hot Chili Peppers}
	\label{fig:guitar_californication}
\end{figure}

\newpage

\section{Kruisen en mollen}

Eerder werd al vermeld dat kruisen ($\sharp$) de noot met een halve toon verhogen en mollen ($\flat$) de noot met een halve toon verlagen. Het is ook al genoemd dat kruisen en mollen geldig zijn voor de duur van een maat. Als een noot zijn 'normale' geluid terug moet krijgen, wordt er een herstel ($\natural$) symbool voor geplaatst. Dit maakt de kruis/mol ongedaan voor de rest van de maat.

Wat nog niet is vermeld, is dat een kruis/mol geplaatst op een noot alleen geldig is voor die toonhoogte van de noot (positie op de notenbalk). Zie bijvoorbeeld \autoref{fig:guitar_usage_of_sharps_and_naturals}. Hier zie je dat de eerste G (open derde snaar) een kruis heeft gekregen, en daarom nu een halve toon (1 fret) hoger wordt gespeeld op de 1e fret. De G die een octaaf hoger wordt gespeeld op de eerste snaar is nog steeds een G. Wanneer de G\sharp noot op de derde snaar dan een herstel teken krijgt, wordt het weer de normale G noot die op de open derde snaar wordt gespeeld. Hetzelfde voorbeeld kan worden gegeven voor mollen (\autoref{fig:guitar_usage_of_flats_and_naturals}).

\begin{figure}[h]
	\begin{subfigure}[b]{0.45\textwidth}
		\centering
		\includegraphics[height=0.15\textheight]{../../MuseScore/Guitar/GuitarSharpApplyExample.png}
		\caption{Gebruik van kruisen en hersteltekens}
		\label{fig:guitar_usage_of_sharps_and_naturals}
	\end{subfigure}
	\hfill
	\begin{subfigure}[b]{0.45\textwidth}
		\centering
		\includegraphics[height=0.15\textheight]{../../MuseScore/Guitar/GuitarFlatApplyExample.png}
		\caption{Gebruik van mollen en hersteltekens}
		\label{fig:guitar_usage_of_flats_and_naturals}
	\end{subfigure}
	\caption{Kruisen, mollen en hersteltekens}
\end{figure}

Soms gebruikt een nummer een noot met een mol of kruis heel vaak. Je kan dan zeggen dat het nummer in een bepaalde toonsoort staat (hier komen we later op terug). Het is dan niet gewenst om overal op kruisen/mollen toe te voegen. Dat zou rommelig worden. In plaats daarvan worden de kruisen/mollen van de toonsoort die voor het nummer wordt gebruikt, aan het begin van het stuk weergegeven en zijn ze van toepassing op alle tonen van de noten (tenzij hersteltekens worden gebruikt). Dit wordt een \textbf{sleutel handtekening} genoemd. Een standaardset van sleutel handtekeningen wordt beschreven in een later hoofdstuk over de kwintencirkel (\autoref{sec:circle_of_fifths}).

Zie bijvoorbeeld \autoref{fig:guitar_sharps_at_start_of_music} en \autoref{fig:guitar_flats_at_start_of_music}.

\begin{figure}[h]
	\centering
	\includegraphics[width=\textwidth]{../../MuseScore/Guitar/GuitarKeySharpExample.png}
	\caption{Voorbeeld van het toevoegen van een sleutel handtekening met kruisen}
	\label{fig:guitar_sharps_at_start_of_music}
\end{figure}

\begin{figure}[h]
	\centering
	\includegraphics[width=\textwidth]{../../MuseScore/Guitar/GuitarKeyFlatExample.png}
	\caption{Voorbeeld van het toevoegen van een sleutel handtekening met mollen}
	\label{fig:guitar_flats_at_start_of_music}
\end{figure}

\newpage

Voordat we enkele stukken spelen om de kruisen en mollen te leren, laten we eerst de kruisen en mollen op de notenbalk bekijken:

\begin{figure}[h]
	\centering
	\includegraphics[width=\textwidth]{../../MuseScore/Guitar/PitchesSharpsMultiString.png}
	\caption{Een octaaf van A naar A op de meerdere snaren met kruisen}
	\label{fig:guitar_string_a_octave_multi_string_sharps_chap_music_notation}
\end{figure}

\begin{figure}[h]
	\centering
	\includegraphics[width=\textwidth]{../../MuseScore/Guitar/PitchesFlatsMultiString.png}
	\caption{Een octaaf van A naar A op de meerdere snaren met mollen en hersteltekens}
	\label{fig:guitar_string_a_octave_multi_string_flats_chap_music_notation}
\end{figure}

Vergeet ook niet dat er tussen elke noot, behalve tussen B-C en E-F, twee halve stappen zitten. Tussen B-C en E-F is er maar één halve stap.

\begin{table}[h]
	\centering
	\begin{NiceTabular}{*{12}{P{5mm}}}
		\large{A} & \large{A\sharp} & \large{B} & \large{C} & \large{C\sharp} & \large{D} & \large{D\sharp} & \large{E} & \large{F} & \large{F\sharp} & \large{G} & \large{G\sharp} \\ \\
		\large{A} & \large{B\flat} & \large{B} & \large{C} & \large{D\flat} & \large{D} & \large{E\flat} & \large{E} & \large{F} & \large{G\flat}& \large{G} & \large{A\flat}
	\end{NiceTabular}
	\caption{Kruis en mol intervallen}
	\label{tab:guitar_sharp_flat_intervals}
\end{table}

Vergeet niet dat een kruis en mol de noot gewoon een halve stap omhoog of omlaag verplaatsen. Dus wat zou er gebeuren als de E-noot een $\sharp$ krijgt? Het zou een F worden. En wat is hetzelfde als een F$\flat$? Inderdaad, een E. 

Merk op dat C$\sharp$ en D$\flat$ op dezelfde positie staan in \autoref{tab:guitar_sharp_flat_intervals}. Wanneer twee (of meer) verschillende namen voor dezelfde noot kunnen worden gebruikt, worden deze namen \textbf{enharmonisch} genoemd.

\newpage

Eerder hebben we al Happy Birthday gespeeld zonder kruisen of mollen. Maar de muziek kan worden \textbf{getransponeerd} naar een andere toonsoort. In dit geval naar de toonsoort van G majeur. Zie \autoref{fig:guitar_happy_birthday_sharps}.

\begin{figure}[h]
	\centering
	\includegraphics[width=\textwidth]{../../MuseScore/Guitar/GuitarHappyBirthdaySharps.png}
	\caption{Happy birthday met kruisen}
	\label{fig:guitar_happy_birthday_sharps}
\end{figure}

In \autoref{fig:guitar_cest_la_vie_intro_chorus_melody} staan twee muziek-brede kruisen. De F en de C. Dit is de toonsoort van D majeur. Maar aangezien het refrein begint op de B mineur akkoord, zou je ook kunnen beweren dat het in de toonsoort van B mineur is, die dezelfde sleutel handtekening heeft. Maar dit alles wordt later in een andere sectie besproken.

\begin{figure}[h]
	\centering
	\includegraphics[width=\textwidth]{../../MuseScore/Guitar/GuitarCestLaVieChefSpecial_IntroChorus.png}
	\caption{C'est La Vie - Chef'Special (intro + refrein melodie)}
	\label{fig:guitar_cest_la_vie_intro_chorus_melody}
\end{figure}

In Hedwig's Theme (zie de volgende pagina) zie je het gebruik van kruisen, mollen, hersteltekens en muziek-brede kruisen.

Om je beter te helpen de positie van deze noten te leren, is er een lege tabulatuur toegevoegd. Je kunt deze tabulatuur invullen met de juiste tabs om je te helpen leren.

\includepdf[pages=-,pagecommand={\thispagestyle{headings}}]{../../MuseScore/Guitar/GuitarHarrysPotterHedwigsTheme.pdf}

Het volgende klassieke stuk introduceert de hoge A en B noten (\autoref{fig:guitar_notes_high_a_b}). Eerder was al uitgelegd hoe de noten onder de notenbalk kunnen worden bepaald. Hetzelfde geldt voor noten boven de notenbalk. Je gaat gewoon alfabetisch omhoog over de notenbalk en extra lijnen.

\begin{figure}[h]
	\centering
	\includegraphics[height=0.12\textheight]{../../MuseScore/Guitar/GuitarNotesHighAB.png}
	\caption{De hoge A en B noten}
	\label{fig:guitar_notes_high_a_b}
\end{figure}

Een ander ding dat je in dit nummer zult zien, is dat er twee delen zijn. Eén voor de melodie en één voor de baslijn. Dit nummer is bedoeld om door twee mensen samen te worden gespeeld.

\includepdf[pages=-,pagecommand={\thispagestyle{headings}}]{../../MuseScore/Guitar/GuitarMinuetInG.pdf}

Zoals beloofd, de hele Tetris melodie zou worden gespeeld wanneer we meer zouden leren over kruisen. Dus hier is het (\autoref{fig:guitar_tetris_full}).

Dit introduceert ook de \textbf{D.C. al Fine} term. De term "D.C. al Fine" betekent teruggaan naar het begin van het muziekstuk en spelen totdat je de "Fine" tekst ziet. Dan is de muziek afgelopen. Hier betekent "D.C." "Da Capo" en is Italiaans voor "van het begin".

\begin{figure}[h]
	\centering
	\includegraphics[width=\textwidth]{../../MuseScore/Guitar/GuitarTetrisFull.png}
	\caption{Tetris melodie (volledig)}
	\label{fig:guitar_tetris_full}
\end{figure}

In het nummer "He's a pirate" (zie de volgende pagina) van de "Pirates of the Caribbean" films is er één nieuwe noot. De Hoge C (\autoref{fig:guitar_note_high_c}).

\begin{figure}[h]
	\centering
	\includegraphics[height=0.12\textheight]{../../MuseScore/Guitar/GuitarNotesHighC.png}
	\caption{De hoge C noot}
	\label{fig:guitar_note_high_c}
\end{figure}

Dit nummer heeft een muziek-brede B mol. Dit komt overeen met de D mineur toonsoort. Je ziet dat het nummer zowel begint als eindigt op een D noot. De verschillende melodie delen/loopjes in de nummers beginnen ook vaak en/of eindigen op een D noot. Dit maakt D als het ware een 'thuis' noot.

\textbf{Dit nummer introduceert ook het concept van spelen in een andere positie.} Maat 32 - 39 worden gespeeld vanuit de 3e positie, en van maat 40 tot het einde speel je in de 5e positie. Wat dit betekent is dat je respectievelijk de 3e en 5e fret als 'startpunt' neemt. Stel je voor dat de fretten daarvoor niet bestaan. Dit wordt gedaan om het spelen gemakkelijker/sneller te maken. Dit dwingt dit je om te leren waar identieke noten op het fretbord kunnen worden gespeeld.

Het voordeel van deze posities is dat je niet met je hand over het hele fretbord hoeft te vliegen. In plaats daarvan, door de juiste vingerposities te gebruiken, kun je je hand op één positie houden.

Om je te helpen, denk aan het stemmen van de gitaar door andere snaren te gebruiken (\autoref{fig:guitar_relative_tuning}) en de interval van elke fret (een halve toon) samen met hoe deze stappen zich verhouden tot de verschillende noten (\autoref{sec:fretboard_introduction}). 

\includepdf[pages=-,pagecommand={\thispagestyle{headings}}]{../../MuseScore/Guitar/GuitarHesAPirate.pdf}

TODO: Nummers die triolen, gallops, legato/hammer-on-pull-off, natuurlijke en kunstmatige harmonischen tonen.

% TODO: Maak een voorbeeld voor een triolen nummer (moonlight sonata of star wars theme?)

% TODO: Maak een voorbeeld voor een gallop