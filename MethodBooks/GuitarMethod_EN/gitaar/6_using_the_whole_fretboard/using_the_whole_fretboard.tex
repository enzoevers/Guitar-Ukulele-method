\section{Modes} \label{sec:modes}

\subsection{Diatonic modes}

So far we have seen the major and minor scales. Both the diatonic and pentatonic versions. And all starting from the 6th string. This will allow you to improvise a nice melody over a song.

You have also learned about relative minor and major scales during the circle of fifths. In \autoref{tab:guitar_mode_intervals} you see that by starting on the 6th degree of the major scale (Ionian mode), you get the intervals of the natural minor scale (also called \textnormal{A}eolian mode) while still having the same notes.

So d\textnormal{o}es that mean that you can improvise with the notes of A minor scale over a song in C major. YES!. After all, they are the same notes.

D\textnormal{o}es that mean that C major and A minor are identical. \textbf{No!} That also means that you can only say that a song is in C major, A minor, E phrygian, etc. (which have the same notes) by looking at the \mbox{\textbf{tonal center}} (the tone that is most prominent) of the musical context. Note that the intervals from the root note of each mode is different.

\underline{A mnemonic for the mode name order: \textbf{I} \textbf{D}on't \textbf{P}articularly \textbf{L}ike \textbf{M}odes \textbf{A} \textbf{Lo}t.}

\begin{table}[h]
	\centering
	\begin{NiceTabular}{*{30}{c}}
		\Block{}{Ionian (major)} & \Block{}{} & \Block{1-2}{\large{W}} & & \Block{1-2}{\large{W}} & & \Block{1-2}{\large{H}} & & \Block{1-2}{\large{W}} & & \Block{1-2}{\large{W}} & & \Block{1-2}{\large{W}} & & \Block{1-2}{\large{H}} & & \Block{}{} \\
		\Block{}{\textit{Chords}} & \Block{1-2}{\RomanNumeralCaps{1}} & & \Block{1-2}{\RomanNumeral{2}} & & \Block{1-2}{\RomanNumeral{3}} & & \Block{1-2}{\RomanNumeralCaps{4}} & & \Block{1-2}{\RomanNumeralCaps{5}} & & \Block{1-2}{\RomanNumeral{6}} & & \Block{1-2}{\RomanNumeral{7}\textsuperscript{o}} & & \\
		\Block{}{\textit{Intervals}} & \Block{1-2}{1} & & \Block{1-2}{2} & & \Block{1-2}{3} & & \Block{1-2}{4} & & \Block{1-2}{5} & & \Block{1-2}{6} & & \Block{1-2}{7} & & \\
		\hline \\
		\Block{}{Dorian} & \Block{1-3}{} & & & \Block{1-2}{\large{W}} & & \Block{1-2}{\large{H}} & & \Block{1-2}{\large{W}} & & \Block{1-2}{\large{W}} & & \Block{1-2}{\large{W}} & & \Block{1-2}{\large{H}} & & \Block{1-2}{\large{W}} & & \\
		\Block{}{} & \Block{1-2}{} & & \Block{1-2}{\RomanNumeral{1}} & & \Block{1-2}{\RomanNumeral{2}} & & \Block{1-2}{\RomanNumeralCaps{3}} & & \Block{1-2}{\RomanNumeralCaps{4}} & & \Block{1-2}{\RomanNumeral{5}} & & \Block{1-2}{\RomanNumeral{6}\textsuperscript{o}} & & \Block{1-2}{\RomanNumeralCaps{7}} & & \\
		\Block{}{} & \Block{1-2}{} & & \Block{1-2}{1} & & \Block{1-2}{2} & & \Block{1-2}{3\flat} & & \Block{1-2}{4} & & \Block{1-2}{5} & & \Block{1-2}{6} & & \Block{1-2}{7\flat} & & \\
		\hline \\
		\Block{}{Phrygian} & \Block{1-5}{} & & & & & \Block{1-2}{\large{H}} & & \Block{1-2}{\large{W}} & & \Block{1-2}{\large{W}} & & \Block{1-2}{\large{W}} & & \Block{1-2}{\large{H}} & & \Block{1-2}{\large{W}} & & \Block{1-2}{\large{W}} & & \\
		\Block{}{} & \Block{1-4}{} & & & & \Block{1-2}{\RomanNumeral{1}} & & \Block{1-2}{\RomanNumeralCaps{2}} & & \Block{1-2}{\RomanNumeralCaps{3}} & & \Block{1-2}{\RomanNumeral{4}} & & \Block{1-2}{\RomanNumeral{5}\textsuperscript{o}} & & \Block{1-2}{\RomanNumeralCaps{6}} & & \Block{1-2}{\RomanNumeral{7}} & & \\
		\Block{}{} & \Block{1-4}{} & & & & \Block{1-2}{1} & & \Block{1-2}{2\flat} & & \Block{1-2}{3\flat} & & \Block{1-2}{4} & & \Block{1-2}{5} & & \Block{1-2}{6\flat} & & \Block{1-2}{7\flat} & & \\
		\hline \\
		\Block{}{Lydian} & \Block{1-7}{} & & & & & & & \Block{1-2}{\large{W}} & & \Block{1-2}{\large{W}} & & \Block{1-2}{\large{W}} & & \Block{1-2}{\large{H}} & & \Block{1-2}{\large{W}} & & \Block{1-2}{\large{W}} & & \Block{1-2}{\large{H}} & & \\
		\Block{}{} & \Block{1-6}{} & & & & & & \Block{1-2}{\RomanNumeralCaps{1}} & & \Block{1-2}{\RomanNumeralCaps{2}} & & \Block{1-2}{\RomanNumeral{3}} & & \Block{1-2}{\RomanNumeral{4}\textsuperscript{o}} & & \Block{1-2}{\RomanNumeralCaps{5}} & & \Block{1-2}{\RomanNumeral{6}} & & \Block{1-2}{\RomanNumeral{7}} & & \\
		\Block{}{} & \Block{1-6}{} & & & & & & \Block{1-2}{1} & & \Block{1-2}{2} & & \Block{1-2}{3} & & \Block{1-2}{4\sharp} & & \Block{1-2}{5} & & \Block{1-2}{6} & & \Block{1-2}{7} & & \\
		\hline \\
		\Block{}{Mixolydian} & \Block{1-9}{} & & & & & & & & & \Block{1-2}{\large{W}} & & \Block{1-2}{\large{W}} & & \Block{1-2}{\large{H}} & & \Block{1-2}{\large{W}} & & \Block{1-2}{\large{W}} & & \Block{1-2}{\large{H}} & & \Block{1-2}{\large{W}} & & \\
		\Block{}{} & \Block{1-8}{} & & & & & & & & \Block{1-2}{\RomanNumeralCaps{1}} & & \Block{1-2}{\RomanNumeral{2}} & & \Block{1-2}{\RomanNumeral{3}\textsuperscript{o}} & & \Block{1-2}{\RomanNumeralCaps{4}} & & \Block{1-2}{\RomanNumeral{5}} & & \Block{1-2}{\RomanNumeral{6}} & & \Block{1-2}{\RomanNumeralCaps{7}} & & \\
		\Block{}{} & \Block{1-8}{} & & & & & & & &  \Block{1-2}{1} & & \Block{1-2}{2} & & \Block{1-2}{3} & & \Block{1-2}{4} & & \Block{1-2}{5} & & \Block{1-2}{6} & & \Block{1-2}{7\flat} & & \\
		\hline \\
		\Block{}{\textnormal{A}eolian (natural minor)} & \Block{1-11}{} & & & & & & & & & & & \Block{1-2}{\large{W}} & & \Block{1-2}{\large{H}} & & \Block{1-2}{\large{W}} & & \Block{1-2}{\large{W}} & & \Block{1-2}{\large{H}} & & \Block{1-2}{\large{W}} & & \Block{1-2}{\large{W}} & & \\
		\Block{}{} & \Block{1-10}{} & & & & & & & & & & \Block{1-2}{\RomanNumeral{1}} & & \Block{1-2}{\RomanNumeral{2}\textsuperscript{o}} & & \Block{1-2}{\RomanNumeralCaps{3}} & & \Block{1-2}{\RomanNumeral{4}} & & \Block{1-2}{\RomanNumeral{5}} & & \Block{1-2}{\RomanNumeralCaps{6}} & & \Block{1-2}{\RomanNumeralCaps{7}} & & \\
		\Block{}{} & \Block{1-10}{} & & & & & & & & & &  \Block{1-2}{1} & & \Block{1-2}{2} & & \Block{1-2}{3\flat} & & \Block{1-2}{4} & & \Block{1-2}{5} & & \Block{1-2}{6\flat} & & \Block{1-2}{7\flat} & & \\
		\hline \\
		\Block{}{Locrian} & \Block{1-13}{} & & & & & & & & & & & & & \Block{1-2}{\large{H}} & & \Block{1-2}{\large{W}} & & \Block{1-2}{\large{W}} & & \Block{1-2}{\large{H}} & & \Block{1-2}{\large{W}} & & \Block{1-2}{\large{W}} & & \Block{1-2}{\large{W}} & & \\
		\Block{}{} & \Block{1-12}{} & & & & & & & & & & & & \Block{1-2}{\RomanNumeral{1}\textsuperscript{o}} & & \Block{1-2}{\RomanNumeralCaps{2}} & & \Block{1-2}{\RomanNumeral{3}} & & \Block{1-2}{\RomanNumeral{4}} & & \Block{1-2}{\RomanNumeralCaps{5}} & & \Block{1-2}{\RomanNumeral{6}} & & \Block{1-2}{\RomanNumeral{7}} & & \\
		\Block{}{} & \Block{1-12}{} & & & & & & & & & & & &  \Block{1-2}{1} & & \Block{1-2}{2\flat} & & \Block{1-2}{3\flat} & & \Block{1-2}{4} & & \Block{1-2}{5\flat} & & \Block{1-2}{6\flat} & & \Block{1-2}{7\flat} & & \\
	\end{NiceTabular}
	\caption{Mode intervals}
	\label{tab:guitar_mode_intervals}
\end{table}

\newpage

\subsubsection{Using modes to give a different feeling}

So far we have looked at modes as being a variant of the major scale. Meaning that if a song is in C major, that you can play notes from all the different modes that are derived from C major (D dorian, E phrygian, etc.). Note that this assumes that the tonal center, the note that the songs always wants to come back to, is C. And by playing notes and chords from the C major scale, it gets the major feeling.

To get the feeling of a different mode, you most often want to start on the chord of your mode (so the E minor chord for E phrygian), and then play chords that emphasize the characteristic notes in the mode.

What are the characteristic notes of the mode? For that we first have to determine if a mode is major or minor. This means looking at the 3rd degree to see if it is a major or minor 3rd (3 or 3$\flat$). Next, you compare the mode intervals with either the major or minor intervals and see which are different. This is shown in \autoref{tab:guitar_mode_characteristic_notes}.

\begin{table}[h]
	\centering
	\begin{NiceTabular}{*{8}{c}}
		\textbf{Ionian (major)} & 1 & 2 & 3 & 4 & 5 & 6 & 7 \\
		Lydian (\textit{major}) & 1 & 2 & 3 & \Block[fill=ColorOtherNote]{}{4$\sharp$} & 5 & 6 & 7 \\
		Mixolydian (\textit{major}) & 1 & 2 & 3 & 4 & 5 & 6 & \Block[fill=ColorOtherNote]{}{7$\flat$} \\
		\\
		\textbf{\textnormal{A}eolian (natural minor)} & 1 & 2 & 3$\flat$ & 4 & 5 & 6$\flat$ & 7$\flat$ \\
		Dorian (\textit{minor}) & 1 & 2 & 3$\flat$ & 4 & 5 & \Block[fill=ColorOtherNote]{}{6} & 7$\flat$ \\
		Phrygian (\textit{minor}) & 1 & \Block[fill=ColorOtherNote]{}{2$\flat$} & 3$\flat$ & 4 & 5 & 6$\flat$ & 7$\flat$ \\
		Locrian (\textit{diminished}) & 1 & \Block[fill=ColorOtherNote]{}{2$\flat$} & 3$\flat$ & 4 & \Block[fill=ColorOtherNote]{}{5$\flat$} & 6$\flat$ & 7$\flat$ \\
	\end{NiceTabular}
	\caption{Mode characteristic notes}
	\label{tab:guitar_mode_characteristic_notes}
\end{table}

The next step is to see which chords in the mode that you want to play have the characteristic note. This is important because we want to emphasis the characteristic note of the mode to give the right 'feel'. In \autoref{sec:building_chords_with_diatonic_scale} you learned how to build chords from the diatonic scales. The chords in \autoref{tab:guitar_mode_intervals} are created the same way. With this knowledge, table \autoref{tab:guitar_mode_characteristic_chords} can be created that shows which chords of the mode are characteristic for the mode. This table is limited to the 3-note chords and (maj)7th chords. But you can of course alter a chord to become, for example, sus2 chords which could add the characteristic note in the chord. 

\begin{table}[h]
	\centering
	\begin{NiceTabular}{*{5}{c}}
		\textbf{Ionian (major)} &  \\
		Lydian (\textit{major}) & \RomanNumeralCaps{2} & \RomanNumeral{4}\textsuperscript{o} & \RomanNumeralCaps{5}\textsuperscript{M7} & \RomanNumeral{7}  \\
		Mixolydian (\textit{major}) & \RomanNumeralCaps{1}\textsuperscript{7} & \RomanNumeral{3}\textsuperscript{o} & \RomanNumeral{5} & \RomanNumeralCaps{7} \\
		\\
		\textbf{\textnormal{A}eolian (natural minor)} & \\
		Dorian (\textit{minor}) &  \RomanNumeral{2} & \RomanNumeralCaps{4} & \RomanNumeral{6}\textsuperscript{o} & \RomanNumeralCaps{7}\textsuperscript{M7} \\
		Phrygian (\textit{minor}) &  \RomanNumeralCaps{2} & \RomanNumeralCaps{3}\textsuperscript{7} & \RomanNumeral{4} & \RomanNumeralCaps{6} \\
		Locrian (\textit{diminished}) & all \\
	\end{NiceTabular}
	\caption{Mode characteristic chords}
	\label{tab:guitar_mode_characteristic_chords}
\end{table}

Now when you create a chord progression for a mode, it would be good to use at least one chord listed in \autoref{tab:guitar_mode_characteristic_chords} besides having the first chord of the mode.

\subsubsection{Playing the mode scales}

We have already seen shapes for the major and minor scales. \autoref{fig:guitar_mode_scale_shapes} shows the shape for all the modes. The numbers represent the interval of the note.

\newpage

\begin{figure}[h]
	\begin{subfigure}[b]{0.31\textwidth}
		\centering
		\includegraphics[height=0.175\textheight]{../../Images/IonianScaleShape.png}
		\caption{Ionian}
		\label{fig:guitar_mode_shape_ionian}
	\end{subfigure}
	\hfill
	\begin{subfigure}[b]{0.31\textwidth}
		\centering
		\includegraphics[height=0.175\textheight]{../../Images/DorianScaleShape.png}
		\caption{Dorian}
		\label{fig:guitar_mode_shape_dorian}
	\end{subfigure}
	\hfill
	\begin{subfigure}[b]{0.31\textwidth}
		\centering
		\includegraphics[height=0.175\textheight]{../../Images/PhrygianScaleShape.png}
		\caption{Phrygian}
		\label{fig:guitar_mode_shape_phrygian}
	\end{subfigure}
	
	\vspace{0.5cm}
	
	\begin{subfigure}[b]{0.31\textwidth}
		\centering
		\includegraphics[height=0.175\textheight]{../../Images/LydianScaleShape.png}
		\caption{Lydian}
		\label{fig:guitar_mode_shape_lydian}
	\end{subfigure}
	\hfill
	\begin{subfigure}[b]{0.31\textwidth}
		\centering
		\includegraphics[height=0.175\textheight]{../../Images/MixolydianScaleShape.png}
		\caption{Mixolydian}
		\label{fig:guitar_mode_shape_mixolydian}
	\end{subfigure}
	\hfill
	\begin{subfigure}[b]{0.31\textwidth}
		\centering
		\includegraphics[height=0.175\textheight]{../../Images/AeolianScaleShape.png}
		\caption{\textnormal{A}eolian}
		\label{fig:guitar_mode_shape_aeolian}
	\end{subfigure}
	
	\vspace{0.5cm}
	
	\begin{subfigure}[b]{0.31\textwidth}
		\centering
		\includegraphics[height=0.175\textheight]{../../Images/LocrianScaleShape.png}
		\caption{Locrian}
		\label{fig:guitar_mode_shape_locrian}
	\end{subfigure}
	
	\caption{F$\sharp$ scale in all different modes}
	\label{fig:guitar_mode_scale_shapes}
\end{figure}

\newpage

TODO: Give examples of songs in different modes

\newpage

\subsubsection{Using modes-shapes to open up the fretboard}

Mode-shape is a tricky word that can be misunderstood. Shapes are just that, shapes. The musical context determines the mode.

\autoref{fig:guitar_diatonic_modes_on_guitar} shows the different mode-shapes. In this case we start with F$\sharp$ Ionian (major) scale. But remember that these can be shifted up or down to be in a different key.

The gray frets with numbers on the 6th string show the F$\sharp$ major scale on the 6th string. This is used to show the connection between the starting position of the different modes with the major scale intervals. Because we are looking from the major scale's perspective, the numbers in the frets of a shape also correspond to the intervals in the major scale.

Just as before, the different colors indicate different octaves in the shape. Meaning that each shape covers 2 and a bit octaves.

\infobox
{
	\begin{itemize}
		\item The mode names in \autoref{fig:guitar_diatonic_modes_on_guitar} correspond to the shape in this case.
		\item The relative position of these shapes simply cover the whole fretboard with notes of a certain scale/mode.
		\item All shapes shown in \autoref{fig:guitar_diatonic_modes_on_guitar} are relative to the F$\sharp$ major scale (ionian mode). If they would be relative to the G$\sharp$ dorian mode then the shapes would be the same, but the numbers (intervals) would be different.
	\end{itemize}
}

\newpage

\begin{figure}[h]
	\centering
	\includegraphics[width=0.9\textwidth]{../../Images/guitar_mode_all.png}
	\caption{Diatonic modes on the guitar}
	\label{fig:guitar_diatonic_modes_on_guitar}
\end{figure}

\clearpage

\subsection{Pentatonic modes}
TODO

\newpage

\section{CAGED}

The CAGED system gets its name from the open-chord shapes that it uses (\autoref{fig:guitar_caged_open_chords}). So far these are just chords shown in a certain order such that it spells CAGED.

\begin{figure}[h]
	\centering
	\includegraphics[width=0.95\textwidth]{../../Images/CagedOpenChords.png}
	\caption{Open chord shapes used in the CAGED system}
	\label{fig:guitar_caged_open_chords}
\end{figure}

What makes this order so convenient is the how the (most right) root note of a chord is on the same string as the (most left) root note of the following chord. See the green arrows in \autoref{fig:guitar_caged_open_chord_shaped_root_connnection}.

\begin{figure}[h]
	\centering
	\includegraphics[width=0.95\textwidth]{../../Images/CagedOpenChordShapesRootConnection.png}
	\caption{Connection of the root notes in the CAGED chord shapes}
	\label{fig:guitar_caged_open_chord_shaped_root_connnection}
\end{figure}

The CAGED system can be used to play any chord with any shape. The main F and B barre chords are created this way. See \autoref{fig:guitar_caged_e_shape_to_f} and \autoref{fig:guitar_caged_a_shape_to_b}.

\begin{figure}[h]
	\centering
	\begin{subfigure}{0.37\textwidth}
		\includegraphics[width=\textwidth]{../../Images/CagedEShapeToFChord.png}
		\caption{Creating the F barre chord using the E shape}
		\label{fig:guitar_caged_e_shape_to_f}
	\end{subfigure}
	\hfill
	\begin{subfigure}{0.37\textwidth}
		\includegraphics[width=\textwidth]{../../Images/CagedAShapeToBChord.png}
		\caption{Creating the B barre chord using the A shape}
		\label{fig:guitar_caged_a_shape_to_b}
	\end{subfigure}
	\label{fig:guitar_caged_create_main_f_b_barre_chords}
	\caption{Create the main F and B chords using the CAGED system}
\end{figure}

\newpage

The same can be done with the minor shapes.

\begin{figure}[h]
	\centering
	\includegraphics[width=0.95\textwidth]{../../Images/CagedOpenMinorChords.png}
	\caption{Open minor chord shapes used in the CAGED system}
	\label{fig:guitar_caged_open_minor_chords}
\end{figure}

\begin{figure}[h]
	\centering
	\includegraphics[width=0.95\textwidth]{../../Images/CagedOpenMinorChordShapesRootConnection.png}
	\caption{Connection of the root notes in the CAGED minor chord shapes}
	\label{fig:guitar_caged_open_minor_chord_shaped_root_connnection}
\end{figure}

\begin{figure}[h]
	\centering
	\begin{subfigure}{0.37\textwidth}
		\includegraphics[width=\textwidth]{../../Images/CagedEmShapeToFmChord.png}
		\caption{Creating the Fm barre chord using the Em shape}
		\label{fig:guitar_caged_em_shape_to_fm}
	\end{subfigure}
	\hfill
	\begin{subfigure}{0.37\textwidth}
		\includegraphics[width=\textwidth]{../../Images/CagedAmShapeToBmChord.png}
		\caption{Creating the Bm barre chord using the Am shape}
		\label{fig:guitar_caged_am_shape_to_bm}
	\end{subfigure}
	\label{fig:guitar_caged_create_main_fm_bm_barre_chords}
	\caption{Create the main Fm and Bm chords using the CAGED system}
\end{figure}

You can imagine that the same holds for open diminished/augmented, sus, 7th chords, etc. shapes. The main thing about the CAGED system is that you learn the root notes in chord shapes, and use these as anchor points when playing.

\newpage

\autoref{fig:guitar_caged_c_chord_over_fretboard} demonstrates how the different chord shapes allow you to play the same chord across the whole fretboard.

\begin{figure}[h]
	\centering
	\includegraphics[width=\textwidth]{../../Images/CagedCChordOverFretboard.png}
	\caption{Playing the C chord using all CAGED shapes across the fretboard}
	\label{fig:guitar_caged_c_chord_over_fretboard}
\end{figure}

\infobox{It is important to emphasize that you are not expected to always play the full shapes of each chord. The CAGED system should be used as a map to see where notes of a chord are on the fretboard, using the root notes are anchor points.}

Now try to start with, for example, the open G chord, and then go through all the CAGED shaped. So in order the shapes would be: (open) G, E, D, C, A.

Also try to play, for example, the E shape on the 7th fret, and go downwards (to the head) using the CAGED shapes. The shape order going downward would be: E, G, A. You will see that you don't have enough 'space' when you try to go to the C shape. But you can still play the notes on the 5th, 4th, and 2nd strings to play a partial C shape.

\newpage

\subsection{Examples in songs}

% https://www.youtube.com/watch?v=5nG7a5PdNyA
% Under The Bridge  - RHCP ()
	% Snow (Hey Oh)  - RHCP (E. C, G)
	% Pray for plagues - BMTH (sweep solo)
% all right now - Free
% wind cries mery - jimi hendrix
% The Bannered Mare - D major for F
	% island in the sun - weezer
% From so Far Away - All Shall Perish

\subsubsection{Island In The Sun - Weezer}

A song to use as an example is "Island In The Sun" from "Weezer" (\autoref{fig:guitar_island_in_the_sun_weezer_intro}). This uses the D minor, E minor, D, and E shapes. Note that only the high 3 string of the E and Em shapes are used.

\begin{itemize}
	\setlength\itemsep{0em}
	\item \writechord{Em}: Dm shape
	\item \writechord{Am}: Em shape
	\item \writechord{D}: D shape
	\item \writechord{G}: E shape
\end{itemize}


\begin{figure}[h]
	\centering
	\includegraphics[width=\textwidth]{../../MuseScore/Guitar/IslandInTheSunWeezerIntro.png}
	\caption{Intro of Island In The Sun - Weezer}
	\label{fig:guitar_island_in_the_sun_weezer_intro}
\end{figure}

\newpage

\subsubsection{Snow(Hey Oh) - Red Hot Chili Peppers}

The song "Snow (Hey Oh)" from "Red Hot Chili Peppers" (\autoref{fig:guitar_snow_rhcp_intro}) uses the Em, C, G, and E-first-inversion shapes. But only a part of each shape is played.

\begin{itemize}
	\setlength\itemsep{0em}
	\item \writechord{G#m}: Em shape
	\item \writechord{E} \& \writechord{F#}: C shape
	\item \writechord{B}: G shape
	\item \writechord{F#/A#} (\writechord{F#} first inversion): E first inversion shape
\end{itemize}

\begin{figure}[h]
	\centering
	\includegraphics[width=\textwidth]{../../MuseScore/Guitar/Snow_RHCP_Intro.png}
	\caption{Intro of Snow (Hey Oh) - Red Hot Chili Peppers}
	\label{fig:guitar_snow_rhcp_intro}
\end{figure}