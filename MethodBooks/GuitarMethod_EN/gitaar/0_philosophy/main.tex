\chapter{Philosophy}

The idea of this method is to give you the tools to be creative with music. Concretely this means that instead of saying "The D chord is played using this shape", the following will be said: "A chord is constructed like \textit{this}. So to play a D chord do \textit{this} and you will end up with this shape". Where the "\textit{this}" is some knowledge you will learn.

During the method you will notice that you will see certain constructs/symbols/etc. that you may not know yet and that are not explained directly. This is with intention. The idea is that by exposing you early on to something, while not consciously needing it yet, it is easier to learn the meaning of it later on.

When putting it in steps it looks as follows:

\begin{enumerate}
	\item Expose you to new concepts so you have seen it, but not necessarily understand it yet.
	\item Guided by exercises and songs, explain the previously shown concepts and how they work together.
	\item Understand the theory of the concepts and be able to use them in playing.
	\item Start at 1. again with new concepts.
\end{enumerate}