\section{Can I already play the that song that I like?}

The answer is, "kinda". So far you've made your first sounds and played along with a known song.

By just reading tabs you can get quite far. But you will most likely be stuck with patterns on the fretboard and playing along with songs without really knowing what is going on. And \textbf{if that is sufficient for you, that is completely fine}.

However, by getting familiar with the fretboard and some theory to connect it all, you can be a lot more creative with music.

Later on you will also learn about scales, which allow to you improvise over songs. You will learn about different kind of chords and where you can find these on the fretboard. To know which frets (or chords) to play, you will need to know which notes belong to the frets.

The goals of this chapter is to get you familiar with reading notes on a staff, how to know at what speed a note should be played relative to other notes, and to learn where the notes are on the fretboard.

\section{Music staff}

You have already seen the music staff from \autoref{fig:music_note_names_on_staff} in the previous exercises. But the meaning of it was not explained yet.

\begin{itemize}
	\item The letters A-G on the staff show which line on the staff has which note name.
	\item The notes go up alphabetically per step (starting again with A after G). 
	\item Notes in between the lines nicely spell out \textbf{"FACE"}, making it easy to remember.
	\item Notes on the lines can be remembered with the mnemonic "\textbf{E}very \textbf{G}ood \textbf{B}oy \textbf{D}oes \textbf{F}ine".
	\item The most left symbol (\clefG) is called the G clef. Note that the curl of the G clef (second line from the bottom) indicates where the G is. More specifically, the G4.
	\subitem Note that the G clef shown in \autoref{fig:music_note_names_on_staff} is slightly different from the ones seen in earlier exercises. For guitar notation you sometimes see a little 8 under the clef. This means that all the notes sound an octave lower than how they would sound with a G clef without the 8 below it. So \textbf{the curl of a G clef that has an 8 below it, would indicate a G3 instead of a G4}. However, when a piece is written for guitar, sometimes the little 8 is omitted. But then you can imaging that it is there.
	\item The vertical line in the middle indicates the start/end of a measure.
	\item The thicker vertical line in at the end indicates the end of the piece.
\end{itemize}

\begin{figure}[h]
	\centering
	\includegraphics[width=0.55\textwidth]{../../Images/MusicNotation_MeasureNoteNames.png}
	\caption{Note names on the staff in two measures}
	\label{fig:music_note_names_on_staff}
\end{figure}

\newpage

\section{Counting}

So far we have also only seen one type of note. The quarter note (\quarterNote). However, there are more. See \autoref{fig:note_duration_basic}. In the same time that a whole note is played, two half notes can be played. And in the same duration as a half note, two quarter notes can be played, etc.

The \lilyTimeSignature{4}{4} means that there can fit 4 (top number) quarter notes (bottom number) in a measure.

\textbf{Important}: A whole note (\wholeNote) equals 4 quarter notes (\quarterNote). It does \textbf{not} equal a whole measure. But because the time signatures says that there are 4 quarter notes in a measure, the whole note takes up a whole measure here.\newline

The "\quarterNote = 60" indicates the tempo and says how many notes of the indicated duration fit in a minute (the BPM). In this case 60 quarter notes fit in a minute. So it's 60 BPM Resulting in a quarter note being equal to 1 second. "\quarterNote = 80" would be 80 BPM, and a quarter note would be 0.75 seconds.

\begin{figure}[h]
	\centering
	\includegraphics[width=\textwidth]{../../MuseScore/Guitar/MusicNotation/NoteDurations_Basic.png}
	\caption{Note duration}
	\label{fig:note_duration_basic}
\end{figure}

There are also other time signatures. The top value indicates how many notes of the bottom number's duration fit in a measure. So a \lilyTimeSignature{3}{4} time signature can fit 3 quarter notes per measure. And a \lilyTimeSignature{6}{8} time signature can fit 6 eighth notes per measure. Note that \lilyTimeSignature{3}{4} and \lilyTimeSignature{6}{8} indicate the same duration per measure, but they provide a different feel. This is demonstrated in \autoref{fig:time_signatures}.

In \autoref{fig:time_signatures} you also see a new duration notation. In the first measure with \lilyTimeSignature{6}{8} timing, there are dots next to the notes (\quarterNoteDottedDown). This means that the note has a duration of 1.5x its original duration.

The ">" symbol means that this note should be played with a more powerful accent. The \textbf{bold} numbers above the notes indicate the counting of the notes. A bold number means to put an accent on it, but played less accented then the ones where there is also an ">" symbol.

\begin{figure}[h]
	\centering
	\includegraphics[width=\textwidth]{../../MuseScore/Guitar/MusicNotation/TimeSignature.png}
	\caption{Time signatures}
	\label{fig:time_signatures}
\end{figure}

Remember exercise \autoref{fig:exercise_nothing_else_matters_metallica_intro_pima} (Metallica - Nothing else matters (intro))? That is also in \lilyTimeSignature{6}{8}.

\newpage

Where notes indicate when to play a sound, rests indicate when to be silent. In \autoref{fig:guitar_rests} the most common rest durations are shown.

\begin{figure}[h]
	\centering
	\includegraphics[width=\textwidth]{../../MuseScore/Guitar/GuitarRests.png}
	\caption{Rest notations of different duration}
	\label{fig:guitar_rests}
\end{figure}

In \autoref{fig:guitar_rests_exercise} an exercise is provided to count the rests. Remember to take this slow and to be conscious about the counts. As a help the tempo is set to the 60 quarter notes per minutes (BPM). This way each quarter note is 1 second. But feel free to play it slower.

\begin{figure}[h]
	\centering
	\includegraphics[width=\textwidth]{../../MuseScore/Guitar/GuitarRestsExercise.png}
	\caption{Rest notations of different duration}
	\label{fig:guitar_rests_exercise}
\end{figure}


\newpage

\section{Learning the main notes} \label{sec:learning_main_notes}

As a first tune that uses multiple note durations, and to learn the first notes on the guitar, Jingle bells will be played (\autoref{fig:jingle_bells}). The notes used for this tune are shown in \autoref{fig:notes_in_jingle_bells}.

\begin{figure}[h]
	\centering
	\includegraphics[height=0.12\textheight]{../../MuseScore/Guitar/NotesInJingleBells.png}
	\caption{Notes used in jingle bells}
	\label{fig:notes_in_jingle_bells}
\end{figure}

Now Jingle bells can be played as shown in \autoref{fig:jingle_bells}.

\begin{figure}[h]
	\centering
	\includegraphics[width=\textwidth]{../../MuseScore/Guitar/GuitarJingleBells.png}
	\caption{Jingle bells}
	\label{fig:jingle_bells}
\end{figure}

% Ma Meilleure Ennemie - Stromea
% BOB - NOFX
% Another One Bites The dust - Queen

\newpage

To learn a few more notes, the "Tetris" tune will be played. The notes from \autoref{fig:notes_for_tetris_first_part} are used in this tune. The only new notes are A and B.

\begin{figure}[h]
	\centering
	\includegraphics[height=0.12\textheight]{../../MuseScore/Guitar/NotesUsedInTetris_FirstPart.png}
	\caption{Notes used for the first part of the Tetris tune}
	\label{fig:notes_for_tetris_first_part}
\end{figure}

In \autoref{fig:tetris_simple_first_part} the first part of the Tetris tune is written. The full tune requires to learn about sharps and flats. So we will play the full tune later. The chords above the staff can be ignored for now.

\begin{figure}[h]
	\centering
	\includegraphics[width=\textwidth]{../../MuseScore/Guitar/GuitarTetrisFirstPart.png}
	\caption{First part of the Tetris tune}
	\label{fig:tetris_simple_first_part}
\end{figure}

\infobox{The "Tetris" tune is derived from a Russian folk song called "Korobeiniki", which is based the a similar named poem written by Nikolay Nekrasov. \cite{KorobeinikiWiki}}

\newpage

The song "Shake It Off" from "Taylor Swift" (\autoref{fig:shake_it_off_taylor_swift}) only requires three notes for the whole song. These notes are shown in \autoref{fig:notes_for_shake_it_off_taylor_swift}. This includes the new low G note.

\begin{figure}[h]
	\centering
	\includegraphics[height=0.12\textheight]{../../MuseScore/Guitar/NotesUsedInShakeIfOffTaylorSwift.png}
	\caption{Notes used for "Shake It Off - Taylor Swift"}
	\label{fig:notes_for_shake_it_off_taylor_swift}
\end{figure}

A new symbol is shown. The repeat symbol as seen in \autoref{fig:guitar_repeat_symbol}. When you come to the end of the measure that has the right side of the repeat symbol, you go back to the left repeat symbol. If there is no left repeat symbol (like in \autoref{fig:shake_it_off_taylor_swift}), then you start from the beginning of the song. When you come to the right repeat symbol again, you will just play further this time.

\begin{figure}[h]
	\centering
	\includegraphics[height=0.05\textheight]{../../MuseScore/Guitar/GuitarRepeatSymbol.png}
	\caption{The repeat symbol}
	\label{fig:guitar_repeat_symbol}
\end{figure}

\begin{figure}[h]
	\centering
	\includegraphics[width=\textwidth]{../../MuseScore/Guitar/GuitarShakeItOffTaylorSwift.png}
	\caption{Repeated part for "Shake It Off - Taylor Swift"}
	\label{fig:shake_it_off_taylor_swift}
\end{figure}

\newpage

We have now played all non-sharp/flat notes. But each note can be played in different locations, and with different pitches.

Let's take the melody of "Memory" from the musical "Cats" \autoref{fig:memory_cats}. It uses most of the notes we already learned, but also uses a lower G, F, and E (\autoref{fig:notes_g_f_e_3}).

\begin{figure}[h]
	\centering
	\includegraphics[height=0.12\textheight]{../../MuseScore/Guitar/NotesUsedInMemoryCats.png}
	\caption{The G, F, and G, notes on the 3rd and 4th strings}
	\label{fig:notes_g_f_e_3}
\end{figure}

It also uses a \textbf{new symbol}. The \textbf{tie symbol} (seen to connect notes from measure 5 and 6 in \autoref{fig:memory_cats}). This symbol indicates that the duration of the first note that starts the tie has the summed duration of all consecutive identical notes that are tied together. All tied notes after the note that starts the tie are therefore not played. Only counted.

\begin{figure}[h]
	\centering
	\includegraphics[width=\textwidth]{../../MuseScore/Guitar/GuitarMemoryCats.png}
	\caption{Memory from the musical Cats}
	\label{fig:memory_cats}
\end{figure}

Another song that you know that uses all the notes that you have learned so far is Happy birthday (\autoref{fig:happy_birthday}). 

\begin{figure}[h]
	\centering
	\includegraphics[width=\textwidth]{../../MuseScore/Guitar/GuitarHappyBirthday.png}
	\caption{Happy birthday}
	\label{fig:happy_birthday}
\end{figure}

\newpage

On the next page you see a version of the song "Golden" from the series "KPop Demon Hunters". This song introduces the low D note. The notes used in this song can be seen in \autoref{fig:notes_for_golden_kpop_demon_hunters}.

\begin{figure}[h]
	\centering
	\includegraphics[height=0.12\textheight]{../../MuseScore/Guitar/NotesUsedInGoldenKPopDemonHunters.png}
	\caption{Notes used for "Golden" from the series "KPop Demon Hunters"}
	\label{fig:notes_for_golden_kpop_demon_hunters}
\end{figure}

This song also introduces a extra option for the repeat symbol as seen in \autoref{fig:different_measure_on_second_repeat}. What this means is that the first time that you play the measures between the repeat symbols, you simply play the measures 17, 18, 19, \textbf{20}. But the second time you instead play the measures 17, 18, 19, \textbf{21}.

\begin{figure}[h]
	\centering
	\includegraphics[width=\textwidth]{../../MuseScore/Guitar/GuiterGoldenKPopDemonHunters_Simple_DifferentRepeatMeasure.png}
	\caption{Extra option for the repeat symbol}
	\label{fig:different_measure_on_second_repeat}
\end{figure}

One more thing before playing the song. At some repeat symbols the play count is shown (see \autoref{fig:extra_repeat_count}). Normally you only repeat a section once (so you play it twice). But in this song you play some sections more than two times. In those cases the amount of times that you play the section is indicated.

\begin{figure}[h]
	\centering
	\includegraphics[height=0.05\textheight]{../../MuseScore/Guitar/GuiterGoldenKPopDemonHunters_Simple_MultiRepeat.png}
	\caption{Multiple repeats indication}
	\label{fig:extra_repeat_count}
\end{figure}

\includepdf[pages=-,pagecommand={\thispagestyle{headings}}]{../../MuseScore/Guitar/GuiterGoldenKPopDemonHunters_Simple.pdf}

\newpage

So far only single-note tunes have been played. But sometimes you also want to have some more bass notes ringing through to give a fuller sound. One way to do that is to play arpeggios. An arpeggio is basically playing a chord (group of notes) one note at a time. Chords are explained in a lot more detail in a later section.

On the next page several chords are used to play arpeggios. The boxes above the staff show the shape of the chord that should be played on the fretboard. The name of the chord is stated above the fretboard box. There is no need to known what these chords mean for now. Just look at the notes/tabs to play the arpeggios.

The numbers above the notes indicate the fretting hand finger to use. Keep the fingers on the strings to let the notes ring through after they are played.

The letters \textit{p}, \textit{i}, \textit{m}, and \textit{a} indicate which right hand fingers should be used (assuming right-handed playing).

See \autoref{fig:finger_names} for a reminder about which letter relates to which finger. 

\includepdf[pages=-,pagecommand={\thispagestyle{headings}}]{../../MuseScore/Guitar/ArpeggiosInCMajor_1.pdf}

\newpage

In the following song you will learn the low C note.

\begin{figure}[h]
	\centering
	\includegraphics[height=0.12\textheight]{../../MuseScore/Guitar/GuitarNotesUsedInVogeltjesdans.png}
	\caption{Notes used for the song "De Vogeltjesdans"}
	\label{fig:guitar_notes_for_vogeltjesdans}
\end{figure}

\infobox{In \autoref{fig:guitar_notes_for_vogeltjesdans} you not only see the notes used in the song, but you also see the C major scale. Later on we will talk more about scales.}

\begin{figure}[h]
	\centering
	\includegraphics[width=\textwidth]{../../MuseScore/Guitar/GuitarVogeltjesdansDeElectronicas.png}
	\caption{De vogeltjesdans - De Electronica's}
	\label{fig:guitar_vogeltjesdans}
\end{figure}

\infobox{While most people know this as the Dutch titled "De vogeltjesdans". It is based on the original song called "Der Ententanz" composed by Werner Thomas. \cite{DeVogeltjesDansWiki}}

\newpage

In the next song the low B, A, G, and E notes is introduced.

\begin{figure}[h]
	\centering
	\includegraphics[height=0.12\textheight]{../../MuseScore/Guitar/GuitarNotesUsedInSevenNationArmy.png}
	\caption{Notes used for the song "Seven Nation Army"}
	\label{fig:guitar_notes_for_seven_nation_army}
\end{figure}

Before playing \autoref{fig:guitar_seven_nation_army}. Lets see how these notes work that are below the normal lines. In the beginning of this chapter, the names of the notes that correspond to the lines of the staff where shown (\autoref{fig:music_note_names_on_staff}). Note there that each line and space between the lines had the sequence of "A, B, C, D, E, F, G, A, B, etc." if you go up on the staff lines (and the other direction if you go down on the staff lines). This sequence simply continues below and above the normal staff lines. Lines through a note that are above or below the main staff are called \textbf{ledger lines}. They help to keep track of how many lines the note is below or above the main staff.

Play the song once using the tab positions indicated in \autoref{fig:guitar_notes_for_seven_nation_army}.

\begin{figure}[h]
	\centering
	\includegraphics[width=\textwidth]{../../MuseScore/Guitar/GuitarSevenNationArmyTheWhiteStripes.png}
	\caption{Seven Nation Army - The White Stripes}
	\label{fig:guitar_seven_nation_army}
\end{figure}

\newpage

This song (Seven Nation Army) is typically played with power chords (discussed in a later chapter). When you do that, its more convenient to play the (root) notes mainly on the 5th and 6th string. \autoref{fig:guitar_seven_nation_army_alternative_tabs} shows alternative tab positions which are more convenient when playing power chords. You can give it a try.

\begin{figure}[h]
	\centering
	\includegraphics[width=\textwidth]{../../MuseScore/Guitar/GuitarSevenNationArmyTheWhiteStripes-singlestring-tab.png}
	\caption{Seven Nation Army - The White Stripes (alternative tabs)}
	\label{fig:guitar_seven_nation_army_alternative_tabs}
\end{figure}

To introduce the last non-sharp/flat note withing the first 3 frets, we will play the first part from "Californication" from "Red Hot Chili Peppers". This introduces the low F note.

\begin{figure}[h]
	\centering
	\includegraphics[height=0.12\textheight]{../../MuseScore/Guitar/GuitarNotesUsedInCalifornication.png}
	\caption{Notes used for the song "Californication"}
	\label{fig:guitar_notes_for_californication}
\end{figure}

Note the fingering in \autoref{fig:guitar_californication}. In this piece, keep you fingers on the frets for the duration of the measure after playing them to let them ring through.

\begin{figure}[h]
	\centering
	\includegraphics[width=\textwidth]{../../MuseScore/Guitar/GuitarCalifornicationRedHotChiliPeppers.png}
	\caption{Californication - Red Hot Chili Peppers}
	\label{fig:guitar_californication}
\end{figure}

\newpage

\section{Sharps and flats}

Earlier it was already mentioned that sharps ($\sharp$) increase the note by a half step and flats ($\flat$) decrease the note by a half step. It has also been mentioned that sharps and flats are valid for the duration of a measure. If a note should get its 'normal' sound back, a natural ($\natural$) symbol is placed in front of it. This undoes the sharp/flat for the rest of the measure .

What has not been mentioned yet, is that a sharp/flat placed at a note is valid only for that pitch of the note (position on the staff). See for example \autoref{fig:guitar_usage_of_sharps_and_naturals}. Here you see that the first G (open third string) got a sharp, and is therefore now played a half tone (1 fret) higher on the 1st fret. The G that is played one octave higher on the first string is still a G. When the G\sharp note on the third string then gets a natural sign, it becomes the normal G note again which is played on the open third string. The same example can be given for flats (\autoref{fig:guitar_usage_of_flats_and_naturals}).

\begin{figure}[h]
	\begin{subfigure}[b]{0.45\textwidth}
		\centering
		\includegraphics[height=0.15\textheight]{../../MuseScore/Guitar/GuitarSharpApplyExample.png}
		\caption{Usage of sharps and naturals}
		\label{fig:guitar_usage_of_sharps_and_naturals}
	\end{subfigure}
	\hfill
	\begin{subfigure}[b]{0.45\textwidth}
		\centering
		\includegraphics[height=0.15\textheight]{../../MuseScore/Guitar/GuitarFlatApplyExample.png}
		\caption{Usage of flats and naturals}
		\label{fig:guitar_usage_of_flats_and_naturals}
	\end{subfigure}
	\caption{Sharps, flats and naturals}
\end{figure}

Sometimes a song uses a note with a flat or sharp a lot of times. It can then be considered to be in a certain key (we will come back to that later). It is then not desired to add sharps/flats all over the sheet music. That could get messy. Instead, the sharps/flats of the key used for the song are shown at the beginning of the piece and apply to all pitches of the notes (unless natural symbols are used). This is called a \textbf{key signature}. A standard set of key signatures will be described in a later section about the circle of fifths (\autoref{sec:circle_of_fifths}).

See for example \autoref{fig:guitar_sharps_at_start_of_music} and \autoref{fig:guitar_flats_at_start_of_music}.

\begin{figure}[h]
	\centering
	\includegraphics[width=\textwidth]{../../MuseScore/Guitar/GuitarKeySharpExample.png}
	\caption{Example of adding a key signature with sharps}
	\label{fig:guitar_sharps_at_start_of_music}
\end{figure}

\begin{figure}[h]
	\centering
	\includegraphics[width=\textwidth]{../../MuseScore/Guitar/GuitarKeyFlatExample.png}
	\caption{Example of adding a key signature with flats}
	\label{fig:guitar_flats_at_start_of_music}
\end{figure}

\newpage

Before playing some pieces to learn the sharps and flats, lets first show the sharps and flats on the fretboard again:

\begin{figure}[h]
	\centering
	\includegraphics[width=\textwidth]{../../MuseScore/Guitar/PitchesSharpsMultiString.png}
	\caption{An octave from A to A on the multiple strings using sharps}
	\label{fig:guitar_string_a_octave_multi_string_sharps_chap_music_notation}
\end{figure}

\begin{figure}[h]
	\centering
	\includegraphics[width=\textwidth]{../../MuseScore/Guitar/PitchesFlatsMultiString.png}
	\caption{An octave from A to A on the multiple strings using flats and naturals}
	\label{fig:guitar_string_a_octave_multi_string_flats_chap_music_notation}
\end{figure}

Also remember that between each note, except for B-C and E-F, there is are two half steps. Between B-C and E-F there is only a half step.

\begin{table}[h]
	\centering
	\begin{NiceTabular}{*{12}{P{5mm}}}
		\large{A} & \large{A\sharp} & \large{B} & \large{C} & \large{C\sharp} & \large{D} & \large{D\sharp} & \large{E} & \large{F} & \large{F\sharp} & \large{G} & \large{G\sharp} \\ \\
		\large{A} & \large{B\flat} & \large{B} & \large{C} & \large{D\flat} & \large{D} & \large{E\flat} & \large{E} & \large{F} & \large{G\flat}& \large{G} & \large{A\flat}
	\end{NiceTabular}
	\caption{Sharp and flat intervals}
	\label{tab:guitar_sharp_flat_intervals}
\end{table}

Remember that a sharp and flat simply move the note a half step up or down respectively. So what would happen when the E note gets a $\sharp$? It would become an F. And what is the same as an F$\flat$? An E indeed. 

Note that C$\sharp$ and D$\flat$ are on in same position in \autoref{tab:guitar_sharp_flat_intervals}. When two (or more) different names for the same note can be used, these names are called \textbf{enharmonic}.

\newpage

Previously we have already played Happy Birthday without any sharps or flats. But the music can be \textbf{transposed} to a different key. In this case to the key of G major. See \autoref{fig:guitar_happy_birthday_sharps}.

\begin{figure}[h]
	\centering
	\includegraphics[width=\textwidth]{../../MuseScore/Guitar/GuitarHappyBirthdaySharps.png}
	\caption{Happy birthday with sharps}
	\label{fig:guitar_happy_birthday_sharps}
\end{figure}

In \autoref{fig:guitar_cest_la_vie_intro_chorus_melody} there are two song-wide sharps. The F and the C. This is the key of D major. But since the chorus starts on the B minor chord, you could also argue that it is in the key of B minor which has the same key signature. But this will all be discussed in a later section.

\begin{figure}[h]
	\centering
	\includegraphics[width=\textwidth]{../../MuseScore/Guitar/GuitarCestLaVieChefSpecial_IntroChorus.png}
	\caption{C'est La Vie - Chef'Special (intro + chorus melody)}
	\label{fig:guitar_cest_la_vie_intro_chorus_melody}
\end{figure}

In Hedwig's Theme (see the next page) you will see the usage of sharps, flats, naturals and music-wide sharps. It uses the same music-wide F$\sharp$ as Happy birthday.

To better help you learn the position of these notes, there is an empty tablature added. You can fill this tablature with the correct tabs to help you learn.

\includepdf[pages=-,pagecommand={\thispagestyle{headings}}]{../../MuseScore/Guitar/GuitarHarrysPotterHedwigsTheme.pdf}

The next classical piece introduces the high A and B notes (\autoref{fig:guitar_notes_high_a_b}). Previously it was already explained how the notes below the staff lines can be determined. The same holds for notes above the staff. You simply go alphabetically up the staff and ledger lines.

\begin{figure}[h]
	\centering
	\includegraphics[height=0.12\textheight]{../../MuseScore/Guitar/GuitarNotesHighAB.png}
	\caption{The high A and B notes}
	\label{fig:guitar_notes_high_a_b}
\end{figure}

Another thing you will see in this song is that there are two parts. One for the melody and one for the bass line. This sheet music is meant to be played by two people together.

\includepdf[pages=-,pagecommand={\thispagestyle{headings}}]{../../MuseScore/Guitar/GuitarMinuetInG.pdf}

As promised, the whole Tetris tune would be played when we learned about sharps. So here it is (\autoref{fig:guitar_tetris_full}).

This also introduces the \textbf{D.C. al Fine} term. The "D.C. al Fine" term means to go back to the beginning of the music piece and play until you see the "Fine" text. Then the music is finished. Here "D.C." means "Da Capo" and is Italian for "from the beginning".

\begin{figure}[h]
	\centering
	\includegraphics[width=\textwidth]{../../MuseScore/Guitar/GuitarTetrisFull.png}
	\caption{Tetris tune (full)}
	\label{fig:guitar_tetris_full}
\end{figure}

In the song "He's a pirate" (see the next page) from the "Pirates of the Caribbean" movies there is one new note. The High C (\autoref{fig:guitar_note_high_c}).

\begin{figure}[h]
	\centering
	\includegraphics[height=0.12\textheight]{../../MuseScore/Guitar/GuitarNotesHighC.png}
	\caption{The high C note}
	\label{fig:guitar_note_high_c}
\end{figure}

This song has a song-wide flat B. This corresponds to the D minor key. You see that the song both starts and ends on a D note. The different melody parts/walks in the songs also frequently start and/or stop on a D note. This makes D like a 'home' note.

\textbf{This song also introduces the concept of playing in a different position.} Measure 32 - 39 are played from the 3rd position, and from measure 40 till the end you are playing in the 5th position. What this means is that you take the 3rd and 5th fret respectively as the 'starting' point. Imagine that the frets before that don't exist. This is done to make for easier/quicker playing. In terms of learning, this forces you to learn where identical pitched notes can be played on the fretboard.

The benefit of these positions are that you don't have to fly with your hand all over the fretboard. Instead, by utilizing the correct finger positions you can keep your hand in one position.

To help you, think about the relative tuning diagram from the beginning (\autoref{fig:guitar_relative_tuning}) and the interval of each fret (a semitone) together with how these steps relate to the different notes (\autoref{sec:fretboard_introduction}). 

\includepdf[pages=-,pagecommand={\thispagestyle{headings}}]{../../MuseScore/Guitar/GuitarHesAPirate.pdf}

TODO: Songs that show triplets, gallops, legato/hammer-on-pull-off, natural and artificial harmonics.

% TODO: Create example for triplet song (moonlight sonata or star wars theme?)

% TODO: Create example for a gallop