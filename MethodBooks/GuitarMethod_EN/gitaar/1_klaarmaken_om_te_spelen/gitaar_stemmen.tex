\section{Tuning}

Your guitar needs to be in tune. This means that each string has a certain pitch.

In \autoref{fig:guitar_string_names} you see the names (letters) from the thinnest (\textit{e}) to the thickest (\textit{E}) string.

\begin{figure}[h]
    \centering
    \includegraphics[width=0.8\textwidth]{../../Images/guitar-neck-string-names.png}
    \caption{Names of the guitar strings}
    \label{fig:guitar_string_names}
\end{figure}

A mnemonic is (from the high/thin to the low/thick string):

\begin{minipage}{0.25\textwidth}
    \vspace{3mm}
    \begin{itemize}
        \setlength\itemsep{0em}
        \item[6)] \textbf{E} ddie
        \item[5)] \textbf{A} te
        \item[4)] \textbf{D} ynamite
        \item[3)] \textbf{G} ood
        \item[2)] \textbf{B} ye
        \item[1)] \textbf{e} ddie 
    \end{itemize}
    \vspace{3mm}
\end{minipage}
\hfill
\begin{minipage}{0.7\textwidth}
    \infobox{Note that things is the standard tuning. Sometimes the guitar will be tuned differently. But that will then be explicitly mentioned}
\end{minipage}

\begin{minipage}{0.5\textwidth}
You use a tuner to tune (see \autoref{fig:tuning}). The tuner either gives a note value, and then you have to tune up or down to get the correct note on the screen. Or it shows a string number and you have to get the 'pointer' in the middle.

Be careful with tuning the string up (to a higher pitch). Especially the thinner strings can break if they are too tight.
\end{minipage}
\hfill
\begin{minipage}{0.3\textwidth}
    \centering
    \includegraphics[width=\textwidth]{../../Images/guitar-tuning.jpg}
    \captionof{figure}{Using a tuner on a guitar}
    \label{fig:tuning}
\end{minipage}

Another tuning options relies on the difference in pitch between the strings. In \autoref{fig:guitar_relative_tuning} you see which positions on the neck have the same pitch as the thinner string below it.

\begin{figure}[h]
    \centering
    \includegraphics[width=0.4\textwidth]{../../Images/GuitarRelativeTuning.png}
    \caption{Relative tuning}
    \label{fig:guitar_relative_tuning}
\end{figure}
