\section{The pentatonic scale}

In the last section when we talked about the diminished cord, it was hinted that the pentatonic scale allows you to avoid the restlessness (dissonance) of the tritone interval. But why?

The pentatonic scale means to have 5 (penta) notes in the scale. Meaning that in general, any 5 notes can form a pentatonic scale. But the pentatonic scale that we know and use is known as the major pentatonic scale. This is the major diatonic scale without the 4th and the 7th degree. This removes the tritone interval from the scale. But it also removes all the semitone intervals from the scale.

\infobox{Note that the word 'removed' was used. This doesn't really does justice to the pentatonic scale. Most likely the pentatonic scale pre-dates the major (diatonic) scale. For example, in 2008 a flute was found that was 30-40 thousand years old and which was tuned to the pentatonic scale. Additionally, Pythagoras studied how the pentatonic scale tones occurred naturally in nature. \cite{PentatonicFlute}}

What the pentatonic scale gives us with this is a 'safer' scale to improvise with. By removing these two notes, we remove the chance that a note that we use while improvising would form a tritone with a note that is played in the chord progression for example.

Again, no one is stopping you from going outside the pentatonic scale and be creative. But especially in the beginning, the pentatonic scale is a safe scale for improvising.

\newpage

We will relate the major pentatonic scale to the major (diatonic) scale.

\autoref{tab:guitar_major_pentatic_scale} shows the major pentatonic scale. Here \textbf{W+} means 3 semitones (a whole step (W) + a half step (H)). Note that the indexes are still 1-7 and not 1-5. This is to more easily connect it with the diatonic scales that we've learned in previous chapters.

\begin{table}[h]
	\centering
	\begin{NiceTabular}{*{12}{P{0.05mm}}}
		\Block{}{} & \Block{1-2}{\large{W}} & & \Block{1-2}{\large{W}} & & \Block{1-2}{\large{W+}} & & \Block{1-2}{\large{W}} & & \Block{1-2}{\large{W+}} & & \Block{}{} \\
		\Block{1-2}{1} & & \Block{1-2}{2} & & \Block{1-2}{3} & & \Block{1-2}{5} & & \Block{1-2}{6} & & \Block{1-2}{8} &
	\end{NiceTabular}
	\caption{Major pentatonic scale intervals}
	\label{tab:guitar_major_pentatic_scale}
\end{table}

For the minor pentatonic scale we remove the 2nd and 6th degree for the same reasons. This results in the following intervals.

\begin{table}[h]
	\centering
	\begin{NiceTabular}{*{12}{P{0.05mm}}}
		\Block{}{} & \Block{1-2}{\large{W+}} & & \Block{1-2}{\large{W}} & & \Block{1-2}{\large{W}} & & \Block{1-2}{\large{W+}} & & \Block{1-2}{\large{W}} & & \Block{}{} \\
		\Block{1-2}{1} & & \Block{1-2}{3$\flat$} & & \Block{1-2}{4} & & \Block{1-2}{5} & & \Block{1-2}{7$\flat$} & & \Block{1-2}{8} & 
	\end{NiceTabular}
	\caption{Minor pentatonic scale intervals}
	\label{tab:guitar_minor_pentatic_scale}
\end{table}


That's all nice and well, but how to use this? \autoref{fig:guitar_major_minor_pentatonic_shapes} shows the shape of both the major and minor pentatonic scales. These are basically the same shapes as the diatonic shapes that you learned earlier, but with some notes removed (4 and 7 for major, 2 and 6$\flat$ for minor).

\begin{figure}[h]
	\begin{subfigure}[b]{0.45\textwidth}
		\centering
		\includegraphics[height=0.175\textheight]{../../Images/guitar_major_pentatonic_standard.png}
		\caption{Major pentatonic shape from the 6th string}
		\label{fig:guitar_major_pentatonic_shape}
	\end{subfigure}
	\hfill
	\begin{subfigure}[b]{0.45\textwidth}
		\centering
		\includegraphics[height=0.175\textheight]{../../Images/guitar_major_pentatonic_standard_start_5th_string.png}
		\caption{Major pentatonic shape from the 5th string}
		\label{fig:guitar_major_pentatonic_shape_start_5th_string}
	\end{subfigure}
	
	\vspace{0.5cm}
	
	\begin{subfigure}[b]{0.45\textwidth}
		\centering
		\includegraphics[height=0.175\textheight]{../../Images/guitar_minor_pentatonic_standard.png}
		\caption{Minor pentatonic shape from the 6th string}
		\label{fig:guitar_minor_pentatonic_shape}
	\end{subfigure}
	\hfill
	\begin{subfigure}[b]{0.45\textwidth}
		\centering
		\includegraphics[height=0.175\textheight]{../../Images/guitar_minor_pentatonic_standard_start_5th_string.png}
		\caption{Minor pentatonic shape from the 5th string}
		\label{fig:guitar_minor_pentatonic_shape_start_5th_string}
	\end{subfigure}
	\caption{}
	\label{fig:guitar_major_minor_pentatonic_shapes}
\end{figure}

\subsection{Your turn}

Find a song that you like, find the key, and use the pentatonic scale to improvise over the song. The key can often be found on the internet as well.

Note that some songs may change key during the song. This will result in the pentatonic scale to sound a bit off during those parts. When that happens, you can find the key that was switched to and continue in the pentatonic scale in that key.

Another thing you can try is to search for backing tracks in a certain key on the internet.