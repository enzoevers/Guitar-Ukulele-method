\section{Tuning}

Your Ukulele needs to be in tune. This means that each string has a certain pitch. Even though this is already implied, it is important to note that the relative pitch different per string is important as well.

In \ref{fig:ukulele_string_names} you see the names (letters) of the string.

\begin{figure}[h]
    \centering
    \includegraphics[width=0.5\textwidth]{../Images/UkuleleNeck-StringNames.png}
    \caption{Namen van de ukulele snaren}
    \label{fig:ukulele_string_names}
\end{figure}

A mnemonic is (from string 4 to 1):

	\begin{itemize}
		\setlength\itemsep{0em}
		\item[4)] \textbf{G} ood
		\item[3)] \textbf{C} cooks
		\item[2)] \textbf{E} eat
		\item[1)] \textbf{A} all 
	\end{itemize}


\begin{minipage}{0.5\textwidth}
You use a tuner to tune (see \ref{fig:tuning}). The tuner either gives a note value, and then you have to tune up or down to get the correct note on the screen. Or it shows a string number and you have to get the 'pointer' in the middle.

Be careful with tuning the string up (to a higher pitch). Especially the thinner strings can break if they are too right.
\end{minipage}
\hfill
\begin{minipage}{0.34\textwidth}
	\centering
	% TODO: Change picture to ukulele
	\includegraphics[width=\textwidth]{../Images/guitar-tuning.jpg}
	\captionof{figure}{Tuning \cite{Tuning}}
	\label{fig:tuning}
\end{minipage}

Another tuning options relies on the previously mentioned difference in pitch between the strings. In \ref{fig:ukulele_relative_tuning} you see which positions on the neck have the same pitch the a thinner open string.

\begin{figure}[h]
	\centering
	\includegraphics[width=0.5\textwidth]{../Images/UkuleleRelativeTuning.png}
	\caption{Relative tuning}
	\label{fig:ukulele_relative_tuning}
\end{figure}
